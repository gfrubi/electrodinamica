\chapter{Campos vectoriales, rotores y divergencias}\label{Apcamps}
Todo campo vectorial en 3 dimensiones, continuo y diferenciable (de clase $C^2$), y \textit{que se anula en el infinito m'as r'apidamente que} $r^{-3/2}$, est'a 'unicamente determinado por ``sus fuentes'': su rotor (``densidad de circulaci'on'') y por su divergencia (``densidad de fuente'').

Considere un campo vectorial $\vec V(\vec x)$, definimos su divergencia y su
rotor, respectivamente, por
\begin{equation}
 D(\vec x):=\vec\nabla\cdot\vec V, \qquad \vec C(\vec x):=\vec\nabla\times\vec
V. \label{divrot}
\end{equation}
Entonces es posible descomponer $\vec V(\vec x)$ del modo siguiente:
\begin{equation}
 \vec V(\vec x)=-\vec\nabla\phi+\vec\nabla\times\vec A, \label{decomp1}
\end{equation}
con
\begin{equation}
 \phi(\vec x):=\frac{1}{4\pi}\int_{R^3}\frac{D(\vec x')}{\left|\vec x-\vec
x'\right|}\,dV',
\end{equation}
\begin{equation}
 \vec A(\vec x):=\frac{1}{4\pi}\int_{R^3}\frac{\vec C(\vec x')}{\left|\vec
x-\vec x'\right|}\,dV' .
\end{equation}
En efecto:
\begin{eqnarray}
\vec\nabla\cdot\vec
V&=&\vec\nabla\cdot\left(-\vec\nabla\phi+\vec\nabla\times\vec
A\right) \\
&=&-\nabla^2\phi \\
&=&- \frac{1}{4\pi}\nabla^2\int_{R^3}\frac{D(\vec x')}{\left|\vec x-\vec
x'\right|}\,dV'\\
&=&- \frac{1}{4\pi}\int_{R^3}D(\vec x')\nabla^2\frac{1}{\left|\vec x-\vec
x'\right|}\,dV'\\
&=&- \frac{1}{4\pi}\int_{R^3}D(\vec
x')\left(-4\pi\delta^{(3)}(\vec x-\vec x')\right)\,dV'\\
&=&\int_{R^3}D(\vec x')\delta^{(3)}(\vec x-\vec x')\,dV'\\
&=&D(\vec x).
\end{eqnarray}
Adem'as
\begin{eqnarray}
\vec\nabla\times\vec
V&=&\vec\nabla\times\left(-\vec\nabla\phi+\vec\nabla\times\vec
A\right) \\
&=& \vec\nabla\times(\vec\nabla\times\vec A) \\
&=& \vec\nabla(\vec\nabla\cdot\vec A) -\nabla^2\vec A\\
&=& \frac{1}{4\pi}\int_{R^3}(\vec
C(\vec x')\cdot\vec\nabla)\vec\nabla\left(\frac{1}{\left|\vec
x-\vec x'\right|}\right)\,dV' -\frac{1}{4\pi}\int_{R^3}\vec
C(\vec x')\nabla^2\left(\frac{1}{\left|\vec
x-\vec x'\right|}\right)\,dV' \\
&=& \frac{1}{4\pi}\int_{R^3}(\vec
C(\vec x')\cdot\vec\nabla)\vec\nabla\left(\frac{1}{\left|\vec
x-\vec x'\right|}\right)\,dV' +\int_{R^3}\vec
C(\vec x')\delta^{(3)}\left(\vec x-\vec x'\right)\,dV' \\
&=& \frac{1}{4\pi}\int_{R^3}(\vec
C(\vec x')\cdot\vec\nabla)\vec\nabla\left(\frac{1}{\left|\vec
x-\vec x'\right|}\right)\,dV' +\vec C(\vec x). \label{rotv}
\end{eqnarray}
Probaremos ahora que el primer t'ermino de (\ref{rotv}) es cero si $\vec C$ es
un campo acotado. Usando notaci'on indicial, tenemos que la (componente
$i$-'esima de la) expresi'on en el primer t'ermino de (\ref{rotv}), puede
escribirse como
\begin{eqnarray}
\int_{R^3}C_j'\partial_j\partial_i\frac{1}{\left|\vec
x-\vec x'\right|}\,dV'
&=&\int_{R^3}C_j'\partial'_j\partial'_i\frac{1}{
\left|\vec x-\vec x'\right|}\,dV'\\
&=&\int_{R^3}\left[\partial'_j\left(C_j'\partial'_i\frac{1}{
\left|\vec x-\vec x'\right|}\right)-(\partial'_jC_j')\partial_i'\frac{1}{
\left|\vec x-\vec x'\right|}\right]\,dV'\\
&=&\int_{R^3}\left[\partial'_j\left(C_j'\partial'_i\frac{1}{
\left|\vec x-\vec x'\right|}\right)+0\right]\,dV'\\
&=&\oint_{\partial R^3}C_j'\partial'_i\frac{1}{\left|\vec x-\vec
x'\right|}\,dS'_j\\
\end{eqnarray}
En el infinito ($\partial R^3$), $|\vec x|\rightarrow \infty$, de modo que
$\frac{1}{\left|\vec x-\vec x'\right|}\sim \frac{1}{r}$,
$\partial_i'\frac{1}{\left|\vec x-\vec x'\right|}\sim \frac{1}{r^2}$. Adem'as
$dS\sim r^2d\Omega$ en el mismo l'imite, de modo que
$\partial_i'\frac{1}{\left|\vec x-\vec x'\right|}dS_j\sim d\Omega$. En otras
palabras, la contribuci'on de $\partial_i'\frac{1}{\left|\vec x-\vec
x'\right|}dS_j$ es finita en el infinito espacial. Por lo tanto, la integral
en (\ref{rotv}) se anular'a si $\vec C\rightarrow\vec 0$ en el infinito.

Hemos as'i verificado que, dada la divergencia y el rotor de un campo vectorial
tridimensional, es posible reconstruir el campo original. Adem'as,
todo campo vectorial tridimensional puede ser descompuesto en un t'ermino
derivado de un campo escalar (de rotor nulo) y un t'ermino derivado de un
potencial vectorial (de divergencia nula). Note, sin embargo, que los campos
``potenciales'' $\phi$ \textit{no son 'unicos} (siempre es posible sumar una
constante al potencial escalar $\phi$ y un gradiente de otro campo escalar al
potencial vectorial $\vec A$).

Por otro lado, el vector $\vec V$ calculado usando (\ref{decomp1}) es el 'unico
que tiene divergencia $D$ y rotor $\vec C$. En efecto, suponga que existen 2
campos $\vec V_1$ y $\vec V_1$ que satisfacen (\ref{divrot}). Definiendo $\vec
W:=\vec V_1-\vec V_2$ encontramos
\begin{equation}
 \vec\nabla\cdot\vec W=0, \qquad \vec\nabla\times\vec W=\vec 0. \label{drW}
\end{equation}
La condici'on (\ref{drW}b) implica que es posible escribir $\vec
W=-\vec\nabla\psi$ con una campo escalar $\psi$. Con esto, (\ref{drW}a) implica
que $\psi$ debe satisfacer la ecuaci'on de Laplace $\nabla^2\psi=0$.

Usando la identidad \textit{primera identidad de Green}:
\begin{equation}
 \oint_{\partial V}\psi \vec\nabla\psi\cdot d\vec S\equiv \int_V
\left[\psi\nabla^2\psi+(\vec\nabla\psi)^2\right]\,dV ,
\end{equation}
encontramos:
\begin{equation}
 \oint_{\partial V}\psi \vec\nabla\psi\cdot d\vec S\equiv \int_V
(\vec\nabla\psi)^2\,dV . \label{iG2}
\end{equation}
La integral del lado izquierdo de (\ref{iG2}) se anula en el caso que $V=R^3$
puesto que la familia de campos vectoriales $\vec V$ considerados decrece m'as
r'apido que $r^{-3/2}$ en el infinito\footnote{Como consecuencia $\psi$
decrece m'as r'apido que $r^{-3/2}$ y $\psi \vec\nabla\psi$ decrece m'as
r'apido que $r^{-2}$, mientras que $d\vec S$ aumenta como $r^2$.}. Finalmente,
la anulaci'on del lado derecho de  (\ref{iG2}) requiere que
$\vec\nabla\psi=\vec 0$, de modo que $\psi$ puede a lo sumo ser una constante.
En este caso, sin embargo, $\vec W=-\vec\nabla\psi=\vec 0$, es decir $\vec
V_1=\vec V_2$.

Como consecuencia de lo anterior, podemos establecer los siguientes resultados
para un campo vectorial $\vec V$ que se anula en el infinito m'as
r'apidamente que $r^{-3/2}$:
\begin{itemize}
\item Si la divergencia y el rotor de un campo vectorial son conocidos y si
este campo no tiene fuentes en el infinito, entonces el campo est'a 'unicamente
determinado
\item Si $\vec\nabla\cdot\vec V\neq 0$ y $\vec\nabla\times\vec V=\vec 0$
entonces $\vec V$ puede ser derivado de un campo escalar: $\vec
V=-\vec\nabla\phi$.
\item Si $\vec\nabla\times\vec V\neq \vec 0$ y $\vec\nabla\cdot\vec V= 0$
entonces $\vec V$ puede ser derivado de un campo vectorial: $\vec
V=\vec\nabla\times\vec A$.
\item $\vec{V}$ puede ser descompuesto como una superposici'on (suma) de un
campo libre de rotaci'on y uno libre de fuentes.
\item Si $\vec\nabla\times\vec V=\vec 0$ y $\vec\nabla\cdot\vec V= 0$ \textit{en
una cierta regi'on}, entonces $\vec V$ puede ser derivado de un potencial $\phi$
que satisface $\nabla^2\phi=0$. Decimos que $\vec V$ es \textit{arm'onico} en
dicha regi'on.
\item Si $\vec\nabla\times\vec V=\vec 0$ y $\vec\nabla\cdot\vec V= 0$ \textit{en
todo el espacio}, entonces $\vec V=\vec 0$.
\end{itemize}


\chapter{Sistemas de Unidades}
% \section{Tabla de equivalencias para unidades gaussianas y S.I.}
% \begin{tabular}[c]{|l|lll|}
% \hline
% Corriente &  1 statampere & = & $\frac{10}{c}$ ampere\\\hline
% Carga  &  1 statcoulomb & = & $\frac{10}{c}$ coulomb\\\hline
% voltaje &   1 statvolt & $\approx$ & 300 volt\\\hline
% Resistencia  & 1 statohm & $\approx$ & $9\times10^{11}$ ohm\\\hline
% Resistividad & 1 segundo & $\approx$ & $9\times10^{9}$ ohm $\times$
% metro\\\hline
% Capacitancia  & 1 cent'imetro & $\approx$ & $\frac{1}{9\times10^{11}}$
%farad\\\hline
% Campo el'ectrico  & 1 $\frac{dina}{statcoulomb}$ & $\approx$ &
% $3\times10^4\frac{newton}{coulomb}$\\\hline
% Autoinductancia & 1 stathenry & $\approx$ & $9\times10^{11}$ henry\\\hline
% Inducci'on magn'etica  & 1 gauss & $=$ & $10^{-4}$ tesla\\\hline
% Flujo magn'etico  & 1 maxwell & $=$ & $10^{-8}$ tesla$\times$ m$^2$\\\hline
% Intensidad de campo  & 1 oersted & $=$ &
%$\frac{10^3}{4\pi}\frac{ampere}{metro}$
% \\\hline
% \end{tabular}
% 
% \begin{tabular}
% [c]{|l|l|l|l|l|}\hline
% Cantidad & S'imbolo & MKS &  & CGS\\\hline
% Longitud & $l$ & 1m & 10$^2$ & cm\\\hline
% Masa & $m$ & 1kg & 10$^3$ & g\\\hline
% Tiempo & $t$ & 1s & 1 & s\\\hline
% Frecuencia & $\nu$ & 1Hz & 1 & Hz\\\hline
% Fuerza & $F$ & 1N & 10$^{5}$ & dinas\\\hline
% Energ'ia & $U,W$ & 1J & 10$^{7}$ & erg\\\hline
% Potencia & $P$ & 1W & 10$^{7}$ & $\frac{\text{erg}}{\text{s}}$\\\hline
% Carga & $q$ & $1$coulomb & 3$\times10^{9}$ & statcoulombs\\\hline
% Densidad de Carga & $\rho$ & 1$\frac{\text{coulomb}}{\text{m}^3}$ &
% 3$\times10^3$ & $\frac{\text{statcoulombs}}{\text{cm}^3}$\\\hline
% Corriente & $I$ & 1amp & 3$\times10^{9}$ & statamperes\\\hline
% Densidad de Corriente & $J$ & 1$\frac{\text{amp}}{^{\text{m}^2}}$ &
% 3$\times10^{5}$ & $\frac{\text{statamperes}}{\text{cm}^2}$\\\hline
% Campo El'ectrico & $E$ & 1$\frac{\text{volt}}{\text{m}}$ & $\frac{1}%
% {3}\times10^{-4}$ & $\frac{\text{statvolt}}{\text{cm}}$\\\hline
% Potencial & $\phi$ & 1volt & $\frac{1}{300}$ & statvolt\\\hline
% Polarizaci'on & $P$ & 1$\frac{\text{coul}}{\text{m}^2}$ & 3$\times10^{5}$
% & $\frac{\text{momento dipolar}}{\text{cm}^3}$\\\hline
% Desplazamiento & $D$ & 1$\frac{\text{coul}}{\text{m}^2}$ & 12$\pi
% \times10^{5}$ & $\frac{\text{statvolt}}{\text{cm}}$ 'o $\frac
% {\text{statcoulombs}}{\text{cm}^2}$\\\hline
% Conductividad & $\sigma$ & 1$\frac{\text{mho}}{\text{m}}$ & 9$\times10^{9}$ &
% $\frac{1}{\text{s}}$\\\hline
% Resistencia & $R$ & 1ohm & $\frac{1}{9}10^{-11}$ & $\frac{\text{s}}{\text{cm}%
% }$\\\hline
% Capacitancia & $C$ & 1farad & 9$\times10^{11}$ & cm\\\hline
% Inducci'on Magn'etica & $B$ & 1tesla & 10$^4$ & gauss\\\hline
% Magnetizaci'on & $M$ & 1$\frac{\text{ampere}}{\text{m}}$ & 10$^{-3}$ &
% $\frac{\text{momento magn'etico}}{\text{cm}^3}$\\\hline
% Inductancia & $L$ & 1henry & $\frac{1}{9}\times10^{-11}$ &
% \textquestiondown ?\\\hline
% Acci'on & $S$ & kg$\frac{\text{m}^2}{\text{s}}$ & $10^{7}$ &
% g$\frac{\text{cm}^2}{\text{s}}$\\\hline
% \end{tabular}
Los diferentes sistemas de unidades usados en la teor'ia electromagn'etica pueden entenderse en funci'on de la libertad que existe para definir las unidades de carga y campo el'ectrico y magn'etico. Aqu'i discutiremos algunas posibilidades, que incluyen al sistema S.I. y al gaussiano. Discutiremos c'omo estos sistemas de unidades ``separan'' de forma diferente la magnitud de la carga el'ectrica de la de los campos electromagn'eticos. Las cantidades mec'anicas, sin embargo, no son alteradas por estas diferentes formas de separar cargas y campos.

Para esta discusi'on, es conveniente considerar las cantidades puramente mec'anicas relacionadas con las definiciones b'asicas de las cantidades electromagn'eticas. En particular, la ley de Coulomb establece que la fuerza entre dos cargas es proporcional al producto de cargas e inversamente proporcional al cuadrado de la distancia que las separa:
\begin{equation}
 F_{\rm e}\propto\frac{q q'}{r^2}.
\end{equation} 
Las unidades en que se miden las cargas depende del valor de la constante de proporcionalidad,
\begin{equation}
 \vec{F}_{\rm e}=\alpha_1\,\frac{q q'}{r^2}\hat{r}.
\end{equation} 
Esto determina adem'as las unidades del campo el'ectrico que es dado, \textit{por definici'on}\footnote{En pricipio, es posible definir el campo el'ectrico y sus unidades asociadas introduciendo otra constante a elecci'on, por ejemplo, $\vec{E}:=8 \vec{F}_{\rm e}/q$. No exploraremos aqu'i esta posibilidad.}, por $\vec{E}:=\vec{F}_{\rm e}/q$. De esta forma:
\begin{equation}
 [q]=[\alpha_1]^{-1/2}[F]^{1/2}L =[\alpha_1]^{-1/2}M^{1/2}L^{3/2}T^{-1},
\end{equation}
\begin{equation}
 [E]=[\alpha_1]^{1/2}[F]^{1/2}L^{-1}=[\alpha_1]^{1/2}M^{1/2}L^{-1/2}T^{-1}.
\end{equation}
Por otro lado, la fuerza magn'etica entre cargas en movimiento (corrientes) est'an dadas, por ejemplo, por la ley de Biot-Savart:
\begin{equation}
 dF_{\rm m}\propto II' dx \times dx' \frac{1}{r^2}.
\end{equation}
Aqu'i debemos introducir nuevamente una constante, tal que,
\begin{equation}
 dF_{\rm m}=\alpha_2\, II'dx \times dx' \frac{1}{r^2}. \label{dfe}
\end{equation}
El campo magn'etico, es \textit{definido} tal que
\begin{equation}
 d\vec{F}_{\rm m}\propto I'd\vec{x}' \times d\vec{B}.
\end{equation}
En particular, existen sistemas de unidades que difieren en la constante de proporcionalidad de esta definici'on. Introducimos entonces otra constante de modo que
\begin{equation}\label{dFa3}
 d\vec{F}_{\rm m}=\alpha_3\,I'd\vec{x}' \times d\vec{B},
\end{equation}
y as'i
\begin{equation}
 d\vec{B}=\frac{\alpha_2}{\alpha_3} \frac{Id\vec{x}\times\hat{r}}{r^2}.
\end{equation}
Esto implica que
\begin{equation}
[B]=[\alpha_1]^{-1/2}[\alpha_2][\alpha_3]^{-1}M^{1/2}L^{1/2}T^{-2}.
\end{equation}
De la expresi'on (\ref{dFa3}) encontramos, equivalentemente, que la fuerza que ejerce un campo magn'etico sobre cargas puntuales es dado por
\begin{equation}
\vec{F}_{\rm m}=\alpha_3\, q\,\vec{v}\times\vec{B},
\end{equation} 
de modo que la fuerza de Lorentz adopta la forma
\begin{equation}
 \vec{F}=q\left(\vec{E}+\alpha_3\,\vec{v}\times\vec{B}\right).
\end{equation} 
Usando (\ref{dfe}) encontramos en particular que la fuerza entre dos l'ineas de corriente $I$ e $I'$, de largo $L$ y separadas una distancia $d$ es
\begin{equation}
 F=2\alpha_2\,II' \frac{L}{d}.
\end{equation} 
A partir de estas definiciones b'asicas es posible encontrar la forma de las ecuaciones de Maxwell correspondientes, de modo que se satisfaga la ley de Faraday y la ecuaci'on de continuidad. As'i obtenemos:
\begin{equation}
 \vec{\nabla}\cdot\vec{E}=4\pi\alpha_1\rho,
\end{equation} 
\begin{equation}
 \vec{\nabla}\cdot\vec{B}=0,
\end{equation} 
\begin{equation}
 \vec{\nabla}\times\vec{E}+\alpha_3\frac{\partial\vec{B}}{\partial t}=\vec{0},
\end{equation} 
\begin{equation}
 \vec{\nabla}\times\vec{B}=4\pi\frac{\alpha_2}{\alpha_3}\vec{J}+\frac{\alpha_2}{\alpha_1\alpha_3}\frac{
\partial\vec{E}}{\partial t}.
\end{equation} 
En una regi'on libre de fuentes podemos derivar la ecuaci'on de onda para las componentes del campo el'ectrico, obteniendo
\begin{equation}
 \left(\frac{\alpha_2}{\alpha_1}\frac{\partial^2\
}{\partial t^2}-\nabla^2\right)\vec{E} =\vec {0},
\end{equation}
de donde encontramos que la velocidad de la luz es dada por la relaci'on
 \begin{equation}
\frac{\alpha_2}{\alpha_1}=\frac{1}{c^2}.
 \end{equation} 
Ya que $c$ es una cantidad mec'anica, su valor no debe depender del sistema de unidades usado para las cantidades electromagn'eticas. Esto significa que s'olo 2 de las 3 constantes pueden elegirse arbitrariamente, por ejemplo, $\alpha_2$ y $\alpha_3$, mientras que $\alpha_1=\alpha_2c^2$. A continuaci'on resumiremos los dos sistemas de unidades m'as com'unmente usados.

\section{Sistema Internacional de unidades (S.I.)}
En este sistema de unidades se define\footnote{Ver, por ejemplo, la p'agina correspondiente del Instituto Nacional de Est'andars y Tecnolog'ia de Estados Unidos (NIST): \url{http://physics.nist.gov/cuu/Units/ampere.html}.} un Ampere como ``la intensidad de una corriente constante que manteni'endose en dos conductores paralelos, rectil'ineos, de longitud infinita, de secci'on circular despreciable y situados a una distancia de un metro uno de otro en el vac'io, produce una fuerza igual a $2\times 10^{-7}$N por metro de longitud". Esto implica que
\begin{equation}
\alpha_2\stackrel{!}{=} 10^{-7}NA^{-2}=:\frac{\mu_0}{4\pi},
\end{equation}
\begin{equation}
\alpha_3\stackrel{!}{=} 1,
\end{equation}
y adem'as se denota
\begin{equation}
\alpha_1=:\frac{1}{4\pi\varepsilon_0}.
\end{equation}
Por lo tanto, la velocidad de la luz es dada por
\begin{equation}
c=\frac{1}{\sqrt{\varepsilon_0\mu_0}}.
\end{equation}
La tabla \ref{TUSI} resume las ecuaciones de Maxwell en este sistema de unidades y las unidades en que son medidas cada cantidad. 
\begin{table}
\begin{center}
\begin{tabular}{|l|l|}\hline
$\varepsilon_0\vec{\nabla}\cdot\vec{E}=\rho$ & $\left[ \vec{E}
\right] =\frac{\mathbf{N}}{\mathbf{C}}=\frac{\mathbf{V}}{\mathbf{m}}$\\\hline
$\frac{1}{\mu_0}\vec{\nabla}\times \vec{B}-\varepsilon_0\frac{\partial
\vec{E}}{\partial t}= \vec{J}$ &
$\left[\vec{B}\right]=\frac{\mathbf{N}\mathbf{s}}{\mathbf{C}\mathbf{m}}
=\frac{\mathbf{kg}}{\mathbf{C}\mathbf{s}}$\\\hline
$\vec{\nabla}\times \vec{E}+\frac{\partial\vec{B}}{\partial t}=0$ &
$\left[\varepsilon_0\right]=\frac{\mathbf{C}}{\mathbf{m}\mathbf{V}}$\\\hline
$\vec{\nabla}\cdot\vec{B}=0$ & $\left[\mu_0\right]=\frac{\mathbf{N}\mathbf{s}^2}
{\mathbf{C}^2}$\\\hline
$\vec{F}=q\left(\vec{E}+\vec{v}\times\vec{B}\right) $ &
$\left[\vec{A}\right]=\frac{\mathbf{N}}{\mathbf{A}}$\\\hline
$\vec{E}=-\vec{\nabla}\phi-\frac{\partial\vec{A}}{\partial t}$ &
$\left[F\right]=\mathbf{N}$\\\hline
$\vec{B}=\vec{\nabla}\times \vec{A}$ & $\left[\phi\right]=\mathbf{V}$ \\\hline
\end{tabular}
\label{TUSI}
\caption{Resumen Sistema Internacional de Unidades.}
\end{center}
\end{table}


\section{Sistema Gaussiano de Unidades (C.G.S.)}\label{appunid}
En este caso, se impone que
\begin{equation}
 \alpha_1\stackrel{!}{=} 1, \qquad \alpha_3\stackrel{!}{=} \frac{1}{c},
\end{equation}
y como consecuencia
\begin{equation}
\alpha_2=\frac{1}{c^2}.
\end{equation} 
Como resultado, las cargas el'ectricas as'i como los campos electromagn'eticos necesariamente se miden en unidades con exponentes fraccionarios de las magnitudes b'asicas de longitud, tiempo y masa:
\begin{equation}
 \left[q\right]=M^{1/2}L^{3/2}T^{-1}, \qquad
\left[E\right]=\left[B\right]=M^{1/2}L^{-1/2}T^{-1}.
\end{equation} 
Una ventaja relativa de este sistema es que los campos el'ectricos y magn'eticos tienen las mismas unidades. Otra es que en las ecuaciones de Maxwell s'olo aparece \textit{una constante fundamental}, la velocidad de la luz.
\begin{table}
\begin{center}
\begin{tabular}{|c|}\hline
$\vec{\nabla}\cdot\vec{E}=4\pi\rho$ \\\hline
$\vec{\nabla}\times \vec{B}-\frac{1}{c}\frac{\partial\vec{E}}{\partial t}=
\frac{4\pi}{c}\vec{J}$ \\\hline
$\vec{\nabla}\times \vec{E}+\frac{1}{c}\frac{\partial\vec{B}}{\partial t}=0$
\\\hline
$\vec{\nabla}\cdot\vec{B}=0$\\\hline
$\vec{F}=q\left(\vec{E}+\frac{\vec{v}}{c}\times \vec{B}\right)$\\\hline
$\vec{E}=-\vec{\nabla}\phi-\frac{1}{c}\frac{\partial \vec{A}}{\partial
t}$\\\hline
$\vec{B}=\vec{\nabla}\times \vec{A}$\\\hline
\end{tabular}
\label{TUSG}
\caption{Resumen Sistema Gaussiano de Unidades.}
\end{center}
\end{table}

\section{Conversi'on de magnitudes S.I. a Gaussianas}
\begin{equation}
\vec{E}_{\rm SI}=\frac{1}{\sqrt{4\pi\varepsilon_0}}\vec{E}_{\rm G}, \qquad 
\vec{B}_{\rm SI}= \frac{1}{c\sqrt{4\pi\varepsilon_0}}\vec{B}_{\rm G}, \qquad 
q_{\rm SI}=\sqrt{4\pi\varepsilon_0}\,q_{\rm G}
\end{equation}
\begin{equation}
\vec{A}_{\rm SI}=\frac{1}{c\sqrt{4\pi\varepsilon_0}}\,\vec{A}_{\rm G}, \qquad
\phi_{\rm SI}=\frac{1}{\sqrt{4\pi\varepsilon_0}}\,\phi_{\rm G}.
\end{equation}
%\begin{center}
%\begin{tabular}{|lcl|}\hline
%S.I. &$\rightarrow$& Gauss.\\\hline\hline
%$\vec{E}$ & $\rightarrow$& $\frac{\vec{E}}{\sqrt{4\pi\varepsilon_0}}$\\\hline
%$\vec{B}$ & $\rightarrow$& $\frac{\vec{B}}{c\sqrt{4\pi\varepsilon_0}}$\\\hline
%$q$ & $\rightarrow$ & $\sqrt{4\pi\varepsilon_0}q$\\\hline
%$\vec{A}$ & $\rightarrow$ & $\frac{\vec{A}}{c\sqrt{4\pi\varepsilon_0}}$\\\hline
%$\phi$ & $\rightarrow$ & $\frac{\phi}{\sqrt{4\pi\varepsilon_0}}$\\\hline
%$\vec{J}$ & $\rightarrow$ & $ \sqrt{4\pi\varepsilon_0}\vec{J}$\\\hline
%$\rho$ & $\rightarrow$ & $\sqrt{4\pi\varepsilon_0}\rho$\\\hline
%\end{tabular}
%\end{center}

\chapter{Constantes F'isicas (S.I.)}\label{app:constantes}
\begin{center}
\begin{tabular}{||l|lll||}
\hline
{\bf Nombre}&{\bf S'imbolo}&{\bf Valor}&{\bf Unidad}\\
\hline
\hline
N'umero $\pi$                 &$\pi$&3.1415926535\dots&\\
N'umero e                     &e    &2.7182818284\dots&\\
\hline
Carga elemental            &$e$&$1.60217733\cdot10^{-19}$&C\rule{0pt}{13pt}\\
Constante gravitacional  &$G$&$6.67259\cdot10^{-11}$&m$^3$kg$^{-1}$s$^{-2}$\\
Constante de Estructura fina &$\alpha=e^2/2hc\varepsilon_0$&$\approx1/137$&\\
Rapidez de la luz en el vac'io    &$c$&$2.99792458\cdot10^8$&m/s (def)\\
Permitividad del vac'io   &$\varepsilon_0$&$8.854187\cdot10^{-12}$&F/m\\
Permeabilidad del vac'io &$\mu_0$&$4\pi\cdot10^{-7}$&H/m (def)\\
$(4\pi\varepsilon_0)^{-1}$   &&$8.9876\cdot10^9$&Nm$^2$C$^{-2}$\\
\hline
Constante de Planck           &$h$&$6.6260755\cdot10^{-34}$&Js\rule{0pt}{13pt}\\
Constante de Dirac             &$\hbar=h/2\pi$&$1.0545727\cdot10^{-34}$&Js\\
Magnet'on de Bohr              &$\mu_{\rm B}=e\hbar/2m_{\rm e}$&$9.2741\cdot10^{-24}$&Am$^2$\\
Radio de Bohr                 &$a_0$&$0.52918$&\AA\\
Constante de Rydberg     &$Ry$&13.595&eV\\
Longitud de Compton del electr'on  &$\lambda_{\rm Ce}=h/m_{\rm e} c$&$2.2463\cdot10^{-12}$&m\\
Longitud de Compton del prot'on   &$\lambda_{\rm Cp}=h/m_{\rm p}c$&$1.3214\cdot10^{-15}$&m\\
%Masa reducida del 'atomo de Hidr'ogeno &$\mu_{\rm H}$&$9.1045755\cdot10^{-31}$&kg\\
%\hline
%Constante de Stefan-Boltzmann &$\sigma$&$5.67032\cdot10^{-8}$&Wm$^{-2}$K$^{-4}$\rule{0pt}{13pt}\\
%Constante de Wien   &$k_{\rm W}$&$2.8978\cdot10^{-3}$&mK\\
%\hline
%Constante molar de los gases         &$R$&8.31441&J/mol\\
%N'umero de Avogadro          &$N_{\rm A}$&$6.0221367\cdot10^{23}$&mol$^{-1}$\\
%Constante de Boltzmann         &$k=R/N_{\rm A}$&$1.380658\cdot10^{-23}$&J/K\\
%\hline
Masa del electr'on     &$m_{\rm e}$&$9.1093897\cdot10^{-31}$&kg\rule{0pt}{13pt}\\
Masa de prot'on        &$m_{\rm p}$&$1.6726231\cdot10^{-27}$&kg\\
Masa de neutr'on      &$m_{\rm n}$&$1.674954\cdot10^{-27}$&kg\\
% Unidad elemental de masa     & $m_{\rm u}=\frac{1}{12}m(^{12}_{~6}$C)&$1.6605656\cdot10^{-27}$&kg\\
% Magnet'on nuclear         &$\mu_{\rm N}$&$5.0508\cdot10^{-27}$&J/T\\
\hline
Di'ametro del Sol       &$D_\odot$&$1392\cdot10^6$&m\rule{0pt}{13pt}\\
Masa del Sol             &$M_\odot$&$1.989\cdot10^{30}$&kg\\
Periodo de rotaci'on del Sol &$T_\odot$&25.38&d'ias\\
Radio de la Tierra     &$R_{\rm A}$&$6.378\cdot10^6$&m\\
Masa de la Tierra     &$M_{\rm A}$&$5.976\cdot10^{24}$&kg\\
Periodo de rotaci'on de la Tierra   &$T_{\rm A}$&23.96&hours\\
Periodo orbital Terrestre        &A~no tropical&365.24219879&d'ias\\
Unidad Astron'omica            &AU&$1.4959787066\cdot10^{11}$&m\\
A~no Luz                   &ly&$9.4605\cdot10^{15}$&m\\
Parsec                       &pc&$3.0857\cdot10^{16}$&m\\
% Constante de Hubble &$H$&$\approx(75\pm25)$&km$\cdot$s$^{-1}\cdot$Mpc$^{-1}$\\
\hline
\end{tabular}
\end{center}
