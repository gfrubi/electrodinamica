\chapter{Formulación Covariante de la Electrodinámica}

\section{Conservación de la carga eléctrica y ecuación de continuidad}\label{ccer}

En el vacío (es decir, en ausencia de un medio material) las ecuaciones de Maxwell pueden escribirse ($c^2=1/\varepsilon_0\mu_0$) como:
\begin{equation}\label{inrho}
\vec{\nabla}\cdot \vec{E} = \mu_0 c^2\rho ,
\end{equation}
\begin{equation}\label{inj}
\vec{\nabla}\times \vec{B} = \frac{1}{c^2} \frac{\partial \vec{E}}{\partial t} +\mu_0\vec{J} ,
\end{equation}
\begin{equation}\label{db0}
\vec{\nabla}\cdot \vec{B} =0 ,
\end{equation}
\begin{equation}\label{de+}
\vec{\nabla}\times \vec{E} = - \frac{\partial \vec{B}}{\partial t} .
\end{equation}
Estas ecuaciones se complementan con la expresión para la fuerza de Lorentz que, para una carga (puntual) $q$ moviéndose con velocidad $\vec{v}$, asume la forma
\begin{equation}
\vec{F} = q \left( \vec{E} + \vec{v}\times \vec{B}\right) .
\end{equation}
Las ecuaciones de Maxwell implican la ley de conservación de la carga eléctrica. Usando (\ref{inj}) y (\ref{inrho}) podemos escribir:
\begin{eqnarray}
\vec{\nabla}\cdot (\vec{\nabla}\times \vec{B}) &=& \frac{1}{c^2} \vec{\nabla}\cdot\frac{\partial \vec{E}}{\partial t} + \mu_0 \vec{\nabla}\cdot \vec{J}\\
 &=& \frac{1}{c^2} \frac{\partial (\vec{\nabla}\cdot \vec{E})}{\partial t} +
\mu_0 \vec{\nabla}\cdot \vec{J}\\
&=&\mu_0\frac{\partial \rho}{\partial t} + \mu_0
\vec{\nabla}\cdot \vec{J} ,
\end{eqnarray}
de modo que obtenemos la \textit{ecuación de continuidad} \eqref{eccont}.
%que expresa la conservación de la carga eléctrica ($\int_V \rho dV=\rm cte$
%si $\vec{J}=\vec{0}$ en la superficie $\partial V$).

Deseamos asegurar que la teoría electromagnética sea compatible con los
principios de la teoría de RE y, en particular, con el Principio de Relatividad. Para esto estudiemos cómo deben cambiar los campos bajo cambios de SRI, de modo que se asegure que las ecuaciones de Maxwell sean válidas \textit{con la misma forma} en todo SRI. Esto puede ser realizado si logramos escribir las ecuaciones de Maxwell en términos de 4-vectores y/o 4-tensores.

Por simplicidad, comenzaremos por la ecuación de continuidad (\ref{eccont}). Para asegurar que la ley de conservación de la carga eléctrica sea válida en todo SRI la
ecuación (\ref{eccont}) debe ser covariante bajo TL's.  Dado que
$c^{-1}\partial_t$ y $\vec{\nabla}$ son las componentes de
$\partial_\mu=(c^{-1}\partial_t, \vec\nabla)$, que transforma
como un 4-vector (operador) covariante bajo TL's, podemos escribir
(\ref{eccont}) como
\begin{equation}
\partial_\mu J^\mu=\frac{\partial(c \rho)}{\partial x^0} + \partial_i J^i =0
,\label{dj0}
\end{equation}
donde hemos denotado $J^0:=c \rho $, 
%$J^1:=J_{x}$, $J^2:=J_{y}$, $J^3:=J_{z}$,
de modo que $J^\mu :=(c \rho,J^i)=(c \rho,\vec{J})$.

La invariancia de forma de la ecuación de continuidad estará asegurada si el conjunto de 4 cantidades $J^\mu$ \textit{transforma como un vector contravariante bajo TL's}. Llamamos al 4-vector $J^\mu$ la \textit{4-densidad de corriente} o simplemente la \textit{4-corriente}. En efecto, si en un SRI $K'$ tenemos que $x'^\mu=\Lambda^\mu_{\ \nu}x^\nu$ entonces $J'^\mu=\Lambda^\mu_{\ \nu}J^\nu$ además de $\partial'_\mu=(\Lambda^{-1})^\nu_{\  \mu}\partial_\nu$, de modo que $\partial'_\mu J'^\mu=\partial_\mu J^\mu$ (en otras palabras ``la 4-divergencia de la 4-corriente es un escalar") y por consiguiente $\partial'_\mu J'^\mu=0$.

El asumir que las componentes de la 4-corriente (la densidad y las componentes de la densidad de corriente) forman un 4-vector bajo TL's es consistente con la conocida expresión $\vec{J}=\rho\vec{v}$, que es válida para la densidad de corriente de una
distribución con densidad de carga $\rho$ moviéndose con velocidad $\vec{v}$.
En efecto, si en un SRI $K_0$, comóvil con (un punto dado de) la distribución de carga, tenemos que $\vec{v}_0=\vec{0}$ y la densidad de carga es $\rho_0$ entonces, \textit{asumiendo} que $\vec{J}^\mu$ es un 4-vector tendremos, usando el boost (\ref{bg1})-(\ref{bg2}), que en otro SRI en el que las cargas se mueven con velocidad $\vec\beta c$ las componente de la 4-corriente serán
\begin{equation}
\vec{J}=\gamma\vec\beta J^0=\gamma\vec\beta c \rho_0, \qquad  c\rho=\gamma J^0=\gamma c
\rho_0,
\end{equation}
de modo que $\vec{J}=\rho\vec{v}$. Note que el cambio en la densidad de
carga (carga por unidad de volumen), $\rho=\gamma\rho_0$ es consistente con la contracción de longitudes (y por tanto, de volumen) en la dirección de movimiento.

Puede verificarse que si $J^\mu$ es un 4-vector
contravariante, entonces \textit{la carga total en un volumen} $V$, $Q=\int_V (J^0/c)dV$ \textit{es un escalar bajo TL's}. Alternativamente, podemos justificar el carácter
vectorial de la \textit{4-densidad de corriente} $J^\mu$ exigiendo (inspirado en la información experimental disponible) que la carga eléctrica sea un escalar.

Note además que para el caso de una distribución de carga con densidad $\rho$ y 4-velocidad $u^\mu$ podemos escribir la 4-densidad de corriente $J^\mu$ como
\begin{equation}
\boxed{J^\mu=\rho_0 u^\mu,}
\end{equation}
donde $\rho_0:={\rho}/{\gamma}$ es un \textit{escalar} que describe la \textit{densidad de carga en el SRI comóvil con la distribución}: la \textit{densidad propia de carga}.


\section{Ecuaciones de Maxwell inhomogéneas}

Sabiendo que $J^\mu$ es un 4-vector, podemos analizar las propiedades de transformación de los campos eléctrico y magnético de modo que las ecuaciones de Maxwell inhomogeneas preserven su forma bajo TL's. Podemos escribir (\ref{inrho}) y (\ref{inj})  más explícitamente como:
\begin{eqnarray}
\frac{1}{c}\frac{\partial E^x}{\partial x} + \frac{1}{c}\frac{\partial E^y}{\partial y} + \frac{1}{c}\frac{\partial E^z}{\partial z} &=& \mu_0\, J^0 , \\
\frac{\partial B^z}{\partial y} - \frac{\partial B^y}{\partial z} - \frac{1}{c^2}\frac{\partial E^x}{\partial t} &=& \mu_0\,  J^1 , \\
\frac{\partial B^x}{\partial z} - \frac{\partial B^z}{\partial x} - \frac{1}{c^2}\frac{\partial E^y}{\partial t} &=& \mu_0\,  J^2 , \\
\frac{\partial B^y}{\partial x} - \frac{\partial B^x}{\partial y} - \frac{1}{c^2}\frac{\partial E^z}{\partial t} &=& \mu_0\,  J^3 .
\end{eqnarray}
Reescribimos estas ecuaciones en la forma siguiente,
\begin{eqnarray}
0 + \partial_1 F^{10} + \partial_2 F^{20} + \partial_3 F^{30} &=& \mu_0\,
J^0 ,\\
\partial_0 F^{01} + 0+ \partial_2 F^{21} + \partial_3 F^{31} &=& \mu_0\,
J^1 ,\\
\partial_0 F^{02} + \partial_1 F^{12} +0+ \partial_3 F^{32} &=& \mu_0\,
J^2 ,\\
\partial_0 F^{03} + \partial_1 F^{13} + \partial_2 F^{23} +0&=& \mu_0\,
J^3,
\end{eqnarray}
o, en forma compacta,
\begin{eqnarray}
\partial_\mu  F^{\mu \nu} = \mu_0\, J^\nu  , \label{coninhom}
\end{eqnarray}
con
\begin{eqnarray}
F^{\mu \nu}:=
\left(
\begin{array}{cccc}
0&-E^x/c&-E^y/c&-E^z/c\\
E^x/c&0&-B^z&B^y\\
E^y/c&B^z&0&-B^x\\
E^z/c&-B^y&B^x&0
\end{array}
\right) . \label{Fupup}
\end{eqnarray}
Es claro de (\ref{coninhom}) que la invariancia de forma de las ecuaciones inhomogéneas de Maxwell queda asegurada si las 6 cantidades independientes correspondientes a los campos eléctrico y magnético \textit{forman un tensor antisimétrico} $F^{\mu \nu}=-F^{\nu\mu}$, llamado \textit{tensor de electromagnético} (o \textit{tensor de Faraday}).

En notación indicial 3-dimensional, tenemos
\begin{equation}
F^{0i}   =-\frac{1}{c}E^i, \qquad F^{ij}   =-\varepsilon^{ijk}B^{k}, \qquad i,j,k=1,2,3,
\end{equation}
donde $E^i$ y $B^i$ son las tres componentes de $\vec{E}$ y $\vec{B}$, respectivamente.

Concluimos entonces que:
\begin{quotation}
\textit{Las ecuaciones de Maxwell inhomogéneas, sin modificación alguna, son covariantes bajo TL's si el campo eléctrico y magnético son componentes del
tensor antisimétrico $F^{\mu \nu}$.}
\end{quotation}

\section{Ecuaciones de Maxwell homogéneas}

Puede verificarse fácilmente que las ecuaciones homogéneas (\ref{db0}) y (\ref{de+}) pueden escribirse como
\begin{equation}
\partial_\mu F_{\nu\lambda}+\partial_\nu F_{\lambda\mu}+\partial_\lambda
F_{\mu\nu}=0,
\end{equation}
o, equivalentemente,
\begin{equation}
\partial_{[\mu}F_{\nu\lambda]}=0,
\end{equation}
o bien,
\begin{equation}
\epsilon^{\mu\nu\lambda\rho}\,\partial_{\nu}F_{\lambda\rho}=0,
\end{equation}
donde $F_{\mu\nu}$ es el tensor completamente covariante (tipo $(^0_2)$)
correspondiente a $F_{\mu\nu}$, es decir,
$F_{\mu\nu}:=\eta_{\mu\lambda}\eta_{\nu\rho}F^{\lambda\rho}$. La matriz asociada
a este tensor es de la forma
\begin{equation}
F_{\mu\nu}=\left(
\begin{array}
[c]{cccc}
0 & E^x/c & E^y/c & E^z/c\\
-E^x/c & 0 & -B^z & B^y\\
-E^y/c & B^z & 0 & -B^x\\
-E^z/c & -B^y & B^x & 0
\end{array}
\right). \label{Fdndn}
\end{equation}
Resulta útil considerar además el tensor tipo $(^1_1)$ correspondiente, tal que
\begin{equation}
F^\mu_{\ \ \nu}=\left(
\begin{array}
[c]{cccc}
0 & E^x/c & E^y/c & E^z/c\\
E^x/c & 0 & B^z & -B^y\\
E^y/c & -B^z & 0 & B^x\\
E^z/c & B^y & -B^x & 0
\end{array}
\right). \label{Fupdn}
\end{equation}
En resumen, tenemos que las ecuaciones de Maxwell pueden escribirse como
\begin{equation}
\boxed{\partial_\mu  F^{\mu \nu} = \mu_0\, J^\nu,} \label{emihF}
\end{equation}
\begin{equation}
\boxed{\partial_\mu F_{\nu\lambda}+\partial_\nu F_{\lambda\mu}+\partial_\lambda
F_{\mu\nu}=0.} \label{homcov}
\end{equation}

\subsection{Ecuacion de Maxwell homogenea y tensor dual*}
La ecuación de Maxwell homogénea puede ser escrita, alternativamente, como
\begin{equation}
\boxed{\partial_\mu {\cal F}^{\mu\nu}=0,}
\end{equation}
donde hemos definido el \textit{tensor electromagnético dual}
\begin{equation}
{\cal F}^{\mu\nu}:=\frac{1}{2}\varepsilon^{\mu\nu\lambda\rho}F_{\lambda\rho} .
\end{equation}
En efecto, sabemos que (\ref{homcov}) es equivalente a
\begin{equation}
\partial_{\lambda}F_{\mu\nu}+\partial_{\nu}F_{\lambda\mu}+\partial_\mu 
F_{\nu\lambda}  =0.
\end{equation}
Multiplicamos esta ecuación por $-(1/6)\varepsilon^{\kappa\lambda\mu\nu}$
y encontramos que
\begin{eqnarray}
0&=&-\frac{1}{6}\varepsilon^{\kappa\lambda\mu\nu}\left(  \partial_{\lambda}%
F_{\mu\nu}+\partial_{\nu}F_{\lambda\mu}+\partial_\mu F_{\nu\lambda}\right)\\
&=&-\frac{1}{6}\left(  \varepsilon^{\kappa\lambda\mu\nu}\partial_{\lambda}%
F_{\mu\nu}+\varepsilon^{\kappa\nu\lambda\mu}\partial_{\nu}F_{\lambda\mu
}-\varepsilon^{\kappa\mu\lambda\nu}\partial_\mu F_{\nu\lambda}\right) \\
&=&-\frac{1}{6}\left(  \varepsilon^{\kappa\lambda\mu\nu}\partial_{\lambda}%
F_{\mu\nu}+\varepsilon^{\kappa\nu\lambda\mu}\partial_{\nu}F_{\lambda\mu
}+\varepsilon^{\kappa\mu\nu\lambda}\partial_\mu F_{\nu\lambda}\right) \\
&=&-\frac{1}{2}\varepsilon^{\kappa\lambda\mu\nu}\partial_{\lambda}F_{\mu\nu} \\
&=&\frac{1}{2}\varepsilon^{\lambda\kappa\mu\nu}\partial_{\lambda}F_{\mu\nu} \\
&=&\partial_{\lambda}{\cal F}^{\lambda\kappa}  .
\end{eqnarray}
La representación matricial de ${\cal F}^{\mu\nu}$ es
\begin{equation}
{\cal F}^{\mu\nu}=\left(
\begin{array}
[c]{cccc}
0 & -B^x & -B^y & -B^z\\
B^x & 0 & E^z/c & -E^y/c\\
B^y & -E^z/c & 0 & E^x/c\\
B^z & E^y/c & -E^x/c & 0
\end{array}
\right). \label{Fdualupup}
\end{equation}

\section{Transformación de Campos Electromagnéticos}
Como vimos anteriormente, el Principio de Relatividad aplicado a las ecuaciones de Maxwell queda asegurado asumiendo que los campos eléctrico y magnético forman parte de un tensor antisimétrico de segundo orden, el tensor electromagnético $F$. Esta condición determina la forma en
que el campo electromagnético debe cambiar de un SRI a otro. Si $F^{\mu\nu}$
son las componentes del campo electromagnético respecto a un SRI $K$ entonces,
respecto a otro SRI $K'$ relacionado con $K$ por medio de una TL $\Lambda$
($x'^\mu=\Lambda^\mu_{\  \nu}x^\nu$), tendremos que
\begin{equation}
F'^{\mu\nu}=\Lambda^\mu_{\ \lambda}\Lambda^\nu_{\ \rho}F^{\lambda\rho}.
\label{FLLF}
\end{equation}

A continuación estudiaremos más explícitamente cómo esta ley de transformación de $F$ determina
la forma en que los campos $\vec{E}$ y $\vec{B}$ cambian de un SRI a otro.

\subsection{Caso de un boost simple}
En el caso más simple de un boost en el eje $x$, ver (\ref{boost1}), tenemos
que
\begin{equation}
\Lambda^\mu_{\ \nu}=\left( \begin{array}{cccc}
\gamma & -\beta\gamma & 0 & 0\\
-\beta\gamma & \gamma & 0 & 0\\
0 & 0 & 1 & 0\\
0 & 0 & 0 & 1\\
\end{array}\right) ,
\end{equation}
donde $c\beta$ es la velocidad de $K'$ respecto a $K$. Usando la identificación
(\ref{Fupup}), encontramos que (\ref{FLLF}) es equivalente a
\begin{eqnarray}
E'^x &=& E^x ,\\
E'^y &=& \gamma ( E^y - \beta cB^z ) ,\\
E'^z &=& \gamma ( E^z + \beta cB^y ) ,\\
B'^x &=& B^x ,\\
B'^y &=& \gamma (B^y + \frac{\beta}{c} E^z) ,\\
B'^z &=& \gamma (B^z - \frac{\beta}{c} E^y).
\end{eqnarray}
Notamos una importante característica de esta ley transformación: \textbf{los campos $\vec{E}$ y $\vec{B}$ cambian sólo en sus componentes perpendiculares a la dirección de movimiento relativo entre $K$ y $K'$ (es decir, al eje $x$ en nuestro ejemplo), mientras que las componentes a lo largo de $\beta$ permanecen inalteradas}. 
Puede haberse esperado este comportamiento, por ejemplo,
analizando el caso del campo magnético generado por una línea de corriente
recta e infinita, desde el punto de vista de dos SRI's con movimiento relativo a lo largo de la línea de corriente. Puede también verificarse este comportamiento en el caso de ondas electromagnéticas planas, considerando un boost en la dirección de propagación.

\subsection{Caso de un boost general}
En este caso la matriz de Lorentz correspondiente es aquella dada en
(\ref{boostgen}). Usando nuevamente (\ref{Fupup}) y (\ref{FLLF}) encontramos, luego de algo de álgebra:
\begin{equation}
\boxed{\vec{E}'=\gamma \vec{E}+\gamma \left(c\vec{\beta}\times
\vec{B}\right) -\frac{(\gamma -1)}{\beta^2}\left( \vec{\beta}\cdot
\vec{E}\right) \vec{\beta} ,} \label{TLE}
\end{equation}
\begin{equation}
\boxed{\vec{B}'=\gamma \vec{B}-\frac{\gamma}{c} \left(\vec{\beta}\times
\vec{E}\right) -\frac{(\gamma -1)}{\beta^2}\left( \vec{\beta}\cdot
\vec{B}\right) \vec{\beta}.} \label{TLB}
\end{equation}
Estas transformaciones son \textit{distintas} de aquellas correspondientes a las componentes espaciales de un 4-vector, y presentan las mismas propiedades
discutidas en el caso del boost simple. Por ejemplo,
$\vec{\beta}\cdot\vec{E}'=\vec{\beta}\cdot\vec{E}$ y
$\vec{\beta}\cdot\vec{B}'=\vec{\beta}\cdot\vec{B}$, de modo que sólo las
componentes perpendiculares a $\vec{\beta}$ son modificadas.

Puede verificarse que estas expresiones son consistentes con nuestro conocimiento previo de los campos generados por distribuciones de carga en movimiento.

\subsubsection{Ejemplo 1}
Considere una línea recta infinita con densidad lineal de carga $\lambda$ moviéndose con velocidad $v$ a lo largo de su eje, que elegiremos como eje $x$. En este caso, la configuración de cargas y corrientes es estática, de modo que los campos producidos son independientes del tiempo. Por lo tanto, las ecuaciones de Maxwell se reducen a las usuales ecuaciones de la electro- y magnetoestática. Esto implica que los campos están dados por
\begin{equation}
 \vec{E}=\frac{\lambda}{2\pi\varepsilon_0}\frac{\hat{r}}{r}=\frac{\lambda}{2\pi\varepsilon_0}\left( \frac{y\hat{y}+z\hat{z}}{y^2+z^2}\right) , \qquad \vec{B}=\frac{\mu_0}{2\pi}\frac{\lambda v}{r}\hat\theta=\frac{\mu_0\lambda v}{2\pi}\left( \frac{y\hat{z}-z\hat{y}}{y^2+z^2}\right) . \label{clci1}
\end{equation}
\begin{center}
\begin{figure}[H]
\centerline{\includegraphics[height=3.5cm]{fig/fig-boost-campos-01.pdf}}
\caption{Una línea de carga moviéndose con velocidad $v$ a lo largo de su eje.} \label{fig:bc01}
\end{figure}
\end{center}
En particular, en el SRI $K'$ comóvil con las cargas $v'=0$ y la línea de carga tiene una densidad lineal $\lambda'$, entonces
\begin{equation}
 \vec{E}'=\frac{\lambda'}{2\pi\varepsilon_0}\left( \frac{y'\hat{y}+z'\hat{z}}{y'^2+z'^2}\right) , \qquad \vec{B}'=\vec{0}.
\end{equation}
En el SRI comóvil $K'$ sólo existe campo eléctrico, que es normal a la dirección de la línea. Aplicamos ahora la transformación de campos dada por (\ref{TLE}) y (\ref{TLE}). De acuerdo a esto, en el SRI donde la distribución lineal de carga se mueve con velocidad $\vec{v}=v\hat{x}$, tendremos
\begin{equation}
 \vec{E}=\gamma\vec{E}'=\frac{\gamma\lambda'}{2\pi\varepsilon_0}\left( \frac{y\hat{y}+z\hat{z}}{y^2+z^2}\right) , \qquad \vec{B}= \frac{\gamma}{c^2}\vec{v}\times\vec{E}'=\frac{\gamma\lambda'}{2\pi\varepsilon_0c^2}\left( \frac{y\hat{z}-z\hat{y}}{y^2+z^2}\right). \label{clci2}
\end{equation}
Aquí hemos usado el hecho que bajo un boost a lo largo del eje $x$, $y'=y$ y $z'=z$. Vemos que (\ref{clci2}) coincide con (\ref{clci1}) luego de identificar $\lambda=\gamma\lambda'$, que es la relación esperada entre las densidades lineales de carga, debido a la contracción relativista de longitudes a lo largo de la dirección de movimiento. Note que de las expresiones anteriores podemos escribir  $\vec{B}=\vec{v}\times\vec{E}/c^2$.

\subsubsection{Ejemplo 2}
Un ejemplo complementario lo suministra el caso de un magneto que genera, en su SRI comóvil $K'$, un campo un campo magnético como el ilustrado en la figura. Aquí entonces $\vec{E}'=\vec{0}$ y $\vec{B}'\neq\vec{0}$. En un SRI donde el magneto se mueve con velocidad $v$ a lo largo del eje $x$ común existirá, como consecuencia de la ley de Faraday, un campo eléctrico inducido perpendicular a la dirección de movimiento. Las expresiones encontradas para la transformación de los campos bajo un boost implican, en este caso,
\begin{equation}
 \vec{E}=-\gamma\left(\vec{v}\times\vec{B}'\right), \qquad
\vec{B}=\gamma \vec{B}'-\frac{(\gamma -1)}{v^2}\left( \vec{v}\cdot
\vec{B}'\right) \vec{v}.
\end{equation}
Verificamos, en particular, que el campo eléctrico en este nuevo SRI es normal a la dirección de movimiento del magneto.

Adicionalmente, considere el caso en que se ubica una carga $q$ a una cierta distancia del magneto, con velocidad inicial nula respecto de él. Entonces, en el SRI comóvil la carga tendrá velocidad inicial nula y la fuerza total sobre ella será también cero, ya que no existe campo eléctrico en dicho SRI. En el SRI en el que el magneto se mueve con velocidad $\vec{v}$ la carga tendrá una velocidad inicial igual a la del magneto ($\vec{v}$). De esta forma, la carga $q$ ``sentirá'ún campo eléctrico y uno magnético. No obstante, la fuerza total sobre ella es también nula:
\begin{eqnarray}
\vec{F}&=&q\left(\vec{E}+\vec{v}\times\vec{B}\right) \\
&=&q\left(-\gamma\,\vec{v}\times\vec{B}'+ \vec{v}\times\left[\gamma \vec{B}'-\frac{(\gamma -1)}{v^2}\left( \vec{v}\cdot
\vec{B}'\right) \vec{v}\right]\right)\\
&=&q\left(-\gamma\, \vec{v}\times\vec{B}'+\gamma\,\vec{v}\times\vec{B}'+0\right)\\
&=&\vec{0}.
\end{eqnarray}
Verificamos en este ejemplo que la descripción de la interacción entre el magneto y la carga es diferente en los dos SRI's discutidos, en un SRI sólo hay presente un campo magnético. En el otro, debido a la ley de Faraday, se induce además un campo eléctrico. Ambas descripciones, sin embargo, conciden en su predicción final: la carga no alterará su estado de movimiento ya que la fuerza sobre ella es nula.


\subsubsection{Ejemplo 3: Campo generado por una carga moviéndose con velocidad constante}\label{secEjEBvc}
Aquí aplicaremos la ley de transformación del campo electromagnético bajo
TL's para encontrar el campo producido por una carga moviéndose con velocidad constante respecto a un SRI $K$. Para ello, encontraremos primero los campos en el SRI $K'$ que es comóvil con la carga y luego transformaremos estos campos al SRI $K$ usando la TL apropiada.

Elegimos los ejes de modo que la carga se mueva a lo largo del eje $x$ ($x'$ respecto a $K'$). En el SRI $K'$, un evento $P$, donde queremos determinar el campo electromagnético, tiene coordenadas $x'^\mu(P)=(ct',\vec{x}')$. Bajo estas condiciones, el campo electromagnético, que en $K'$ corresponde simplemente a un campo eléctrostático coulombiano, adopta la forma:
\begin{equation}
\vec{E}'(P)=\frac{q}{4\pi\varepsilon_0}\frac{\vec{x}'}{r'^3}, \qquad \vec{B}' =\vec{0}. \label{EBp}
\end{equation}
Aplicando la ley de transformación (\ref{TLE}) y (\ref{TLB}) (en realidad, la
\textit{inversa} de estas transformaciones, que equivalen a reemplazar
$\vec{\beta}$ por $-\vec{\beta}$) obtenemos que en $K$
\begin{equation}
\vec{E}=\gamma \vec{E}'-\frac{(\gamma -1)}{\beta^2}\left( \vec{\beta}\cdot
\vec{E}'\right) \vec{\beta} , \qquad
\vec{B}=\frac{\gamma}{c} \left(\vec{\beta}\times\vec{E}'\right).
\end{equation}
Usando (\ref{EBp}) podemos escribir los campos en función de las coordenadas
$\vec{x}'$:
\begin{equation}
\vec{E}=\frac{q}{4\pi\varepsilon_0}\frac{\gamma \vec{x}'-\frac{(\gamma -1)}{\beta^2}(\vec{\beta}\cdot
\vec{x}')\vec{\beta}}{r'^3}  , \qquad
\vec{B}=\frac{q}{4\pi\varepsilon_0 c}\gamma \left(\vec{\beta}\times\frac{\vec{x}'}{r'^3}\right),
\end{equation}
Finalmente, usando la expresión de $\vec{x}'$ en función de $\vec{x}$ y $t$
correspondiente, es decir, aquella dada en (\ref{bg1}), llegamos a
\begin{align}
\vec{E}(x) &= \frac{q}{4\pi\varepsilon_0}\frac{\gamma\left( \vec{x}-\vec{\beta}ct\right)}
{\left[\vec{x}^2+\gamma^2\left(\vec{\beta}\cdot\vec{x}-ct\right)^2-c^2t^2\right]^{3/2}} , \\
\vec{B}(x) &= \frac{q}{4\pi\varepsilon_0c}\frac{\gamma\left( \vec{\beta}\times\vec{x}\right) }
{\left[\vec{x}^2+\gamma^2\left(\vec{\beta}\cdot\vec{x}-ct\right)^2-c^2t^2\right]^{3/2}}.
\end{align}
Aquí hemos usado $\gamma \vec{x}'-(\gamma -1)(\vec{\beta}\cdot\vec{x}')\vec{\beta}/\beta^2=\gamma\left( \vec{x}-\vec{\beta}ct\right) $ y
$r'^2=\vec{x}'^2=\vec{x}^2+\gamma^2\left(\vec{\beta}\cdot\vec{x}-ct\right)^2-c^2t^2$, que se deducen directamente de (\ref{bg1}).

Si además consideramos $\vec{\beta}=({v}/{c},0,0)$, donde $v$ es
la rapidez de la partícula, entonces
\begin{eqnarray}
\vec{E}(x)&=&\frac{q}{4\pi\varepsilon_0}\frac{\gamma (\vec{x}-\vec{x}t)}{\left[\gamma^2 (x-vt)^2+y^2+z^2\right]^{3/2}} ,
\label{Evx}\\
\vec{B}(x)&=&\frac{q}{4\pi\varepsilon_0c^2}\frac{\gamma v \left(z\hat{y}-y\hat{z}\right)}{\left[\gamma^2 (x-vt)^2+y^2+z^2\right]^{3/2}}, \label{Bvx}
\end{eqnarray}
ya que $ x^2+\gamma^2(\beta x)^2=\gamma^2x^2$.

Este resultado coincide con aquel en el capítulo \ref{caprad}, obtenido resolviendo directamente las ecuaciones de Maxwell para una partícula en movimiento rectilíneo, ver \eqref{EBqvconst}.

%
% Las superficies de igual intensidad de campo no son esferas, sino elipsoides
% de revolución en función de $\theta$, ya que:
% \begin{equation}
% \left(  b^2+\gamma^2v^2t^2\right)  ^{3/2}=\left(
% r^2\operatorname{sen}^2\theta+\gamma^2r^2\cos^2\theta\right)^{3/2},
% \end{equation}
% y además
% \begin{align}
% r^2\operatorname{sen}^2\theta+\gamma^2r^2\cos^2\theta &
% =r^2\left(  \operatorname{sen}^2\theta+\gamma^2\cos^2\theta\right) \\
% & =r^2\left(  \operatorname{sen}^2\theta+\gamma^2\left(
% 1-\operatorname{sen}^2\theta\right)  \right) \\
% & =r^2\left(  \operatorname{sen}^2\theta+\gamma^2-\gamma^2%
% \operatorname{sen}^2\theta\right) \\
% & =r^2\left(  \gamma^2+\left(  1-\gamma^2\right)  \operatorname{sen}%
%^2\theta\right) \\
% & =r^2\left(  \gamma^2-\gamma^2\beta^2\operatorname{sen}^2%
% \theta\right) \\
% & =r^2\gamma^2\left(  1-\beta^2\operatorname{sen}^2\theta\right) ,
% \end{align}
% de modo que
% \begin{equation}
% \left(  b^2+\gamma^2v^2t^2\right)  ^{3/2}=r^3\gamma^3\left(
% 1-\beta^2\operatorname{sen}^2\theta\right)  ^{3/2} .
% \end{equation}
% Reemplazando esto, obtenemos
% \begin{align}
% E_1  & =-\frac{q\gamma r\cos\theta}{r^3\gamma^3\left(  1-\beta
%^2\operatorname{sen}^2\theta\right)  ^{3/2}}=-\frac{qr\cos\theta}%
% {r^3\gamma^2\left(  1-\beta^2\operatorname{sen}^2\theta\right)
% ^{3/2}} ,\\
% E_2  & =\frac{q\gamma r\operatorname{sen}\theta}{r^3\gamma^3\left(
% 1-\beta^2\operatorname{sen}^2\theta\right)  ^{3/2}}=\frac
% {qr\operatorname{sen}\theta}{r^3\gamma^2\left(  1-\beta^2%
% \operatorname{sen}^2\theta\right)  ^{3/2}}.
% \end{align}
% Ahora bien, el vector que conecta la carga y el punto $P$ tiene componentes
% \begin{equation}
% r^1   =-r\cos\theta, \qquad r^2   =r\operatorname{sen}\theta
% \end{equation}
% y, por lo tanto,
% \begin{equation}
% E^i=\frac{qr^i}{r^3\gamma^2\left(  1-\beta^2\operatorname{sen}%
%^2\theta\right)  ^{3/2}}.
% \end{equation}
% Note que aunque el campo es radial, no es isótropico. A lo largo
% de la dirección de movimiento, cuando $\theta=0$ ó $\pi$, el campo es
% menor que cuando es $\pi/2$ ó $-\pi/2$.


\subsection{Invariantes electromagnéticos}

Si bien hemos visto que bajo un cambio de SRI (las componentes de) los campos
eléctrico y magnético cambian su valor, existen ciertas combinaciones de ellos que permanecen inalterados, es decir, con el mismo valor, bajo una TL. Estas cantidades escalares \textit{pueden ser construidas a partir del tensor electromagnético} y los otros tensores disponibles, que en nuestro caso son la métrica $\eta$ y el símbolo de Levi-Civita (o, equivalentemente, el tensor dual $\cal F$).

Considere entonces los siguientes escalares\footnote{En rigor $I_2$, tal como lo definimos aquí es un \textit{pseudoescalar}. En este curso no haremos
distinción entre escalares y pseudoescalares.}:
\begin{align}
I_1 & :=\frac{1}{2}\eta^{\mu\lambda}\eta^{\nu\rho}F_{\mu\nu}F_{\lambda\rho}=\frac{1}{2}F_{\mu\nu}F^{\mu\nu},\\
I_2 & :=\frac{1}{4}\epsilon^{\mu\nu\lambda\rho}F_{\mu\nu}F_{\lambda\rho}
= \frac{1}{2}F_{\mu\nu}{\cal F}^{\mu\nu}.
\end{align}
Usando (\ref{Fdndn}), (\ref{Fupup}) y (\ref{Fdualupup}) podemos reescribir los invariantes $I_1$ y $I_2$ en función de $\vec{E}$ y $\vec{B}$:
\begin{equation}
I_1=\vec{B}^2-\frac{1}{c^2}\vec{E}^2, \qquad I_2  =-\frac{2}{c}\vec{E}\cdot\vec{B}.
\end{equation}

El hecho que $I_1$ e $I_2$ permanezcan invariantes bajo TL's puede usarse para
obtener diversos resultados, tales como:
\begin{itemize}
\item Si en un SRI $\vec{E}^2<c^2\vec{B}^2$, entonces \textit{no es posible
encontrar un SRI en el que} $\vec{B}=\vec{0}$.
\item Si en un SRI $c^2\vec{B}^2<\vec{E}^2$, entonces \textit{no es posible
encontrar un SRI en el que} $\vec{E}=\vec{0}$.
\item Si en un SRI $\vec{E}$ es perpendicular a $\vec{B}$, entonces $\vec{E}$
será perpendicular a $\vec{B}$ \textit{en todo SRI}.
\item Si en un SRI $\vec{E}\cdot\vec{B}\neq 0$, entonces \textit{no es posible encontrar un SRI en el que} $\vec{E}=\vec{0}$ o $\vec{B}=\vec{0}$.
\end{itemize}
Note que las afirmaciones anteriores son válidas para los valores de $\vec{E}$
y $\vec{B}$ \textit{en un evento} dado. En general, si ninguno de los criterios anteriores lo impide, es posible encontrar un SRI que cumpla con algúna condición requerida. Por ejemplo, si en un SRI $\vec{E}^2<c^2\vec{B}^2$ y $\vec{E}\cdot\vec{B}=0$, entonces es posible encontrar un SRI en el que $\vec{E}=\vec{0}$.


\section{Potenciales y transformaciones de gauge}

El potencial escalar $\phi$ y el potencial vectorial $\vec{A}$ (en cojunto ``los potenciales electromagnéticos'') son campos definidos de forma tal que los campos eléctrico y magnético satisfagan automáticamente las ecuaciones de Maxwell homogéneas: 
\begin{equation}\label{defphi}
\vec{E} =  - \frac{\partial \vec{A}}{\partial t} - \vec{\nabla}\phi ,
\end{equation}
\begin{equation}\label{BrotA}
\vec{B}=\vec{\nabla}\times \vec{A}.
\end{equation}
Los potenciales no son funciones definidas unívocamente dada una configuración de campo electromagético. Esto se manifiesta en la invariancia de los campos $\vec{E}$ y $\vec{B}$ bajo una transformación de gauge:
\begin{equation}
\phi' =\phi +\frac{\partial \chi}{\partial t}, \qquad
{\vec{A}}' = \vec{A} - \vec{\nabla}{\chi},
\end{equation}
donde $\chi=\chi(\vec{x},t)$ es una función \textit{arbitraria} del espaciotiempo

En la formulación relativista, los potenciales electromagnéticos son componentes de un  \textit{4-potencial electromagnético} (o, simplemente 4-potencial), definido por
\begin{equation}
A^\mu =(\phi/c,\vec{A}), \qquad A_\mu =(\phi/c,-\vec{A}). \label{identA}
\end{equation}
En términos de este 4-potencial, las expresiones (\ref{defphi}) y (\ref{BrotA}) se condensan en
\begin{equation}
\boxed{F_{\mu \nu}=\partial_\mu A_{\nu}-\partial_{\nu}A_\mu.} \label{fda}
\end{equation}

Ya que $F_{\mu\nu}$ es un tensor tipo $(0,2)$, para que la relación (\ref{fda})
se válida en todo SRI es necesario que $A_\mu$ constituya un vector
\textit{covariante} bajo TL's.

Una transformación de gauge adopta entonces la forma
\begin{equation}
\boxed{A_\mu' = A_\mu  + \partial_\mu  \chi .}
\end{equation}
Bajo esta transformación, el tensor electromagnético $F_{\mu \nu}$ permanece
invariante. En efecto,
\begin{eqnarray}
F_{\mu\nu}'&=&\partial_\mu A_{\nu}'-\partial_{\nu}A_\mu '  \\
&=&\partial_\mu (A_{\nu}+\partial_{\nu}
\chi)-\partial_{\nu}(A_\mu +\partial_\mu  \chi)\\
&=&\partial_\mu A_{\nu}-\partial_{\nu}A_\mu  \\
&=&F_{\mu \nu}.
\end{eqnarray}

\subsection{Gauge de Lorenz}
Es directo escribir la condición de Lorenz en forma covariante. Usando
(\ref{identA}) obtenemos que
\begin{equation}
\frac{1}{c^2}\frac{\partial\phi}{\partial t}+\vec{\nabla}\cdot\vec{A}=0,
\end{equation}
es equivalente a
\begin{equation}
\partial_\mu A^\mu=0. \label{gLA}
\end{equation}
Ya que el 4-potencial es un vector bajo TL's, vemos que la condición que define el gauge de Lorenz es una ecuación covariante (en particular, es una \textit{ecuación escalar}). Esto asegura que si el gauge de Lorenz se satisface en un SRI $K$ con los potenciales $\phi$ y $\vec{A}$ correspondientes, entonces será también satisfecho en todo SRI $K'$, con los potenciales $\phi'$ y $\vec{A}'$ transformados. Esta propiedad hace que el gauge de Lorenz sea ampliamente usado en el contexto de la teoría de relatividad especial, puesto que permite trabajar con los potenciales electromagnéticos, simplificar muchas ecuaciones al imponer el gauge de Lorenz, pero de forma tal que los resultados sean válidos en cualquier SRI. Esta útil propiedad no es válida, por ejemplo, usando el gauge de Coulomb.

Si el 4-potencial satisface el gauge de Lorenz, la ecuación diferencial que éste debe satisfacer se reduce a la \textit{ecuación de onda inhomogenea}
\begin{equation}
\boxed{\square A^\mu=\mu_0 J^\mu .} \label{caj}
\end{equation}

En efecto, escribiendo las ecuaciones de Maxwell inhomogeneas (\ref{emihF}) en términos del 4-potencial, encontramos que
\begin{align}
\mu_0 J^\nu &=\partial_\mu F^{\mu\nu}  \\
&=\partial_\mu \left(  \partial^\mu A^\nu -\partial^\nu A^\mu \right)  \\
&= \partial_\mu \partial^\mu A^\nu -\partial_\mu \partial^\nu A^\mu   \\
&= \square A^\nu -\partial^\nu (\partial_\mu A^\mu)   ,
\end{align}
que se reduce a (\ref{caj}) cuando $A_\mu$ satisface (\ref{gLA}).

\section{Forma covariante de la Fuerza de Lorentz}
Sabemos que en mecánica no-relativista la fuerza de Lorentz, que describe cómo el campo electromagnético actúa sobre cargas de prueba, es de la forma
\begin{equation}
\frac{d\vec{p}}{dt}=q\left(  \vec{E}+\vec{v}\times\vec{B}\right).
\label{dpdt}
\end{equation}
Sabemos que en RE el moméntum está contenido en el 4-moméntum, junto con la
energía de la partícula. Por tanto, la ecuación anterior debe ser
complementada con la ecuación que expresa la variación de la energía de la
partícula en función del tiempo. Es conocido en mecánica no-relativista que
la potencia (trabajo por unidad de tiempo) realizada por el campo
electromagnético sobre una carga es $q\,\vec{v}\cdot\vec{E}$, de modo que
\begin{equation}
\frac{dE}{dt}=q\,\vec{v}\cdot\vec{E}. \label{dEdt}
\end{equation}
Tomando en cuenta que energía y moméntum forman el vector de 4-moméntum $p^\mu=({E}/{c},\vec{p})$ en RE, vemos que los lados
izquierdos de (\ref{dpdt}) y (\ref{dEdt}) corresponden a las componentes de
${dp^\mu}/{dt}$, mientras que los lados derechos son \textit{lineales en los campos y en la velocidad} (excepto $q\vec{E}$).

Consideremos entonces una expresión tensorial, lineal en el tensor
electromagnético y en la 4-velocidad, a saber: $F^\mu{}_{\nu}u^\nu$. Las
componentes temporales y espaciales de este 4-vector son
\begin{eqnarray}
F^0{}_{\nu}u^\nu &=&F^0{}_{i}u^i=\frac{\gamma}{c}\vec{E}\cdot\vec{v} ,\\
F^i{}_{\nu}u^\nu &=&F^i{}_{0}u^0+F^i{}_{j}u^j=\gamma \left(
E^i+\epsilon^{ijk}B^j v^k\right) .
\end{eqnarray}
Con esto, podemos escribir  (\ref{dpdt}) y (\ref{dEdt}), sin modificación
alguna, como
\begin{equation}
\frac{dp^\mu }{dt}=\frac{q}{\gamma}F^\mu _{\ \ \nu}u^\nu ,
\end{equation}
o, en \textit{términos del tiempo propio de la partícula} ($dt=\gamma d\tau$),
finalmente como
\begin{equation}
\boxed{\frac{dp^\mu }{d\tau}=qF^\mu _{\ \ \nu}u^\nu .} \label{LorRel}
\end{equation}
En resumen, es posible escribir la ecuación de fuerza de Lorentz en forma
covariante (lo que asegura su validez en todo SRI), \textit{sin necesidad de
modificarla explícitamente}. Note, sin embargo que, debido a que la energía y el moméntum relativista tienen expresiones distintas a aquellas usadas en mecánica no-relativista (y por tanto, asumen valores distintos para la mismas masas y velocidades), las \textit{trayectorias} \textit{predichas}\footnote{o `modeladas', o `descritas' \dots} \textit{por (\ref{LorRel}) serán distintas} a aquellas determinadas en la mecánica no-relativista (para una carga y campos dados).

\subsection{Ejemplo}
Considere el caso de una partícula de carga $q$ en una región del espacio(tiempo) donde sólo existe un campo eléctrico uniforme. Orientaremos los ejes espaciales de modo que el campo eléctrico sólo tenga componente a lo largo del eje $x$ positivo, es decir, $\vec{E}=(E,0,0)$ y además $\vec{B}=\vec{0}$. Con esto, el movimiento de la partícula será unidimensional. Asumiremos además las condiciones iniciales $\vec{x}(0)=\vec{0}$ y $\vec{v}(0)=\vec{0}$, y nos proponemos
encontrar la trayectoria que describe esta partícula de acuerdo a las
ecuaciones de movimiento no-relativistas y relativistas.

\subsubsection{Modelo no-relativista:}

En este caso, la ecuación de movimiento que determina la trayectoria de la
partícula es
\begin{equation}
m\frac{d^2x}{dt^2}=qE,
\end{equation}
cuya solución es
\begin{equation}
x(t)=\frac{qE}{2m}t^2, \qquad v(t)=\frac{qE}{m}t.
\end{equation}
Esta solución, tal como se espera en el contexto de la mecánica
no-relativista, permite en principio que la velocidad de la partícula aumente
indefinidamente.

\subsubsection{Modelo relativista:}

De acuerdo a (\ref{LorRel}), y recordando la definición del 4-moméntum (\ref{def4mom}), tenemos que en este caso las ecuaciones de movimiento relativistas adoptan la forma
\begin{eqnarray}
\frac{du^0}{d\tau}&=& \frac{qE}{mc} u^1, \label{mre1} \\
\frac{du^1}{d\tau}&=&\frac{qE}{mc} u^0.\label{mre2}
\end{eqnarray}
Derivando (\ref{mre1}) y reeemplazando en (\ref{mre1}) obtenemos
\begin{equation}
 \frac{d^2u^0}{d\tau^2}=\left( \frac{qE}{mc}\right)^2 u^0,
\end{equation}
cuya solución es de la forma
\begin{equation}
u^0(\tau)=A\sinh\left( \frac{qE}{mc}\tau\right)+B\cosh\left(
\frac{qE}{mc}\tau\right). \label{mrsu0}
\end{equation}
Reemplazando esta solución en (\ref{mre1}) encontramos además que
\begin{equation}
u^1(\tau)=A\cosh\left( \frac{qE}{mc}\tau\right)+B\sinh\left(
\frac{qE}{mc}\tau\right). \label{mrsu1}
\end{equation}
Integrando (\ref{mrsu0}) y (\ref{mrsu1}) encontramos la solución para las coordenadas en términos del tiempo propio,
\begin{eqnarray}
x^0(\tau)&=&\left(\frac{mc}{qE}\right)\left[A\cosh\left( \frac{qE}{mc}\tau\right)+B\sinh\left(
\frac{qE}{mc}\tau\right)+\alpha\right] ,\\
x^1(\tau)&=&\left(\frac{mc}{qE}\right)\left[A\sinh\left( \frac{qE}{mc}\tau\right)+B\cosh\left(
\frac{qE}{mc}\tau\right)+\alpha'\right] .
\end{eqnarray}
Debemos aún determinar las constantes $A$, $B$, $\alpha$ y $\alpha'$. Para esto, imponemos las condiciones iniciales $x^0(0)=0$, $x^1(0)=0$ y $\vec{v}_0=\vec{0}$, que en nuestro caso es equivalente a $u^1(0)=0$. Estas condiciones requieren que $\alpha=-A$ y $\alpha'=-B$ y además $A=0$. Con esto, la solución se reduce a
\begin{eqnarray}
x^0(\tau)&=&\left(\frac{mc}{qE}\right)B\sinh\left(\frac{qE}{mc}\tau\right) ,\\
x^1(\tau)&=&\left(\frac{mc}{qE}\right)B\left[\cosh\left(
\frac{qE}{mc}\tau\right)-1\right] .
\end{eqnarray}
Podemos determinar la constante $B$ de la condición $u^\mu
u_\mu=(u^0)^2-(u^1)^2=c^2$. Así obtenemos $B=c$, de modo que la
solución de las ecuaciones de movimiento relativistas para nuestro ejemplo es:
\begin{eqnarray}
ct(\tau)&=&\frac{mc^2}{qE}\sinh\left( \frac{qE}{mc}\tau\right), \label{cttau}\\
x(\tau)&=&\frac{mc^2}{qE}\left[ \cosh\left( \frac{qE}{mc}\tau\right)-1\right] .
\end{eqnarray}

Finalmente, podemos expresar la posición $x$ en función del tiempo $t$,
despejando $\tau$ en función de $t$ de (\ref{cttau}). Así obtenemos:
\begin{equation}
x(t) = \frac{mc^2}{qE} \left[ \sqrt{1 + \frac{q^2 E^2 t^2}{m^2c^2}} - 1 \right].
\label{xtsol}
\end{equation}
La correspondiente velocidad es entonces
\begin{equation}
v(t) = \frac{qE}{m}\frac{t}{\sqrt{1 + \frac{q^2 E^2 t^2}{m^2c^2}}}.
\end{equation}
Verificamos que para tiempos ``pequeños'', $t\ll {mc}/{qE}$, la solución se
reduce a aquella encontrada en el contexto no-relativista. Sin embargo, la
solución (\ref{xtsol}) difiere de la no-relativista para tiempos grandes
comparados con ${mc}/{qE}$, en los que la carga alcanza velocidades
relativistas ($v={c}/{\sqrt{2}}\approx 0,7c$ para $t={mc}/{qE}$). Como esperamos, la solución relativista asegura que la velocidad de la partícula es siempre subluminal.

\begin{center}
\begin{figure}[H]
\centerline{\includegraphics[height=7cm]{fig/fig-trayectoria-1D-rel-norel.pdf}}
\caption{Comparación entre velocidades según ecuación de Lorentz relativista y no-relativista.}
\label{fig:relnorel}
\end{figure}
\end{center}


\subsubsection{Fórmula de Larmor relativista*}


Consideraremos nuevamente la expresión para la potencia radiada por una carga acelerada, pero ahora en el caso en que su velocidad no es despreciable comparada con la velocidad de la luz. El método directo para calcular la
potencia total radiada es rehacer el cálculo de la sección \ref{sec:Larmor-norel}, pero usando el campo $\vec{E}_{(2)}$ general dado en (\ref{Egen}). Una forma alternativa es generalizar el resultado no-relativista teniendo en cuenta que \textit{la potencia es un escalar bajo TL's}. Esta propiedad de $P$ puede ser inferida considerando que la energía irradiada en un intervalo de tiempo $dt$ es $dE=P(t)dt$, y el hecho que tanto la energía ($dE$) y el tiempo ($dt$) transforman bajo TL's como las componentes temporales de un 4-vector (4-moméntum y 4-posición, respectivamente). Usando ahora el hecho que $P$ es un invariante bajo TL's, podemos buscar una generalización de \eqref{R-larmor} que incluya la 4-aceleración $\mathsf{a}^\mu$ en lugar de la aceleración $\vec{a}$, que se reduzca en el límite no-relativista a \eqref{R-larmor} y que sea un escalar. La expresión apropiada que cumple estos requisitos es:
\begin{equation}
\boxed{P=-\frac{q^2\mu}{6\pi c}\,\mathsf{a}^\mu\mathsf{a}_\mu.}\label{R-larmor-rel}
\end{equation}
Usando (\ref{4a2}) podemos escribir la potencia en el caso
relativista en términos de la velocidad y aceleración de la carga:
\begin{equation}
\boxed{P=\frac{q^2\mu}{6\pi c}\,\gamma^6\left[ \vec{a}^2-\left(\vec{\beta}\times\vec{a}\right)^2\right] .}\label{R-larmor2}
\end{equation}
Note que, tanto en el caso relativista como no-relativista, si las cargas son
aceleradas por una misma fuerza (por ejemplo por un campo eléctrico externo)
entonces las cargas de menor masa irradiarán más energía que las más
masivas, ya que su aceleración será mayor. Por ejemplo, en un plasma con
electrones y protones acelerados por un campo eléctrico externo, la potencia
radiada por los electrones será aproximadamente $(1836)^2\approx 3\times
10^{6}$ veces mayor que aquella radiada por los protones, ya que $m_p\approx
1836\,m_e$.

En general, podemos escribir
\begin{equation}
\mathsf{a}^\mu\mathsf{a}_\mu=\frac{1}{m^2}\frac{dp^\mu}{d\tau}\frac{dp_\mu}{
d\tau}=\frac{1}{m^2c^2}\left( \frac{dE}{d\tau}\right)^2-\frac{1}{m^2}\left(
\frac{d\vec{p}}{d\tau}\right)^2.
\end{equation}
Por otro lado, derivando la relación $E^2=\vec{p}^2c^2+m^2c^4$, obtenemos
\begin{equation}
\frac{dE}{d\tau}=c^2\frac{p}{E}\frac{dp}{d\tau}=v\frac{dp}{d\tau},
\label{dEdtau}
\end{equation}
donde $p$ es el módulo del moméntum $\vec{p}$. Usando (\ref{dEdtau}) podemos
reescribir (\ref{R-larmor-rel}) como
\begin{equation}
P=-\frac{q^2\mu}{6\pi m^2c}\left[ \beta^2\left(
\frac{dp}{d\tau}\right)^2-\left( \frac{d\vec{p}}{d\tau}\right)^2\right] .
\label{4a2-2}
\end{equation}

\subsubsection{Ejemplo: Aceleradores lineales}
Consideramos ahora el caso particular de un movimiento unidimensional. En este
caso $\vec{p}=p\hat{p}$ donde $\hat{p}$ es constante, de modo que
${d\vec{p}}/{d\tau}=({dp}/{d\tau})\hat{p}$ y entonces
\begin{eqnarray}
\beta^2\left(\frac{dp}{d\tau}\right)^2-\left( \frac{dp}{d\tau}\right)^2
&=&\left( \beta^2-1\right) \left(  \frac{dp}{d\tau}\right)^2\\
&=&-\frac{1}{\gamma^2} \left(  \frac{dp}{d\tau}\right)^2\\
&=&-\left(\frac{dp}{dt}\right)^2.
\end{eqnarray}
Aquí usamos el hecho que $dt=\gamma d\tau$. Con esto podemos escribir la fórmula de Larmor relativista (\ref{4a2-2}) para el caso de movimiento unidimensional como:
\begin{equation}
\boxed{P_{1d}=\frac{q^2\mu}{6\pi m^2c}\left(  \frac{dp}{dt}\right)^2.}
\end{equation}
Alternativamente, introduciendo la \textit{tasa de cambio de la energía de la
carga por unidad de longitud recorrida}
\begin{equation}
 \frac{dE}{dl}=\frac{\frac{dE}{dt}}{\frac{dl}{dt}}=\frac{1}{v}\frac{dE}{dt}
= \frac{dp}{dt},
\end{equation}
encontramos
\begin{equation}
\boxed{P_{1d}=\frac{q^2\mu}{6\pi m^2c}\left(  \frac{dE}{dl}\right)^2.}
\end{equation}
Se acostumbra definir la fracción de potencia radiada respecto a la energía
suministrada a la partícula por unidad de tiempo:
\begin{equation}
\eta:=\frac{\left( \text{Potencia radiada}\right) }{\text{(Potencia entregada a la carga)}}=\frac{P}{\frac{dE}{dt}}.
\end{equation}
Usando los resultados anteriores encontramos que
\begin{equation}
\eta=\frac{q^2\mu}{6\pi m^2c}\frac{1}{v}\frac{dE}{dl}.
\end{equation}
En el caso ultra-relativista $v\approx c$ (como en los aceleradores de
partículas), tenemos que
\begin{equation}
\eta\approx\frac{q^2\mu}{6\pi m^2c^2}\frac{dE}{dl}.
\end{equation}
En el caso de los aceleradores lineales se está interesado en minimizar las
``pérdidas'' de energía por radiación, es decir, se requiere que $\eta\ll 1$.
Esto ocurre si las cargas son aceleradas ``lentamente'' de modo que
\begin{equation}
\frac{dE}{dl}\ll \frac{6\pi m^2c^2}{q^2\mu}.
\end{equation}
Para electrones encontramos que ${6\pi m^2c^2}/{q^2\mu}\approx 2\times 10^4\, 
[MeV/m]$. Esta condición es satisfecha, por ejemplo, en un acelerador lineal como
el SLAC\footnote{\url{http://www.slac.stanford.edu}.}, donde se consigue
${dE}/{dl}\approx 50\, [MeV/m]$.

\section{Formalismo Lagrangeano*}
\subsection{La acción para una partícula.}
Nos basaremos en el \emph{principio de mínima acción}, o
principio de Hamilton. El principio de mínima acción
establece que un sistema mecánico es tal que al ir desde una
configuración en el instante $t_1$ hasta otra configuración en el instante
$t_2$, la \textit{acción} $S$,
\begin{equation}
S=\int L\left[  q^i(t),\dot{q}^i(t)  ,t\right]  dt, \label{Sdt}
\end{equation}
\textit{adopta un valor extremo}.

La acción tiene unidades $\left[  S\right]  = \text{[Energía][Tiempo]=[Moméntum][Distancia]}=[\hbar]=ML^2T^{-1}$.

Al considerarse variaciones pequeñas de las coordenadas y
velocidades, la condición de extremo de $S$, $\delta S=0$, implica las
ecuaciones de movimiento de \textit{Euler-Lagrange}:
\begin{equation}
\frac{d}{dt}\left(  \frac{\partial L}{\partial\dot{q}^i}\right)
-\frac{\partial L}{\partial q^i}=0.\label{L-euler}
\end{equation}

Requerimos que $S$ sea un escalar bajo TL's. Esto asegura que las ecuaciones de
movimiento serán covariantes bajo TL's, respetando así el principio de
relatividad. Usando $dt=\gamma d\tau$, encontramos que $S$ será escalar si
$L':=L\gamma$ es escalar, de modo que
\begin{equation}
S=\int L' d\tau.
\end{equation}
Por lo tanto, necesitamos encontrar un escalar $L'=L'(x^\mu,u^\mu)$. Esperamos
además que $L'$ no dependa explícitamente de las coordenadas $x^\mu$,
respetando así la invariancia bajo translaciones asumida en RE. El escalar $m u^\mu u_\mu$  es análogo a $m\vec{v}^2$  que, salvo factores
numéricos, es el lagrangeano no-relativista de una partícula libre de masa $m$. Sin embargo, en RE $m u^\mu u_\mu\equiv mc^2$, es decir, una constante. Esto podría sugerir considerar $L'$ proporcional a la constante $mc^2$. Definamos entonces,
\begin{equation}
L'=\alpha mc^2,
\end{equation}
donde hemos introducido la constante adimensional $\alpha$, y veamos si la acción resultante conduce a las ecuaciones de movimiento correctas para una partícula libre.

En este caso, tenemos
\begin{equation}
S=\alpha mc^2\int  d\tau=\alpha mc^2\int \frac{1}{\gamma}dt=\alpha mc^2\int
\sqrt{1-\frac{v^2}{c^2}}\,dt=\int L dt, \label{lagrellibre0}
\end{equation}
con $L=\alpha mc^2 \sqrt{1-\frac{v^2}{c^2}}$.
Para velocidades no-relativistas $\sqrt{1-\frac{v^2}{c^2}}=1-\frac{v^2}{2c^2}+O(\frac{v^4}{c^4})$  de modo que
\begin{equation}
L=\alpha mc^2-\frac{\alpha}{2}mv^2+O(\frac{v^4}{c^4}). \label{cL1}
\end{equation}
El primer término en el lado derecho de (\ref{cL1}) es irrelevante para nuestros fines puesto que no aporta a las ecuaciones de movimiento (es una derivada total). El segundo término será igual al lagrangiano no-relativista de una partícula libre si elegimos $\alpha=-1$. Con esta elección aseguramos que en el límite no-relativista las ecuaciones de movimiento serán las apropiadas. Para velocidades arbitrarias, nuestro candidato a Lagrangiano para una partícula relativista es
\begin{equation}
L=-mc^2 \sqrt{1-\frac{v^2}{c^2}},\label{lagrellibre}
\end{equation}
y la acción correspondiente,
\begin{equation}
\boxed{S=- mc^2\int  d\tau=\int L\, dt.}
\end{equation}

Considerando las variables de configuración como $q^i=x^i(t)$, el lagrangiano (\ref{lagrellibre}) sólo depende de las velocidades $\dot{q}^i=\dot{x}^i(t)$, por lo que los momenta canónicos\footnote{El primero en introducir este moméntum relativista fue Max Planck en 1906, usando el formalismo hamiltoniano, ver \cite{Planck06}.}
\begin{equation}
\frac{\partial L}{\partial \dot{x}^i}=\frac{mv^i}{\sqrt{1-\frac{v^2}{c^2}}}=p^i,
\end{equation}
son conservados. En efecto, las ecuaciones de movimiento son en este caso
\begin{equation}
\frac{d\ }{dt}\frac{\partial L}{\partial \dot{x}^i}-\frac{\partial L}{\partial x^i}=\frac{d p^i}{dt} =0,
\end{equation}
que es la ecuación de movimiento correcta para una partícula relativista
libre.

\subsubsection{Derivación covariante*}

Podemos además encontrar la ecuación de movimiento en una forma explícitamente
covariante bajo TL's. Para esto introducimos un parámetro admisible $\lambda$
sobre la trayectoria de la partícula, y consideraremos como variables de
configuración a $x^\mu(\lambda)$. En este caso, podemos escribir $S$ como
\begin{equation}
S=-mc^2\int  d\tau=-mc\int  ds=-mc\int  \frac{ds}{d\lambda}d\lambda=-mc\int
\sqrt{\eta_{\mu\nu}\frac{dx^\mu}{d\lambda}\frac{dx^\nu}{d\lambda}}\,
d\lambda=:\int  \tilde{L}\,d\lambda,
\end{equation}
con
\begin{equation}
\tilde{L}:=-mc\sqrt{\eta_{\mu\nu}\frac{dx^\mu}{d\lambda}\frac{dx^\nu}{d\lambda}}
. \label{Ltilde}
\end{equation}
De esta forma, describimos el sistema por un principio variacional análogo a
(\ref{Sdt}), donde $\tilde{L}$ reemplaza a $L$ y $\lambda$ a $t$. Debido a esto,
las ecuaciones de Euler-Lagrange en este caso son de la forma
\begin{equation}
\frac{d}{d\lambda}\left(  \frac{\partial
\tilde{L}}{\partial\left(\frac{dx^\mu}{d\lambda}\right) }\right)-\frac{\partial
\tilde{L}}{\partial x^\mu}=0.
\end{equation}
Usando (\ref{Ltilde}) encontramos que
\begin{equation}
 \frac{\partial \tilde{L}}{\partial\left(\frac{dx^\mu}{d\lambda}\right)
}=-\frac{mc}{\sqrt{\eta_{\mu\nu}\frac{dx^\mu}{d\lambda}\frac{dx^\nu}{d\lambda}}}
\,\eta_{\mu\nu}\frac{dx^\nu}{d\lambda}.
\end{equation}
Además, como
\begin{equation}
\sqrt{\eta_{\mu\nu}\frac{dx^\mu}{d\lambda}\frac{dx^\nu}{d\lambda}}=\frac{ds}{
d\lambda}=c\frac{d\tau}{d\lambda},
\end{equation}
podemos escribir
\begin{equation}
 \frac{\partial \tilde{L}}{\partial\left(\frac{dx^\mu}{d\lambda}\right)
}=-m\,\eta_{\mu\nu}\frac{dx^\nu}{d\lambda}\frac{d\lambda}{d\tau}=-m\,\eta_{
\mu\nu}\frac{dx^\nu}{d\tau}=-m\,\eta_{\mu\nu}u^\mu=-\eta_{\mu\nu}p^\mu=-p_\mu.
\end{equation}
Por otro lado, ${\partial \tilde{L}}/{\partial x^\mu}=0$, de modo que las
ecuaciones de movimiento son
\begin{equation}
\frac{d}{d\lambda}p_\mu=0,
\end{equation}
o, equivalentemente
\begin{equation}
\frac{dp^\mu}{d\tau}  =0.
\end{equation}

\subsection{Partícula en un campo electromagnético externo}

Ahora buscaremos la acción que describe la dinámica de una partícula cargada
que se mueve en un campo electromagnético externo dado. Para esto, recordamos
que en el límite no-relativista, una carga $q$ en un potencial electrostático
$\phi$ tiene una energía potencial de interacción $V=q\phi$. Como el
lagrangeano no-relativista es $T-V$, entonces la acción de interacción $S_\text{int, no-rel}$ es de la forma
\begin{equation}
S_\text{int, no-rel}= -q\int \phi\, dt. \label{L-rela02}
\end{equation}
Consideramos ahora las posibles generalizaciones de (\ref{L-rela02}) al caso
relativista. Nuevamente, requerimos que la acción que describe la interacción
de la carga con el campo $S_{\rm int}$ (que se suma a la acción libre
(\ref{lagrellibre})) debe ser un escalar bajo TL's y reducirse a
(\ref{L-rela02}) para velocidades pequeñas respecto a la de la luz.

Una acción que cumple con estos requisitos es
\begin{equation}
\boxed{S_{\rm int}=-q\int_{\tau_1}^{\tau_2}A_\mu u^\mu d\tau.}
\label{Sint}
\end{equation}
En efecto, escribiendo (\ref{Sint}) en términos de las componentes espaciales y
temporales, encontramos que
\begin{equation}
S_{\rm int}=-q\int\left(  \phi u^0+A_i u^i\right)
d\tau=-q\int \left(\frac{1}{c}\phi \gamma c-\vec{A}\cdot\vec{v}\gamma \right)
\frac{1}{\gamma}dt =\int \left(-q\phi+q\vec{A}\cdot\vec{v}\right) dt ,
\label{L-rela03b}
\end{equation}
que coincide con (\ref{L-rela02}) en el caso no-relativista (y/o
electrostático).

Consideramos ahora la acción completa para una partícula de carga $q$, masa
$m$ en un campo electromagnético:
\begin{equation}
\boxed{S=-\int \left(mc^2+ qA_\mu u^\mu \right) d\tau.}
\end{equation}
Derivaremos las ecuaciones de movimiento en forma covariante usando nuevamente
el parámetro $\lambda$ para parametrizar la trayectoria $x^\mu(\lambda)$ de la
partícula en el espaciotiempo. En este caso $S=\int \tilde{L}d\lambda$ con
\begin{equation}
\tilde{L}:=-mc\sqrt{\eta_{\mu\nu}\frac{dx^\mu}{d\lambda}\frac{dx^\nu}{d\lambda}}-qA_\mu \frac{dx^\mu}{d\lambda}. \label{Ltilde2}
\end{equation}
Extendiendo el caso antes visto, tenemos que
\begin{equation}
 \frac{\partial\tilde{L}}{\partial\left(\frac{dx^\mu}{d\lambda}\right)}=-p_\mu-qA_\mu.
\end{equation}
Además,
\begin{equation}
\frac{\partial \tilde{L}}{\partial x^\mu}=q\left( \partial_\mu
A_\nu\right)  \frac{dx^\nu}{d\lambda}.
\end{equation}
Con esto, las ecuaciones de movimiento son
\begin{equation}
\frac{d}{d\lambda}\left( -p_\mu -qA_\mu\right) -q\left(
\partial_\mu A_\nu\right)=0. \label{casicasi}
\end{equation}
Considerando que ${dA_\mu}/{d\lambda}=\left( \partial_\nu A_\mu \right)
({dx^\nu}/{d\lambda})$, podemos expresar (\ref{casicasi}) como
\begin{equation}
\frac{dp_\mu}{d\lambda} =q\left( \partial_\mu A_\nu-\partial_\nu
A_\mu\right)\frac{dx^\nu}{d\lambda}.
\end{equation}
Retornando a la parametrización en términos del tiempo propio, encontramos que
la ecuación de movimiento es
\begin{equation}
\frac{dp_\mu}{d\tau} =q\left( \partial_\mu A_\nu-\partial_\nu
A_\mu\right)u^\nu,
\end{equation}
que es exactamente la ecuación de movimiento determinada por la fuerza de
Lorentz (\ref{LorRel}), ver ecuación (\ref{fda}).


\subsection{Formalismo Lagrangeano para sistemas continuos (campos)}
Resumiremos aquí los aspectos básicos del formalismo lagrangeano para sistemas
con un continuo de grados de libertad (campos).

 Consideremos una \textit{densidad lagrangeana} que depende de $N$ campos $\phi^A$,
$A=1,\dots ,N$ y sus derivadas $\partial_\mu \phi^A$ y eventualmente de algún
campo {\em externo} $\Phi$. En este caso el lagrangeano del sistema es la
integral de volumen de la densidad lagrangeana:
\begin{equation}
L=\int_V {\cal L}(\phi^A,\partial_\mu\phi^A,\Phi)\, dV ,\label{lag4d}
\end{equation}
de modo que la acción es de la forma
\begin{equation}
\boxed{S=\int L\, dt= \int {\cal L}(\phi^A,\partial_\mu\phi^A,\Phi)\, dt dV=
\frac{1}{c}\int {\cal L}\,d^4x.}
\end{equation}
 Las correspondientes ecuaciones de movimiento de Euler-Lagrange respecto a variaciones del campo $\phi^A$, obtenidas del principio variacional $\delta S=0$, son:
\begin{equation}
\frac{\partial {\cal L}}{\partial \phi ^{A}}-\partial _{\mu }\left(
\frac{\partial {\cal L}}{\partial (\partial_{\mu }\phi^{A})}\right)=0.
\label{eccont2}
\end{equation}
En general, estas ecuaciones serán de segundo orden en las derivadas de los
campos $\phi^A$. A partir de $\cal L$ definimos la \textit{densidad de momento
canónico}:
\begin{equation}
\pi_A(x,t)  := \frac{\partial{\cal L}}{\partial\dot{\phi}^A},
\label{DensidadMomentoCanonicoConjugado}
\end{equation}
 y una \textit{densidad Hamiltoniana} (suma sobre $A$):
 \begin{equation}
{\cal H}(x,t) := \pi_A(x,t) \dot{\phi}^A(x,t)-{\cal L}(x,t) ,
\label{DensidadHamiltoniana}
\end{equation}
tal que $H=\int {\cal H}dV$ es interpretado como el hamiltoniano del sistema.


\subsection{Densidad lagrangeana para el campo electromagnético}
Queremos ahora aplicar el formalismo lagrangeano para campos al caso particular
del campo electromagnético. Para ello, necesitamos encontrar una densidad
lagrangeana cuyas ecuaciones de campo sean (equivalentes a) las ecuaciones de
Maxwell. Debido a que las ecuaciones de Maxwell son de primer orden en
$F_{\mu\nu}$ no es posible encontrar una densidad ${\cal
L}(F_{\mu\nu},\partial_\lambda F_{\mu\nu})$ que suministre las ecuaciones de
Maxwell como ecuaciones de Euler-Lagrange. Por esto, para aplicar el formalismo
lagrangeano al campo electromagnético es necesario considerar otros campos como
los campos dinámicos. Una solución es considerar a los potenciales $A_\mu$
como variables de configuración, y encontrar a partir de ellos un apropiado
${\cal L}(A_\mu,\partial_\nu A_\mu)$ cuyas ecuaciones de Euler-Lagrange sean las
ecuaciones (\ref{coninhom}), con el tensor electromagnético dado en
términos de los potenciales de acuerdo a (\ref{fda}). Requerimos
además que $\cal L$ sea un escalar bajo TL's para asegurar la
covariancia de las ecuaciones de Euler-Lagrange. Por último, la estructura
general de las ecuaciones de campo (\ref{L-euler}) y la linealidad en los campos
de las ecuaciones de Maxwell sugieren considerar una densidad lagrangiana
cuadrática en las derivadas de los potenciales.

Una densidad lagrangeana que cumple con todos estos requisitos es:
% En unidades Gaussianas
%\begin{equation}
%{\cal L}(A_\mu,\partial_\nu A_\mu) =-\frac{1}{16\pi}F_{\mu\nu}F^{\mu\nu}-\frac{1}{c}J^{\lambda
%}A_{\lambda} .\label{lagF}
%\end{equation}
\begin{equation}
{\cal L}(A_\mu,\partial_\nu A_\mu) =-\frac{1}{4\mu_0}F_{\mu\nu}F^{\mu\nu}-J^\mu A_\mu .\label{lagF}
\end{equation}
Note que, para los propósitos aquí discutidos, la 4-corriente $J^\mu$ se asume
conocida y corresponde por tanto a campos \textit{externos}, tales como el campo
$\Phi$ en el formalismo general. Podemos verificar que el principio variacional
basado en (\ref{lagF}) suministra las ecuaciones de Maxwell como ecuaciones de
movimiento. En nuestro caso, las ecuaciones generales (\ref{L-euler}) adoptan la
forma
\begin{equation}
\frac{\partial{\cal L}}{\partial A_{\nu}}-\partial_\mu\left(  \frac
{\partial{\cal L}}{\partial\partial_\mu A_{\nu}}\right) =0. \label{EUF}
\end{equation}
Las derivadas involucradas se determinan a partir de (\ref{lagF}):
% En Gaussianas
%\begin{align}
%\frac{\partial{\cal L}}{\partial\partial_\mu A_{\nu}}  & =-\frac{1}{4\pi
%}F^{\mu\nu}, \label{dLdda}\\
%\frac{\partial{\cal L}}{\partial A_{\nu}}  & =-\frac{1}{c}J^\nu .
%\end{align}
\begin{align}
\frac{\partial{\cal L}}{\partial\partial_\mu A_{\nu}}  & =-\frac{1}{\mu_0}F^{\mu\nu}, \label{dLdda}\\
\frac{\partial{\cal L}}{\partial A_{\nu}}  & =-J^\nu .
\end{align}
de modo que (\ref{EUF}) se reduce a
%En Gaussianas
%\begin{equation}
%-\frac{1}{c}J^\nu +\frac{1}{4\pi}\partial_\mu F^{\mu\nu}=0,
%\end{equation}
\begin{equation}
-J^\nu +\frac{1}{\mu_0}\partial_\mu F^{\mu\nu}=0,
\end{equation}
que es equivalente a las ecuaciones de Maxwell inhomogeneas \eqref{coninhom}.



\subsection{Tensor de Energía-Moméntum del campo electromagnético}

Tal como se discute en las secciones \ref{sec:energia} y \ref{sec:momentum}, el campo electromagnético posee energía y momentum lineal\footnote{También momentum angular, que no discutiremos en detalle aquí.}. Además, como tanto la energía y el momentum están distribuidos en el espacio y pueden ``fluir'' de una región a otra, es necesario describir su distribución y flujo usando ($1+3$) \textit{densidades}  y ($3+9$) \textit{densidades de flujo}, de energía y momentum lineal, respectivamente. ésta es una característica general de cualquier sistema que posea energía y momentum distribuidos en el espacio(-tiempo), requiriendo en total 16 cantidades para describir completamente sus propiedades. Por otro lado, en RE la energía y el momentum lineal de un cuerpo forman un 4-vector bajo TL's y sus valores cambian (se ``mezclan'', linealmente) de un SRI a otro. Además, tal como en el caso de la carga eléctrica (un escalar bajo TL's), ver sección \ref{ccer}, sus densidades forman el 4-vector densidad de corriente, es natural esperar que cada una de las 4 cantidades que describen el campo electromagnético (las componentes de su 4-momentum) se requieran 4 densidades, análogas a la 4-densidad de corriente. En resumen, se requieren en general 16 cantidades para describir completamente el 4-momentum del campo (sus densidades y flujos). En RE estas 16 cantidades son componentes de un tensor de rango 2 (y por tanto con $4\times 4=16$ componentes independientes), llamado \textit{tensor energía-momentum}. El hecho que estas cantidades sean componentes de un tensor bajo TL's aseguran que sus valores integrados en el espacio (4-momentum) transformen como 4-vectores. 

En el caso del campo electromagnético, el tensor de energía-momentum puede considerarse como dado por
\begin{equation}
\boxed{T_\text{em}^{\mu\nu}:=\frac{1}{\mu_0}\left(  F^{\mu\lambda}F_\lambda^{\ \nu}+\frac{1}{4}F_{\rho\sigma}F^{\rho\sigma}\eta^{\mu\nu} \right),}  \label{temsimSI}
\end{equation}
Verificamos que este tensor es simétrico, es decir, 
\begin{equation}
T_\text{em}^{\mu\nu}=T_\text{em}^{\nu\mu}.
\end{equation}


Escribiendo las componentes del tensor simétrico $\Theta^{\mu\nu}$ en términos
de las componentes de $\vec{E}$ y $\vec{B}$, encontramos que:
%En Gaussianas
%\begin{align}
%\Theta_0{}^0  & =\frac{1}{4\pi}\left(  F^{0\lambda}F_{\lambda0}+\frac
%{1}{2}\left(  \vec{B}^2-\vec{E}^2\right)  \delta_0^0\right) \\
%& =\frac{1}{4\pi}\left(  F^{0i}F_{i0}+\frac{1}{2}\left(
%\vec{B}^2-\vec{E}^2\right)
%\right) \\
%& =\frac{1}{4\pi}\left(  \vec{E}^2+\frac{1}{2}\left(  \vec{B}^2-\vec{E}^2\right)
% \right) \\
%& =\frac{1}{8\pi}\left(  \vec{E}^2+\vec{B}^2\right),
%\end{align}%
%\begin{align}
%\Theta_0{}^i  & =\frac{1}{4\pi}\left(  F^{i\lambda}F_{\lambda0}+\frac
%{1}{2}\left(  \vec{B}^2-\vec{E}^2\right)  \delta_0^i\right) \\
%& =\frac{1}{4\pi}F^{ij}F_{j0}\\
%& =\frac{1}{4\pi}(-\varepsilon^{ijk}B^k)  (-E^j) \\
%& =\frac{1}{4\pi}\left(\vec{E}\times\vec{B}\right)^i,
%\end{align}%
%\begin{align}
%\Theta_j{}^i  & =\frac{1}{4\pi}\left(  F^{i\lambda}F_{\lambda j}+\frac
%{1}{2}\left( \vec{B}^2-\vec{E}^2\right)  \delta_j^i\right) \\
%& =\frac{1}{4\pi}\left(  F^{i0}F_{0j}+F^{ik}F_{kj}+\frac{1}{2}\left(
%\vec{B}^2-\vec{E}^2\right)  \delta_j^i\right) \\
%& =\frac{1}{4\pi}\left(  E^i E^j +\left(-\varepsilon^{ikl}B^l\right)  \left(
%-\varepsilon_{kjm}B^m \right)  +\frac{1}{2}\left(
%\vec{B}^2-\vec{E}^2\right)\delta_j^i\right) \\
%& =\frac{1}{4\pi}\left(  E^i E^j
%-\left(\delta_j^i\delta_{m}^{l}-\delta_{m}^i\delta_j^{l}\right)  B^l B^m
%+\frac{1}{2}\left(  \vec{B}^2-\vec{E}^2\right)\delta_j^i\right) \\
%& =\frac{1}{4\pi}\left(  E^i E^j+B^i B^j -\delta_j^i\vec{B}^2+\frac{1}{2}\left(
%\vec{B}^2-\vec{E}^2\right)  \delta_j^i\right) \\
%& =\frac{1}{4\pi}\left(  E^i E^j +B^i B^j -\frac{1}{2}\left(
%\vec{E}^2+\vec{B}^2\right)\delta_j^i\right).
%\end{align}
\begin{align}
T_\text{em}^{00}  & =\frac{1}{\mu_0}\left[F^{0\lambda}F_\lambda^{\ 0}+\frac
{1}{2}\left(\vec{B}^2-\frac{1}{c^2}\vec{E}^2\right) \eta^{00}\right) \\
& =\frac{1}{4\pi}\left(  F^{0i}F_i^{\ 0}+\frac{1}{2}\left(
\vec{B}^2-\vec{E}^2\right)\right] \\
& =\frac{1}{\mu_0}\left[\frac{1}{c^2}\vec{E}^2+\frac{1}{2}\left( \vec{B}^2-\frac{1}{c^2}\vec{E}^2\right)\right] \\
& =\frac{1}{2\mu_0}\left(\frac{1}{c^2}\vec{E}^2+\vec{B}^2\right),
\end{align}%
\begin{align}
T_\text{em}^{i0}  & =\frac{1}{\mu_0}\left[ F^{i\lambda}F_\lambda^{\ 0}+\frac
{1}{2}\left(  \vec{B}^2-\frac{1}{c^2}\vec{E}^2\right)  \eta^{i0}\right] \\
& =\frac{1}{\mu_0}F^{ij}F_j^{\ 0}\\
& =\frac{1}{\mu_0}(-\varepsilon^{ijk}B^k)  (-\frac{1}{c}E^j) \\
& =\frac{1}{\mu_0c}\left(\vec{E}\times\vec{B}\right)^i,
\end{align}%
\begin{align}
T_\text{em}^{ij}  & =\frac{1}{\mu_0}\left[  F^{i\lambda}F_\lambda^{\ j}+\frac
{1}{2}\left( \vec{B}^2-\frac{1}{c^2}\vec{E}^2\right) \eta^{ij}\right] \\
& =\frac{1}{\mu_0}\left[F^{i0}F_0^{\ j}+F^{ik}F_k^{\ j}-\frac{1}{2}\left(
\vec{B}^2-\frac{1}{c^2}\vec{E}^2\right) \delta^{ij}\right] \\
& =\frac{1}{\mu_0}\left[-\frac{1}{c^2}E^i E^j -\left(-\varepsilon^{ikl}B^l\right)  \left(
-\varepsilon^{kjm}B^m \right) -\frac{1}{2}\left(
\vec{B}^2-\frac{1}{c^2}\vec{E}^2\right)\delta^{ij}\right] \\
& =\frac{1}{\mu_0}\left(-\frac{1}{c^2}E^i E^j
+\left(\delta^{ij}\delta^{lm}-\delta^{im}\delta^{lj}\right)B^l B^m
-\frac{1}{2}\left(\vec{B}^2-\frac{1}{c^2}\vec{E}^2\right)\delta^{ij}\right] \\
& =\frac{1}{\mu_0}\left[-\frac{1}{c^2}E^i E^j - B^i B^j +\delta^{ij}\vec{B}^2-\frac{1}{2}\left(\vec{B}^2-\frac{1}{c^2}\vec{E}^2\right)\delta^{ij}\right] \\
& =\frac{1}{\mu_0}\left[\frac{1}{2}\left(\frac{1}{c^2}\vec{E}^2 +\vec{B}^2\right)\delta^{ij}-\frac{1}{c^2}E^i E^j -B^i B^j\right].
\end{align}
En resumen, podemos indentificar
% Gaussianas
%\begin{eqnarray}
%u&:=&\Theta_0{}^0=\frac{1}{8\pi}\left( \vec{E}^2+\vec{B}^2\right),\\
%S^i&:=&c\Theta_0{}^i=\frac{c}{4\pi}\left(\vec{E}\times\vec{B}\right)^i=c^2\pi^i,\\
%T_j{}^i&:=&\Theta_j{}^i=\frac{1}{4\pi}\left(  E^i E^j +B^i B^j
%-\frac{1}{2}\left(  \vec{E}^2+\vec{B}^2\right)\delta_j^i\right),
%\end{eqnarray}
\begin{align}
u &:= \Theta^{00}=\frac{1}{2}\left(\varepsilon_0\vec{E}^2+\frac{1}{\mu_0}\vec{B}^2\right),\label{uTheta}\\
S^i &:= c\Theta^{0i}=\frac{1}{\mu_0}\left(\vec{E}\times\vec{B}\right)^i=c^2\pi^i, \label{STheta}\\
T^{ij} &:= \Theta^{ij}= \frac{1}{2}\left(\varepsilon_0\vec{E}^2+\frac{1}{\mu_0}\vec{B}^2\right)\delta^{ij}-\varepsilon_0 E^i E^j -\frac{1}{\mu_0}B^i B^j , \label{TTheta}
\end{align}
como la \textit{densidad de energía} $u$, la \textit{densidad de flujo de
energía} $\vec{S}$ (vector de Poynting), \textit{densidad de moméntum}
$\vec{\pi}$ y el \textit{tensor de tensiones} $T^{ij}$ del campo
electromagnético, respectivamente, ya discutidos en las secciones \ref{sec:energia} y \ref{sec:momentum}. En otras palabras, hemos verificado que estas densidades y sus respectivos flujos son componentes de un tensor de rango 2 bajo TL's.

% \begin{equation}
% \Theta^\mu_{\ \nu}
% = \left( \text{\small\begin{tabular}{c | c}
%   Densidad & Flujo de Energía \\
%   de energía & \\ \hline
%   Flujo de & \\
%   Energía   & ``tensiones'' \\
%          &
% \end{tabular}}\right).
% \end{equation}

Podemos verificar además que las leyes de balance de la energía y el momentum del campo pueden escribirse en forma covariante como
\begin{equation}
\boxed{\partial_\nu \Theta^{\mu\nu}+ F^\mu{}_\nu J^\nu=0.}
\label{clem}
\end{equation}
En efecto, para $\mu=0$ esta relación se reduce a
\begin{equation}
\partial_\nu \Theta^{0\nu}+F^0{}_\nu J^\nu=0
\end{equation}
que, luego de identificar las componentes de $\Theta$ de acuerdo a \eqref{uTheta}-\eqref{TTheta}, es equivalente a \eqref{econtEem}, es decir,
\begin{equation}
\frac{\partial u}{\partial
t}+\vec{\nabla}\cdot\vec{S}+\vec{E}\cdot\vec{J}  =0.
\end{equation}
%\begin{equation}
%\frac{\partial}{\partial t}\int_V u\,dV+\oint_{\partial V}
%\vec{S}\cdot d\vec{S}+\int_V \vec{E}\cdot\vec{J}\,dV =0 .\label{clen}
%\end{equation}

Análogamente, las componentes espaciales de (\ref{clem}) suministran
\begin{equation}\label{lcm2}
\frac{\partial \pi^i}{\partial t}+f^i  +\partial_jT^{ij}=0,
\end{equation}
donde $f^i$ son las componentes de la \textit{densidad de fuerza} sobre las cargas:
%Gaussianas
%\begin{equation}
%\vec{f} :=\rho \vec{E}  +\frac{1}{c}\vec{J}\times \vec{B}.
%\end{equation}
\begin{equation}
\vec{f} :=\rho \vec{E}  +\vec{J}\times \vec{B}.
\end{equation}
Note que la relación \eqref{lcm2} es idéntica a \eqref{lcm}, salvo el cambio de notación respecto a la posición de los índices espaciales.
% Finalmente, integrando sobre un volumen $V$, obtenemos:
% \begin{equation}
% \frac{\partial}{\partial t}\left(
% P_{\text{campo}}^i+P_{\text{mecánico}}^i\right)  +\oint_{\partial V}
% T^{ij}\,dS^j=0,
%  \end{equation}
% que representa la conservación del moméntum lineal total del campo y las
% cargas.



% Finalmente, integrando sobre un volumen $V$, obtenemos:
% \begin{equation}
% \frac{\partial}{\partial t}\left(
% P_{\text{campo}}^i+P_{\text{mecánico}}^i\right)  +\oint_{\partial V}
% T^{ij}\,dS^j=0,
%  \end{equation}
% que representa la conservación del moméntum lineal total del campo y las
% cargas.

\subsubsection{4-Corriente para una carga puntual*}
Para una carga puntual con línea de mundo $z^\mu(\tau)=(z^0,\vec{z})$ la
densidad de carga es de la forma
\begin{equation}
\rho(x^\mu)=q\delta^{(3)}\left(\vec{x}-\vec{z}\right) ,
\end{equation}
donde $\delta^{(3)}\left(\vec{x}-\vec{z}\right)$ es una delta de Dirac
tridimensional\footnote{es decir, que satisface $\int_V
f(\vec{x})\delta^{(3)}\left(\vec{x}-\vec{z}\right)=f(\vec{z})$, donde $f$ es una
función arbitraria (en el caso en que el punto $\vec{z}$ está contenido en
$V$).}. La densidad de corriente es, por otro lado, dada por
\begin{equation}
\vec{J}=\rho\vec{v}=q\delta^{(3)}\left(\vec{x}-\vec{z}\right)\vec{v}
\end{equation}
Escribiremos la densidad de corriente en términos de cantidades covariantes
bajo TL's. Para esto introduciremos una delta de Dirac 4-dimensional y la
4-velocidad de la partícula.

Para esto usaremos la identidad
\begin{equation}
\boxed{\delta^{(3)}\left(\vec{x}-\vec{z}(t)\right)=\int u^0(\tau)
\delta^{(4)}\left(x-z(\tau)\right)\, d\tau ,} \label{idDirac}
\end{equation}
donde $\delta^{(4)}\left(x-z(\tau)\right)$ es la delta de Dirac
cuadridimensional, que satisface
$\int_{\Omega_4}f(x^\mu)\delta^{(4)}(x-a)\,d^4x=f(a^\mu)$. En efecto, tenemos
que
\begin{eqnarray}
\int u^0(\tau)\delta^{(4)}\left(x-z(\tau)\right)\, d\tau &=&\int
u^0(\tau)\delta(x^0-z^0(\tau))\delta^{(3)}\left(\vec{x}-\vec{z}(\tau)\right)\,
d\tau\\
&=& \int u^0(\tau)\frac{1}{|\frac{dz^0}{d\tau}(\tau)|}\delta(\tau-\tau_0)
\delta^{(3)} \left(\vec {x} -\vec {z } (\tau)\right)\,d\tau \\
&=& \int \delta(\tau-\tau_0) \delta^{(3)} \left(\vec {x} -\vec {z }
(\tau)\right)\,d\tau,
\end{eqnarray}
donde $\tau_0$ satisface $x^0=ct=z^0(\tau_0)$, para un $t$ dado. Esta relación
permite escribir $\tau_0=\tau_0(t)$ y, por lo tanto,
\begin{eqnarray}
\int u^0(\tau)\delta^{(4)}\left(x-z(\tau)\right)\, d\tau &=&
\delta^{(3)}\left(\vec {x} -\vec {z } (\tau_0)\right) \\
&=&\delta^{(3)}\left(\vec {x} -\vec {z}(\tau_0(t))\right)\\
&=&\delta^{(3)}\left(\vec {x} -\vec {z}(t)\right).
\end{eqnarray}
Usando (\ref{idDirac}) podemos entonces escribir:
\begin{eqnarray}
\vec{J}&=&\rho\vec{v}\\
&=&q\delta^{(3)}\left(\vec{x}-\vec{z}\right)\vec{v} \\
&=& q\int u^0\vec{v}\,\delta^{(4)}\left(x-z(\tau)\right)\,d\tau\\
&=& qc\int\vec{u}\,\delta^{(4)}\left(x-z(\tau)\right)\, d\tau .
\end{eqnarray}
Análogamente, expresamos la densidad de carga como:
\begin{eqnarray}
\rho&=&q\delta^{(3)}\left(\vec{x}-\vec{z}\right) \\
&=& q\int d\tau\, u^0\,\delta^{(4)}\left(x-z(\tau)\right).
\end{eqnarray}
En resumen, podemos escribir la 4-densidad de corriente como
\begin{equation}
\boxed{J^\mu(x)=qc\int d\tau\,
u^\mu\,\delta^{(4)}\left(x-z(\tau)\right).} \label{Jcov}
\end{equation}

\subsubsection{Conservación de la energía y el moméntum para el sistema
campo-carga*}

En el caso de que el sistema de cargas conste de una carga puntual de carga $q$,
la 4-corriente es de la forma (\ref{Jcov}). En este caso, integrando la ley de
conservación (\ref{clem}) en un volumen $V$, obtenemos
\begin{eqnarray}
0&=&\int_V \left[ \partial_\nu \Theta^{\mu\nu}+\frac{1}{c} F^\mu_{\
\nu}J^\nu\right]  dV \\
&=&\int_V \left[\frac{1}{c} \frac{\partial}{\partial t} \Theta^{\mu
0}+\partial_i \Theta^{\mu i}+\frac{1}{c} F^\mu_{\ \nu}J^\nu\right]  dV \\
&=&\int_V \left[\frac{1}{c} \frac{\partial}{\partial t} \Theta^{\mu
0}+\partial_i \Theta^{\mu i}+q\int d\tau\,  F^\mu_{\
\nu}u^\nu\,\delta^{(4)}\left(x^\mu-z^\mu(\tau)\right)\right]  dV \\
&=&\int_V \left[\frac{1}{c} \frac{\partial}{\partial t} \Theta^{\mu
0}+\partial_i \Theta^{\mu i}+c\int d\tau\,
\frac{dp^\mu}{d\tau}\,\delta^{(4)}\left(x^\mu-z^\mu(\tau)\right)\right]  dV \\
&=&\int_V \left[\frac{1}{c} \frac{\partial}{\partial t} \Theta^{\mu
0}+\partial_i \Theta^{\mu i}+c\int dt\,
\frac{dp^\mu}{dt}\,\delta^{(4)}\left(x^\mu-z^\mu(\tau)\right)\right]  dV
\\
&=&\frac{d}{dt}\frac{1}{c}\int_V  \Theta^{\mu 0}\,dV+\oint_{\partial V}\Theta^{\mu
i}dS_i+\int \frac{dp^\mu}{dt}\,\delta^{(4)}\left(x^\mu-z^\mu(\tau)\right)  d^4x
\\
&=&\frac{dP^\mu_{\rm campo}}{dt}+\oint_{\partial V}\Theta^{\mu
i}dS_i+\frac{dp^\mu_{\rm carga}}{dt}.
\end{eqnarray}
Integrando sobre todo el espacio y asumiendo que los campos se anulan
suficientemente rápido de modo que la integral de superficie se anule,
encotramos que el 4-moméntum total del campo y la carga es conservado
\begin{equation}
\frac{d}{dt}\left( P^\mu_{\rm campo}+p^\mu_{\rm carga}\right) =0.
\end{equation}

\subsubsection{Tensor energía-moméntum para una carga puntual*}
Existe una asimetría en la descripción de la energía y el moméntum del
sistema campo-carga. Mientras la carga es descrita en términos de su 4-moméntum
$p^\mu_{\rm carga}$, el contenido de energía-moméntum del campo
electromagnético es descrito por su tensor de energía-moméntum
$\Theta^{\mu\nu}$. Esto es debido al carácter puntual de la carga considerada.

Es posible definir un tensor energía-moméntum para la carga puntual de manera
análoga a como definimos la 4-corriente (\ref{Jcov}). Buscamos un tensor de
energía-moméntum $T_{\rm carga}^{\mu\nu}$ tal que el 4-moméntum de la carga
esté dado por
\begin{equation}
p^\mu_{\rm carga}=\frac{1}{c}\int_V T_{\rm carga}^{\mu 0}\,dV,
\end{equation}
donde $V$ es un volumen que contiene a la carga.
\begin{equation}
T_{\rm carga}^{\mu\nu}(x)=mc\int d\tau \delta^4(x-z(\tau)) u^\mu(\tau)
u^\nu(\tau).
\end{equation}
En efecto:
\begin{eqnarray}
T_{\rm carga}^{\mu 0}&=&mc\int d\tau \delta^4(x-z(\tau)) u^\mu u^0 \\
&=&mc\int d\tau \delta^4(x-z(\tau)) u^\mu \frac{dx^0}{d\tau} \\
&=&mc\int dx^0 \delta^4(x-z(t)) u^\mu  \\
&=&mc\,\delta^3(\vec{x}-\vec{z}(t))\, u^\mu \\
&=&mc u^\mu \, \delta^3(\vec{x}-\vec{z}(t)),
\end{eqnarray}
de modo que
\begin{equation}
\frac{1}{c}\int_V T_{\rm carga}^{\mu 0}\,dV=m u^\mu.
\end{equation}
En términos de este tensor, encontramos que
\begin{eqnarray}
\partial_\nu T_{\rm carga}^{\mu\nu}&=&mc\int d\tau \left[ \partial_\nu
\delta^4(x-z(\tau))\right]  u^\mu(\tau) u^\nu(\tau) \\
&=&-mc\int d\tau \left[ \frac{\partial}{\partial z^\nu}
\delta^4(x-z(\tau))\right]  u^\mu(\tau)u^\nu(\tau) \\
&=&-mc\int d\tau \frac{dz^\nu}{d\tau}\left[ \frac{\partial}{\partial z^\nu}
\delta^4(x-z(\tau))\right]  u^\mu(\tau)\\
&=&-mc\int d\tau \left[\frac{d}{d\tau}\delta^4(x-z(\tau))\right]  u^\mu(\tau)\\
&=&-mc\int d\tau  \frac{d}{d\tau}\left[\delta^4(x-z(\tau))
u^\mu(\tau)\right]+mc\int d\tau  \delta^4(x-z(\tau)) \frac{du^\mu}{d\tau}\\
&=&-mc\left[\delta^4(x-z(\tau)) u^\mu(\tau)\right]^{+\infty}_{-\infty}+c\int
d\tau  \delta^4(x-z(\tau)) \frac{dp^\mu}{d\tau}\\
&=&0+q\int d\tau  \delta^4(x-z(\tau))F^\mu_{\ \nu}(z(\tau))u^\nu(\tau)\\
&=&q\int d\tau  \delta^4(x-z(\tau))F^\mu_{\ \nu}(x)u^\nu(\tau)\\
&=&qF^\mu_{\ \nu}(x)\int d\tau  \delta^4(x-z(\tau))u^\nu(\tau)\\
&=&\frac{1}{c}F^\mu_{\ \nu}J^\nu.
\end{eqnarray}
Con este resultado podemos expresar la conservación (\ref{clem}) del 4-moméntum
total del sistema campo + carga como
\begin{equation}
\boxed{\partial_\nu \left( \Theta^{\mu\nu}+T_{\rm carga}^{\mu\nu}\right) =0.}
\label{clem2}
\end{equation}
