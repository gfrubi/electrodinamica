\chapter{Introducción a la teoría Especial de la Relatividad}

\section{Situación previa a 1905}

La mecánica newtoniana respeta el \textbf{Principio de Relatividad} y el carácter absoluto del tiempo. Esto significa que \textit{todos los SRI's son equivalentes entre sí} o, en otras palabras, que \textit{es imposible distinguir si se está en un SRI o en otro realizando experimentos mecánicos}.
Lo anterior se manifiesta en que las ecuaciones que describen los sistemas mecánicos (por ejemplo, la segunda ley de Newton para partículas bajo interacción gravitacional) son \textbf{invariantes de forma} (o \textbf{covariantes}) bajo las transformaciones (de Galileo) que relacionan las cantidades físicas entre SRI's.

En mecánica newtoniana, y de acuerdo a las transformaciones de Galileo, la
posición y velocidad de un cuerpo  son \textbf{cantidades relativas}, es decir, dependen del SRI respecto al cual se describe el movimiento.

En particular, una \textbf{onda mecánica} (por ejemplo, el sonido) tiene una velocidad de propagación que depende del SRI respecto al que se describan sus propiedades. Al mismo tiempo, existe usualmente un SRI \textit{privilegiado} respecto al cual la velocidad de propagación adquiere algún valor especial o destacado.
Por ejemplo, cuando decimos que ``la velocidad del sonido en el aire es de $\approx 343\,\rm m/s$, ¿respecto a qué SRI estamos refiriendo esta velocidad?. Es instructivo verificar (por ejemplo, analizando la ecuación de las ondas sonoras, derivada a partir de las ecuaciones de un fluido) que la respuesta a esta pregunta puede formularse como ``respecto al SRI en el que las moléculas del aire están, en promedio, en reposo'' o, en forma más condensada ``respecto al SRI comóvil con el aire".

Por otro lado, las ecuaciones que describen la electrodinámica clásica,  las ecuaciones de Maxwell (1864), predicen que existen perturbaciones del campo electromagnético que pueden propagarse\footnote{y transmitir energía, momentum lineal y momentum angular.} con una velocidad bien definida: $v_{\rm em}=c ={1}/{\sqrt{\mu_0\varepsilon_0}}\approx 3\times 10^{8}\,\rm m/s$. Si, como indica la mecánica newtoniana, las velocidades son siempre relativas, ¿con respecto a qué sistema de referencia tienen las ondas electromagnéticas este valor?. La explicación que parece más natural dentro del contexto de la mecánica newtoniana es que \textit{la velocidad de
propagación de las ondas electromagnéticas tiene el valor $c\approx 3\times 10^{8}\,\rm m/s$ sólo con respecto a un SRI particular, especial, que probablemente sea el SR donde el ``medio'' por el cual se propagan estas ondas está en reposo (en analogía con el caso sonoro)}. Los físicos de fines del siglo XIX y comienzos del XX denominaban este medio como el \textbf{éter  (luminífero)}. Adicionalmente, si esto es correcto y la velocidad de la luz es relativa, entonces debe ser posible detectar experimentalmente efectos del cambio de velocidad, por ejemplo, a través de experimentos interferométricos como los de Michelson y Morley.

\subsection{Transformaciones de Galileo*}
Las transformaciones de Galileo, expresan la relación entre las coordenadas espaciotemporales de un evento cualquiera respecto a dos observadores inerciales con velocidad relativa $\vec{v}$, en el contexto de la mecánica de Newton.

Podemos derivar las transformaciones de Galileo usando el principio de
relatividad y el carácter absoluto del tiempo. Considere un evento cualquiera
$A$, con coordenadas $(t,\vec{x})$ con respecto a un SRI $K$ y
$(t',\vec{x}') $ con respecto a otro SRI $K'$. Considere un cuerpo
(ficticio) que se mueve libre de fuerzas y que pasa por el evento $A$. Ya que
este cuerpo ficticio se mueve libremente, su trayectoria será una línea recta,
de modo que
\begin{equation}
\vec{x}(t)=\vec{x}_0+\vec{v}_0 t, \label{xt}
\end{equation}
con respecto a $K$. Del Principio de Relatividad, la trayectoria respecto a $K'$
es de la misma forma, es decir,
\begin{equation}
\vec{x}'(t)=\vec{x}'_0+\vec{v}'_0 t'. \label{xtp}
\end{equation}
Si el tiempo es absoluto, entonces los intervalos de tiempo entre cualquier
evento son iguales, $dt=dt'$. Sincronizando los relojes de $K$ y $K'$ de modo
que concidan inicialmente, $t=0$ para $t'=0$, entonces $t=t'$. De la diferencia
entre (\ref{xt}) y (\ref{xtp}) encontramos que podemos relacionar las
coordenadas del evento (arbitrario) $A$ respecto a $K$ y $K'$ de la forma
siguiente:
\begin{equation}
\vec{x}'_A=\vec{x}_A-\vec{b}-\vec{v}t_A, \label{transGal0}
\end{equation}
con $\vec{b}:=\vec{x}_0-\vec{x}'_0$, y con $\vec{v}:=\vec{v}_0-\vec{v}'_0$. Como el evento $A$ es arbitrario, se acostumbra escribir simplemente
\begin{equation}
\vec{x}'=\vec{x}-\vec{b}-\vec{v}t, \label{transGal}
\end{equation}
 La transformación  (\ref{transGal}), junto con $t=t'$ es conocida como \textbf{transformación de Galileo}. El vector $\vec{b}$ es la posición del origen del sistema $K'$ con respecto al $K$ en el instante $t=0$, mientras que $\vec{v}$ la velocidad de $K'$ respecto a $K$.

Como consecuencia, las velocidades de un cuerpo ``1'' respecto a $K$ y $K'$ se relacionan por
\begin{equation}
\vec{v}'_1=\vec{v}_1-\vec{v}. \label{transGalv}
\end{equation}

\subsection{El experimento de Michelson-Morley}

El experimento de Michelson\footnote{\href{http://es.wikipedia.org/wiki/Albert_Abraham_Michelson}{Albert Abraham Michelson} (1852-1931): físico estadounidense.}-Morley\footnote{\href{http://es.wikipedia.org/wiki/Edward_Morley}{Edward Williams Morley} (1838-1923): químico y físico estadounidense.} \cite{MM87} consistía en un interferómetro, usado para medir la \textit{diferencia de tiempo de vuelo de dos rayos de luz} y, en particular, de cómo estos tiempos de vuelo dependían de la orientación del interferómetro respecto de la dirección de movimiento del supuesto éter. En otras palabras, se esperaba que el interferómetro permitiera detectar efectos del movimiento de la Tierra (es decir, del interferómetro) respecto del éter.

Se lanzaba un rayo de luz de una longitud de onda conocida (con una lámpara de sodio) que incidía en un espejo semiplateado y luego sobre dos espejos, tal como lo indica la figura \ref{fig:michelson}.
\begin{figure}[ht]
\begin{center}
\includegraphics[width=5cm]{fig/fig-interferometro.pdf}
\end{center}
\caption{Esquema del interferómetro} \label{fig:michelson}
\end{figure}

Si existe alguna diferencia entre las longitudes de los brazos del interferómetro, aparecerá un patrón de interferencia. Lo mismo ocurre si los rayos se mueven a velocidades diferentes a lo largo de cada brazo. Supondremos por simplicidad que el éter se mueve con velocidad $v$ a lo largo del eje $AB$.

Suponiendo que la velocidad de la luz tiene el valor conocido $c$ respecto al SR comóvil con el éter y que este último se mueve con velocidad $v$ en la dirección del eje $AB$ entonces, de acuerdo a la mecánica de Newton, la velocidad de la luz respecto al interferómetro, en su viaje de $A$ a $B$, es $c+v$, mientras que en el viaje de regreso de $B$ a $A$, es $c-v$. Por lo tanto, el tiempo de vuelo de $A$ a $B$ y de vuelta a $A$ es
\begin{equation} \label{eq:michelson1}
t_{ABA}=\frac{L_{AB}}{c+v}+\frac{L_{AB}}{c-v}=\frac{2L_{AB}}{c}\frac{1}{1-\beta^2},
\end{equation}
con $\beta:=v/c$. Por otro lado, el tiempo que tarda la luz en ir de $A$ a $C$ es más fácilmente calculado en el SR comóvil con el éter. En este caso, la propagación entre $A$ y $C$ se realiza como indica la figura \ref{fig:michelson2}.
\begin{figure}[H]
\begin{center}
\includegraphics[width=5cm]{fig/fig-interferometro-2.pdf}
\end{center}
\caption{Propagación en el segunda brazo, respecto a un SR comóvil con el éter.} \label{fig:michelson2}
\end{figure}
Aplicando el teorema de Pitágoras al triángulo indicado en la figura \ref{fig:michelson2} encontramos:
\begin{equation}\label{eq:michelson2}
 d^2=c^2t_{AC}^2=L_{AC}^2+(vt_{AC})^2.
\end{equation}
De aquí, obtenemos
\begin{equation}
 t_{AC}=\frac{L_{AC}}{c}\frac{1}{\sqrt{1-\beta^2}}
\end{equation}
y, por lo tanto,
\begin{equation} \label{eq:michelson4}
t_{ACA}=\frac{2L_{AC}}{c}\frac{1}{\sqrt{1-\beta^2}}.
\end{equation}
La diferencia de tiempos de vuelo $\Delta t:=t_{ABA}-t_{ACA}$ es entonces
\begin{equation} \label{eq:michelson5}
\Delta t=\frac{2}{c}\left[\frac{L_{AB}}{1-\beta^2}-\frac{L_{AC}}{\sqrt{1-\beta^2}}\right].
\end{equation}
Suponiendo que la velocidad del interferómetro respecto al éter es pequeña comparada con $c$, entonces podemos expandir en potencias de $\beta$ y obtener
\begin{equation} \label{eq:michelson6}
\Delta t\approx\frac{2}{c}\left[(L_{AB}-L_{AC})+\beta^2(L_{AB}-\frac{1}{2}L_{AC})\right].
\end{equation}
Análogamente, \textit{si se rota el interferómetro noventa grados}, entonces 
\begin{equation}
\Delta t'\approx\frac{2}{c}\left[(L_{AB}-L_{AC})+\beta^2(\frac{L_{AB}}{2}-L_{AC})\right].
\end{equation}

Esta diferencia de tiempo entre ambas orientaciones determina una \textbf{diferencia de fase}:
\begin{equation}
(\Delta\varphi)_{\rm rot}=\omega(\Delta t-\Delta t')=2\pi\nu(\Delta t-\Delta t')\approx 4\pi\frac{L}{\lambda} \beta^2,
\end{equation}
donde $L:=(L_{AB}+L_{AC})/2$ es la longitud efectiva de los brazos del interferómetro.

En el interferómetro usado por Michelson y Morley se usaron múltiples reflexiones, con el fin de aumentar el valor efectivo de $L$, de modo que $L\approx 11\,\rm m$, $\lambda\approx 5.9\times 10^{-7}\,\rm m$, $v\approx 3\times 10^4\,\rm m/s$ (la rapidez de la Tierra en su movimiento de traslación en torno al Sol) y $c \approx 3\times 10^{8}\,\rm m/s$, con lo que $\Delta\varphi \approx 0,4\,\pi$, que corresponde aproximadamente a un corrimiento de \textit{media franja} en el patrón de interferencia, que debería ser posible de observar, dado el error experimental del montaje (que podía detectar corrimientos de aproximadamente $0,01$ franjas).

Aún cuando el experimento se realizó múltiples veces, en diferentes épocas del año, (y se sigue repitiendo hasta la actualidad, cada vez con mayor precisión, ver \cite{MHBSP03}) \textit{el experimento de Michelson-Morley arroja un resultado nulo}, es decir, ninguna variación \textit{observable} del patrón de interferencia al variar la orientación del interferómetro respecto al éter.

Resumiendo: el experimento de Michelson-Morley, realizado con la intensión de detectar variaciones de la velocidad de la luz respecto al éter y de esta forma poder determinar aquel SR especial en el que la velocidad de la luz es $c$, no suministra los resultados esperados. En otras palabras, la nula variación de las franjas de interferencia \textit{impide distinguir físicamente }(dentro del error experimental) aquel supuesto SR en el que el éter está en reposo y la luz parece propagarse con una velocidad independiente del estado de movimiento del interferómetro.

Se intentó explicar el nulo resultado del experimento de Michelson-Morley manteniendo la teoría del éter y la mecánica newtoniana. Por ejemplo, Lorentz y FitzGerald\footnote{\href{{http://es.wikipedia.org/wiki/George_Francis_FitzGerald}}{George Francis FitzGerald} (1851-1901): físico irlandés.} propusieron independientemente que el resultado nulo del experimento de Michelson-Morley podía ser explicado suponiendo que el brazo del interferómetro orientado en la dirección del movimiento respecto al éter se contraía de modo que su longitud sería $L'_{AB}=L_{AB}\sqrt{1-\beta^2}$. Lorentz también investigó la forma que debiesen tener las transformaciones de coordenadas entre dos sistemas de referencia de modo que las ecuaciones de Maxwell (y por lo tanto, sus predicciones) resultaran inalteradas. Se encontró que esto ocurre sólo si además de la contracción de Fitzgerald, se agrega una transformación de la variable temporal. Sin embargo, ninguno de estos intentos condujo a una descripción completa y consistente. En 1905, Albert Einstein propuso su teoría de la Relatividad Especial, que asumiría como \textit{principio} que la velocidad de la luz tiene el \textit{mismo valor en todo SRI}, más aún, la \textit{equivalencia de todos los SRI's}. Einstein además propuso que \textit{la luz se propaga en el vacío}, desechando la idea del éter luminífero. Estos postulados, sin embargo, requieren cambiar los conceptos establecidos (newtonianos) de distancias y tiempos.

\section{Principios de la Teoría de Relatividad Especial}
Einstein construyó la teoría de la Relatividad Especial (RE) adoptando como postulados \textit{i}) el Principio de Relatividad, es decir, la equivalencia completa (\textit{incluyendo fenómenos electromagnéticos}) entre SRI's,  y \textit{ii}) el principio de constancia de la velocidad de la luz.

Antes de revisar el contenido físico de estos principios, introduciremos algunos conceptos básicos. Comenzaremos con los conceptos de \textbf{evento} y \textbf{espaciotiempo}.

Llamaremos \textbf{evento} a ``algo'' que ocurre en una región limitada del
espacio en un corto intervalo de tiempo. \textit{Idealizamos} un evento como un \textit{punto} en el espacio y un \textit{instante} en el tiempo. Todos los procesos que ocurren en el Universo constituyen eventos o conjunto de eventos.
El conjunto de \textit{todos} los eventos en el Universo forma lo que
denominamos \textbf{espaciotiempo}. Supondremos entonces que el espaciotiempo es un espacio (una \textbf{variedad}) \textit{cuadridimensional}.

Ilustraremos los eventos en un \textbf{diagrama de espaciotiempo}. Por ejemplo, un cuerpo con movimiento en una dimensión, ver figura \ref{TER1}. La curva que representa la secuencia de eventos que constituyen la ``historia'' de un cuerpo es llamada \textbf{línea de mundo}.

\begin{figure}[!h]
\centerline{\includegraphics[height=5cm]{fig/fig-diagrama-espaciotiempo.pdf}}
\caption{Diagrama de espaciotiempo. Línea de mundo de una partícula.}
\label{TER1}
\end{figure}

\subsection{Principio de Relatividad}
\begin{quotation}
\ovalbox{Todos los sistemas de referencia inerciales (SRI's) son equivalentes.}
\end{quotation}

Recuérdese que un SRI es aquel con respecto al cual un cuerpo \textit{sobre el que no actúan agentes externos (fuerzas)} se mueve en línea recta a velocidad
constante.


\subsection{Principio de la constancia de la velocidad de la Luz}
 \begin{quotation}
\ovalbox{La velocidad (rapidez) de la luz tiene el mismo valor en todo SRI.}
\end{quotation}
Más detalladamente, esto significa que las señales luminosas se propagan, \textit{en el vacío}, rectilíneamente con la misma rapidez ($=:c$) en todo instante, en todas direcciones, en todos los SRI's, independiente del movimiento de sus fuentes. El resultado del experimento de Michelson-Morley es consistente con el principio de constancia de la velocidad de la luz.

Además, se supone que la interacción electromagnética (``la luz'') se propaga a la \textit{máxima velocidad posible}. En otras palabras, se supone que no existe una señal (más rigurosamente, una forma de propagar energía) que viaje con rapidez mayor a la de la luz.

Es importante descatar que la existencia de una velocidad máxima (finita!) de las interacciones es \textit{incompatible con el concepto de
cuerpo rígido}, puesto que éste es, por definición, un objeto idealizado
que no sufre deformaciones que alteren sus dimensiones (largo, ancho, ...).
Esta propiedad, sin embargo, requiere de la propagación \textit{instantánea} de interacciones en el interior del cuerpo. Es necesario, por tanto, desechar el concepto de ``cuerpo
rígido''. Esto, por otra parte, obliga a desechar el concepto de ``regla
estándar'', que es (era) en el contexto newtoniano frecuentemente usado para definir patrones de longitud (metro patrón, etc.). 
Por lo tanto, en la teoría de RE, la existencia de una velocidad máxima de propagación de las interacciones obliga a reducir el número de cantidades físicas en principio medibles independientemente sin ambig\"uedad, eliminando en general el concepto de distancia como concepto absoluto, y a reemplazarlo por el de medidas de tiempo.

De hecho, la definición actual del patrón de longitud en el Sistema
Internacional (SI), el metro, se basa en medidas de tiempo. En 1983 (en la 17a
conferencia de pesos y medidas) se acordó que\footnote{Recuerde además que, por definición, un segundo es ``la duración de 9.192.631.770 periodos de la radiación correspondiente a la transición entre los dos niveles hiperfinos del estado fundamental del átomo de Cesio 133''. Ver \href{http://physics.nist.gov/cuu/Units/current.html}{esta} página del NIST para un resumen de las unidades en el S.I. e información de la historia asociada a cada unidad.}.
\begin{quotation}
\ovalbox{Un \textbf{metro} es la distancia que la luz recorre en
$1/299792458$ segundos.}
\end{quotation}
Como una consecuencia de esta definición, la rapidez de la luz es hoy una
cantidad \textit{exacta} con valor
\begin{equation}
\boxed{c:=299.792.458\rm\ m/s,}
\end{equation}
ver referencia \cite{CODATA00}.

\subsection{Definiendo posiciones y tiempos de eventos respecto a un SRI}

Ya que no podemos recurrir a ``reglas estándar'', debemos revisar cómo definimos la posición y el tiempo que un (observador en un) SRI asocia a un evento. En general, los únicos tiempos \textit{directamente medibles} son los tiempos medidos por algún reloj (con algún estado de movimiento), \textit{entre eventos sobre su propia línea de mundo}. Éstos son los llamados \textbf{tiempos propios} asociados a un \textit{observador comóvil con el reloj}. Es importante convencerse que no es posible medir \textit{directamente} intervalos de tiempo entre eventos \textit{alejados} de un reloj (de un observador dado).

En general, dado un evento $P$ en el espaciotiempo, \textit{definiremos} la posición (distancia) y tiempo del evento con respecto a un SRI $K$ dado, usando el siguiente procedimiento:

Un(a) observador inercial (puntual) define un SRI $K$ (de modo que él/ella esté  ubicado(a) en su origen) envíando señales luminosas de modo que éstas sean reflejadas en cada evento $P$ y vuelvan al observador. El observador dispone de un reloj,  con el que mide los tiempos de salida de la señal, $T_1$
y llegada $T_2$.
\begin{figure}[!h]
\centerline{\includegraphics[height=5cm]{fig/fig-diagrama-definicion-x-y-t.pdf}}
 \caption{Definición de distancia en términos de tiempos (propios) de vuelo.}
\label{defdist}
\end{figure}
A partir de los datos medibles (tiempos propios) $T_1$ y $T_2$, \textit{definimos} la posición
$x$ de $P$ respecto al observador (distancia observador -- evento $P$) y el
tiempo $t$ que $K$ \textit{le asignará a} $P$ como:
\begin{equation}\label{defxt}
\boxed{x:=\frac{c}{2}(T_2-T_1), \qquad t:=\frac{1}{2}(T_2+T_1).}
\end{equation}
Diremos que, respecto al SRI $K$, el evento $P$ ocurrió ``en la posición $x$ en el tiempo $t$''.

Las relaciones inversas a (\ref{defxt}) son las familiares expresiones
\begin{equation}\label{T1T2xt}
T_1=t-\frac{x}{c}, \qquad T_2=t+\frac{x}{c}.
\end{equation}


Es directo extender este procedimiento a mayores dimensiones (ejes $y$ y $z$), para así definir las coordenas $\vec{x}=(x,y,z)$ del evento $P$ respecto a cada SRI.
De este modo, podemos asociar a cada evento, respecto a un SRI, un conjunto de
coordenadas $x^\mu:=\left(ct,x,y,z\right)=(ct,\vec{x}) $.


\subsection{Relacionando mediciones de tiempo entre dos observadores inerciales}

Por simplicidad, consideraremos el caso simplificado de movimientos en
una dimensión.

Considere dos observadores $A$ y $B$, en reposo y moviéndose con
velocidad constante respecto a un SRI $K$, respectivamente.
\begin{figure}[H]
\centerline{\includegraphics[height=5cm]{fig/fig-diagrama-factor-k.pdf}}
 \caption{Dos observadores inerciales intercambiando señales luminosas.}
\label{k1}
\end{figure}
El observador $A$ envía dos señales luminosas a $B$, en un intervalo de tiempo $(\Delta T)_A$, medido por $A$.

\textit{Supondremos}, como parece razonable, que el intervalo de tiempo medido por $B$ entre la llegada
de los dos pulsos, $(\Delta T)_B$, es proporcional a $(\Delta T)_A$:
\begin{equation}
(\Delta T)_B=k (\Delta T)_A. \label{TAkTB}
\end{equation}
Adicionalmente, \textit{supondremos} que:
\begin{itemize}
\item $k$ es \textit{independiente del tiempo} si $A$ y $B$ son observadores inerciales (OI's) (``homogeneidad bajo
translaciones temporales'').
\item $k$ es \textit{independiente de las posiciones} de $A$ y $B$ (``homogeneidad bajo translaciones'').
\item $k$ es \textit{independiente de la orientación relativa} de $A$ y $B$ (``homogeneidad bajo rotaciones'').
\end{itemize}

Por otro lado, el Principio de Relatividad exige que la relación sea recíproca:
\begin{equation}
(\Delta T')_A=k (\Delta T')_B .\label{tkt}
\end{equation}
Note, sin embargo, que esta relación involucra intervalos de tiempo de \textit{pares de eventos distintos} a los referidos en la relación (\ref{TAkTB}).

\subsection{Velocidad relativa de dos observadores inerciales}

Considere dos OI's $A$ y $B$ que sincronizan sus relojes de
modo que éstos marquen cero en el evento común $O$, y que además intercambian señales luminosas, tal como se ilustra en la figura \ref{fig:k2}.
\begin{figure}[H]
\centerline{\includegraphics[height=5cm]{fig/fig-diagrama-velocidad-relativa.pdf}}
 \caption{Diagrama para determinar la velocidad relativa.}
\label{fig:k2}
\end{figure}
Nos concentraremos en las coordenadas que $A$ y $B$ asignan al evento $P$, que  pertenece a la línea de mundo de $B$. Las coordenadas $(t,x)$ de $P$
respecto a $A$ son
\begin{equation}
t_A(P)=\frac{1}{2}(t_1+t_2)=\frac{1}{2}(T + k^2T)=\frac{1}{2}(1+k^2)T,
\end{equation}
\begin{equation}
x_A(P)=\frac{c}{2}(k^2T-T)=\frac{c}{2}(k^2-1)T.
\end{equation}
A medida que $T$ cambia, las coordenadas del punto $P$, $t_A(P)$ y $x_A(P)$, definen la trayectoria de $B$ respecto de $A$. Por tanto, podemos calcular la \textit{velocidad de $B$ respecto de $A$} como
\begin{equation}
v_{BA}:=\frac{dx_A}{dt_A}(P)=\frac{c(k^2-1)}{1+k^2}.
\end{equation}
Despejando $k_{AB}$ obtenemos
\begin{equation}
k_{AB}=\sqrt{\frac{1+\frac{v_{BA}}{c}}{1-\frac{v_{BA}}{c}}} \,.\label{k}
\end{equation}

Con esto, la relación (\ref{TAkTB}) puede escribirse como
\begin{equation}
(\Delta T)_B=\sqrt{\frac{1+\frac{v_{BA}}{c}}{1-\frac{v_{BA}}{c}}}
\,(\Delta T)_A. \label{k2}
\end{equation}
Observe que esta relación es \textit{universal} en el sentido que conecta los intervalos de tiempo de cada par de eventos en la línea de mundo de dos observadores inerciales, siempre que éstos estén conectados por señales luminosas, tal como lo muestra la figura \ref{k1}.

En particular, podemos aplicar esta relación al caso en que $(\Delta T)_A$ y
$(\Delta T)_B$ son los \textit{periodos de una onda electromagnética} que se propaga de $A$ a $B$. En este caso, podemos reescribir (\ref{tkt}) en términos de la \textbf{frecuencia} de la radiación:
\begin{equation}
\nu_B=\nu_A\sqrt{\frac{c-v_{BA}}{c+v_{BA}}},
\end{equation}
que es la expresión relativista para el cambio de frecuencia debido al movimiento relativo (efecto Doppler). Vemos entonces que:
\begin{itemize}
\item Si $v_{BA}>0$ (alejándose), entonces $k_{AB}>1$, $\nu_B <
\nu_A$.
Tenemos un corrimiento hacia el rojo (\textbf{redshift}).
\item Si $v_{BA}<0$ (acercándose), entonces $k_{AB}<1$, $\nu_B >
\nu_A$.
Tenemos un corrimiento hacia el azul (\textit{blueshift}).
\item Si $v_{BA}$ es reemplazado por $-v_{BA}$, entonces $k_{AB}$ es reemplazado
por $1/{k_{AB}}$, y $\nu_A$ por $\nu_B$, respectivamente.
\end{itemize}
Es común en aplicaciones definir el \textbf{redshift} $z$ por
\begin{equation}
z:=\frac{\nu_A}{\nu_B}-1.
\end{equation}
Podemos verificar que si $|v_{BA}|\ll c$ entonces 
\begin{equation}
z=\frac{v_{BA}}{c}+O(\frac{v_{BA}^2}{c^2}),
\end{equation}
de modo que se recupera la conocida expresión no-relativista.

\subsection{Composición de velocidades}

Considere tres observadores inerciales $A$, $B$ y $C$ que intercambian señales
luminosas como en la figura \ref{k3}.
\begin{figure}[!h]
\centerline{\includegraphics[height= 5cm]{fig/fig-diagrama-composicion-velocidades.pdf}}
 \caption{Tres observadores inerciales: composición de velocidades.}
\label{k3}
\end{figure}
 Entonces tenemos que
\begin{equation}
(\Delta T)_B=k_{AB} (\Delta T)_A, \qquad (\Delta T)_C=k_{BC} (\Delta T)_B,
\qquad (\Delta T)_C=k_{AC} (\Delta T)_A.
\end{equation}
De aquí obtenemos que
\begin{equation}
k_{AC}=k_{AB}k_{BC}.\label{kkk}
\end{equation}
Finalmente, usando las expresiones respectivas de los factores $k$ en función
de las velocidades relativas $v_{CA}$, $v_{BA}$ y $v_{CB}$, es decir,
\begin{equation}
k_{AB}=\sqrt{\frac{1+\frac{v_{BA}}{c}}{1-\frac{v_{BA}}{c}}}, \qquad k_{BC}=\sqrt{\frac{1+\frac{v_{CB}}{c}}{1-\frac{v_{CB}}{c}}} , \qquad k_{AC}=\sqrt{\frac{1+\frac{v_{CA}}{c}}{1-\frac{v_{CA}}{c}}},
\end{equation}
en (\ref{kkk}), encontramos
\begin{equation}
\boxed{v_{CA}=\frac{v_{BA}+v_{CB}}{1+\frac{v_{BA}v_{CB}}{c^2}}.} \label{compvel}
\end{equation}

Esta \textbf{ley relativista de composición de velocidades} (válida para un movimiento unidimensional) establece una \textit{velocidad absoluta}, la
velocidad de la luz. En efecto, si por ejemplo $v_{CB}=c$ (más rigurosamente, en el \textit{límite} $v_{CB}\to c$), entonces $v_{CA}=c$ independientemente del valor de $v_{BA}$. Este resultado es consistente con (en realidad, es una \textit{consecuencia de}) nuestro postulado que la velocidad de la luz tiene el mismo valor con respecto a todo SRI. Además, puede verificarse [hágalo!] que si $|v_{BA}|<c$ y $|v_{CB}|<c$, entonces  (\ref{compvel}) implica que $|v_{CA}|<c$. En otras palabras, la ley relativista de composición de velocidades preserva el hecho que la velocidad de los objetos (reales) siempre es menor que la velocidad de la luz, independiente del SRI respecto al cual se determine esta velocidad. Note que esto \textit{no} ocurría en la teoría newtoniana.

% Por tanto, existen tres clases de partículas que en principio podrían existir:
% \begin{itemize}
% \item Partículas con $v>c$: \textit{subluminales}. Todas las partículas
% masivas conocidas ($e^-$, $n$, \dots).
% \item Partículas con $v=c$: \textit{luminales}. Todas las partículas sin
% masa conocidas ($\gamma$, antes también el $\nu$).
% \item Partículas con $v>c$: \textit{superluminales}. Ninguna conocida (violan
% causalidad), se denominan genericamente \textit{takiones}.
% \end{itemize}


\subsection{Experimento de Fizeau*}
La ley relativista de composición de velocidades explica muy simplemente el
resultado del experimento de Fizeau (1851), quien midió con técnicas
interferométricas, la velocidad de propagación de la luz \textit{en el agua}
y su dependencia con la velocidad del agua.
\begin{figure}[!h]
\centerline{\includegraphics[height=5cm]{fig/fig-fizeau.pdf}}
 \caption{Diagrama del interferómetro de Fizeau}
\label{fizeau}
\end{figure}
él encontró que la velocidad de la luz respecto al agua es dada por
\begin{equation}
v'=\frac{c}{n}-\left(1-\frac{1}{n^2}\right)v_{\rm agua} ,
\end{equation}
donde $n$ es el índice de refracción del agua, moviéndose con velocidad $v_{\rm agua}$.

Esta expresión puede ser obtenida, a segundo orden, de (\ref{compvel}) con
$v_{CA}=v'$, $v_{BA}=-v_{\rm agua}$ y $v_{CB}=c/n$ como la velocidad de la luz
respecto del agua en movimiento, del agua en reposo respecto al agua en
movimiento y de la luz respecto al agua en reposo, respectivamente ($A=$ agua
en movimiento, $B=$ agua en reposo, $C=$ luz):
\begin{equation}
v'=\frac{-v_{\rm agua}+\frac{c}{n}}{1-\frac{v_{\rm
agua}}{nc}}=\left(\frac{c}{n}-v_{\rm agua}\right)\left(1+\frac{v_{\rm
agua}}{nc}+O(\frac{v_{\rm agua}^2}{c^2})\right)=\frac{c}{n}-v_{\rm
agua}+\frac{v_{\rm agua}}{n^2}+O(\frac{v_{\rm
agua}^2}{c^2}).
\end{equation}
Antes de la formulación de la teoría especial de la relatividad, los
resultados de este experimento se explicaban suponiendo que el agua en movimiento
``arrastraba'' parcialmente al éter, en una fracción ad-hoc.


\subsection{Boosts de Lorentz} \label{secboostx}
\begin{figure}[!h]
\centerline{\includegraphics[height= 5cm]{fig/fig-diagrama-boost.pdf}}
 \caption{Diagrama para un boost.}
\label{boo}
\end{figure}
Ahora derivaremos la relación entre las coordenadas espaciotemporales $(t,x)$
y $(t',x')$ que dos SRI's $K$ y $K'$ asocian a un \textit{mismo evento} $P$, ver figura \ref{boo}. Por simplicidad, consideraremos que $x>0$, $x'>0$.

De acuerdo a las expresiones (\ref{T1T2xt}), como $P$ tiene coordenadas $(t,x)$ respecto de $K$, entonces el evento $E$ ocurre en el tiempo (propio) $t_E=t-x/c$ respecto de $K$ y la ``respuesta'' llega en el evento $F$, con tiempo $t_F=t+x/c$. Similarmente, el evento $C$ ocurre en $t'_C=t'-x'/c$ respecto a $K'$, mientras que $D$ ocurre $t'_D=t'+x'/c$.

Si $K$ y $K'$ tienen sus relojes sincronizados cuando pasan por el evento común $O$, entonces tenemos, del triángulo $OEC$ de la figura \ref{boo}, que $t'_C=k\,t_E$, es decir,
\begin{equation}\label{rb01}
t'-\frac{x'}{c}=k\left(t-\frac{x}{c}\right),
\end{equation}
donde $k$ depende de la \textit{velocidad relativa $V$ de $K'$ respecto a $K$}:
\begin{equation}\label{kbeta}
k=\sqrt\frac{1+\beta}{1-\beta}, \qquad \beta:=\frac{V}{c}.
\end{equation}
Además, del triángulo $ODF$ de la figura \ref{boo}, obtenemos $t_F=k\,t'_D$, es decir,
\begin{equation}\label{rb02}
t+\frac{x}{c}=k\left(t'+\frac{x'}{c}\right).
\end{equation}
Usando \eqref{rb01}, \eqref{kbeta} y \eqref{rb02} podemos despejar $x'$ y $t'$ en función de $x$, $t$ y la velocidad relativa $V$ (de $K'$ con respecto a $K$), obteniendo
\begin{equation}
t'=\frac{t-\frac{Vx}{c^2}}{\sqrt{1-\frac{V^2}{c^2}}}, \qquad
x'=\frac{x-Vt}{\sqrt{1-\frac{V^2}{c^2}}}. \label{b1}
\end{equation}
Definimos el ``factor de Lorentz'' $\gamma:={1}/{\sqrt{1-\beta^ 2}}$, de modo que podemos escribir:
\begin{equation}\marginnote{Boost 1D}
\boxed{ct'=\gamma (ct-\beta x), \qquad x'=\gamma(x-\beta ct).} \label{boost1}
\end{equation}

Esta transformación es conocida como un \textbf{boost} (o transformación de Lorentz simple) en la dirección $x$ y representa la transformación de las coordenadas espaciotemporales \textit{de un mismo evento} desde un SRI $K$ a otro $K'$, que se mueve con velocidad $V$ con respecto al primero.

Es directo verificar que:
\begin{itemize}
\item Despejando $(t,x)$ en función de $(t',x')$ se obtiene la misma
dependencia funcional que en (\ref{boost1}), pero reemplazando $V$ por $-V$, es decir,
\begin{equation}
\boxed{ct=\gamma (ct'+\beta x'), \qquad x=\gamma(x'+\beta ct').} \label{boost2}
\end{equation}
ésta es una manifestación del Principio de Relatividad.

\item Dos boosts sucesivos con velocidades $V_{BA}$ y $V_{CB}$ es
equivalente a un único boost con velocidad $V_{CA}$, dada por (\ref{compvel}).

\item En el límite en que la velocidad de la luz fuese infinita, es decir, si la luz se propagase instantáneamente, las transformaciones (\ref{b1}) se reducen a las \textbf{transformaciones de Galileo}: $t'=t$, $x'=x-Vt$.

\item A partir de (\ref{boost1}) podemos derivar nuevamente la ley relativista de composición de velocidades. Si $x(t)$ es la trayectoria de una partícula respecto al SRI $K$, entonces su velocidad es definida como $v:=dx/dt$. Similarmente, $v'=dx'/dt'$ es la velocidad respecto al SRI $K'$. Diferenciando (\ref{boost1})  obtenemos:
\begin{equation}
 v'=\frac{dx'}{dt'}=\frac{c\gamma(dx-\beta c dt)}{\gamma(cdt-\beta dx)}=\frac{c(v-\beta c)}{(c-\beta v)}=\frac{v-V}{1-\frac{V\cdot v}{c^2}}.
\end{equation}
La inversa de esta relación puede nuevamente obtenerse [compruébelo!] sustituyendo $V$ por $-V$, es decir,
\begin{equation}
v=\frac{v'+V}{1+\frac{V\cdot v'}{c^2}},
\end{equation}
que coincide con (\ref{compvel}) (en este caso el SRI $C$ corresponde a un sistema comóvil con la partícula y entonces $v_{CA}=v$, $v_{BA}=V$, $v_{CB}=v'$).

\item Considere dos eventos, $P$ y $Q$, muy próximos en el espaciotiempo. Sus coordenadas respecto a un SRI $K$ son $(ct,x)$ y $(ct+cdt,x+dx)$. Tanto $dx^2$ como $dt^2$ cambian sus valores de un SRI a otro, es decir,  \textit{no son invariantes bajo un boost de Lorentz} (\ref{boost1}). Esto significa que (en nuestro ejemplo unidimensional) la distancia que un SRI asocia a un par de eventos no es una cantidad absoluta en RE, y similarmente para los intervalos de tiempo entre dos eventos. Sin embargo, la combinación
\begin{equation}
ds^2:=c^2dt^2-dx^2
\end{equation}
\textit{sí es invariante} bajo (\ref{boost1}), es decir, tiene el mismo valor en cualquier SRI. En efecto, un cálculo directo a partir de (\ref{boost1}) muestra que
\begin{equation}
ds'^2:=c^2dt'^2-dx'^2=c^2dt^2-dx^2=ds^2.
\end{equation}
En este sentido $ds^2$ es una \textit{cantidad absoluta} en la teoría de RE: no importa en qué SRI se calcule, su valor es siempre el mismo. Esta importante cantidad recibe el nombre de \textbf{intervalo}. Note que en general $ds^2$ puede asumir valores positivos, negativos o nulos, dependiendo de la elección de los eventos $P$ y $Q$.

En el caso de \textit{eventos sobre la trayectoria de una partícula}, es posible encontrar una simple interpretación para el valor del intervalo: ya que $ds^2$ puede calcularse en cualquier SRI y siempre se obtendrá el mismo valor, en particular en el SRI comóvil con la partícula (es decir, en el SRI con respecto al cual la partícula está en reposo entre $P$ y $Q$) $dx_{\rm com}=0$ y $dt_{\rm com}=d\tau$ coincide con el intervalo de \textit{tiempo propio}, ya que sería el intervalo de tiempo que mide un observador que se mueve con la partícula. Entonces $ds^2=c^2d\tau^2$. En otras palabras $d\tau=\sqrt{ds^2}/c=ds/c$ es el intervalo de tiempo propio medido por un observador que se mueve con la partícula. Note que en estos casos $ds^2>0$.

\end{itemize}




\subsection{Relatividad de la Simultaneidad}
Consideraremos aquí el ejemplo clásico de cómo dos eventos que son simultáneos respecto a un SRI no lo son respecto de otro en movimiento relativo. Para ello, considere un vagón de tren provisto de una lámpara en su centro que en el evento $O$ emite una señal luminosa hacia los dos extremos del vagón.
\begin{figure}[H]
\centerline{\includegraphics[height= 5cm]{fig/fig-diagrama-simultaneidad-01.pdf}}
\caption{Los eventos $P$ y $Q$ son simultáneos respecto al SRI $K'$, comóvil con el tren. Adaptada a partir de \href{https://commons.wikimedia.org/wiki/File:Traincar_Relativity1.svg}{esta} figura original.}
\label{sim01}
\end{figure}
En el SRI $K'$ comóvil con el tren, el proceso transcurre como se representa en la figura \ref{sim01}. Si en este caso el largo del vagón es $d$ y se elige el origen del tiempo tal que $t'_O=0$, entonces las señales luminosas (los ``fotones'') llegan (chocan) con las paredes izquierda y derecha en los eventos $P$ y $Q$, en el mismo tiempo $t'_P=t'_Q=d/2c$. Por lo tanto, los eventos $P$ y $Q$ son simultáneos respecto a este SRI $K'$.
\begin{figure}[H]
\centerline{\includegraphics[height= 5cm]{fig/fig-diagrama-simultaneidad-02.pdf}}
\caption{Los eventos $P$ y $Q$ no son simultáneos respecto al SRI $K$, respecto al cual el tren se mueve con velocidad $V$. Adaptada a partir de \href{https://commons.wikimedia.org/wiki/File:Traincar_Relativity2.svg}{esta} figura original.}
\label{sim02}
\end{figure}

Por otro lado, la descripción de este mismo proceso desde el SRI $K$ ``fijo a la Tierra'', respecto al cual el vagón se mueve con velocidad $V=\beta c$, se esquematiza en la figura \ref{sim02}. En este caso, la señal luminosa se propaga, de acuerdo a los postulados de la teoría, con (la misma) velocidad $c$ hacia ambos lados. Sin embargo, las paredes del vagón se mueven con velocidad $V$ hacia la derecha de modo que a pared izquierda se mueve hacia la señal mientras que la pared derecha se aleja de ella. Claramente, en estas condiciones el evento $P$ (de intersección de la línea de mundo de la pared izquierda con la línea de mundo del fotón) ocurre \textit{antes que el evento} $Q$. Por lo tanto, respecto al SRI $K$ los eventos $P$ y $Q$ no son simultáneos. Esto significa que, en la teoría de RE, \textit{la simultaneidad de eventos es relativa}.

Podemos cuantificar lo anterior usando la transformación de Lorentz (\ref{boost1}). Si los observadores en $K$ y $K'$ sincronizan sus relojes en el evento $O$ en que los rayos de luz dejan el punto medio (se ``enciende la lámpara''), entonces
\begin{eqnarray}
x'^\mu_O&=&(ct'_O,x'_O)=(0,0), \\
x'^\mu_P&=&(ct'_P,x'_P)=(d/2,-d/2), \\
x'^\mu_Q&=&(ct'_Q,x'_Q)=(d/2,d/2).
\end{eqnarray}
Usando (\ref{boost2}), encontramos que las coordenadas de estos eventos respecto al SRI $K$ son
\begin{eqnarray}
x^\mu_O&=&(ct_O,x_O)=(0,0), \\
x^\mu_P&=&(ct_P,x_P)=\left(\frac{\gamma d}{2}(1-\beta),-\frac{\gamma d}{2} (1-\beta)\right), \\
x^\mu_Q&=&(ct_Q,x_Q)=\left(\frac{\gamma d}{2}(1+\beta),\frac{\gamma d}{2} (1+\beta)\right).
\end{eqnarray}
Estas coordenadas, junto con las líneas de mundo de los fotones y las paredes del vagón son representadas en la figura \ref{sim03-04} desde el punto de vista de los SRI's $K$ y $K'$.
\begin{figure}[!h]
\centerline{\includegraphics[height= 4cm]{fig/fig-diagrama_simultaneidad-03.pdf}\hspace{1cm}
\includegraphics[height= 4cm]{fig/fig-diagrama_simultaneidad-04.pdf}}
\caption{Diagramas de espaciotiempo para el proceso, respecto a los SRI $K'$ y $K$. Adaptadas a partir de \href{https://en.wikipedia.org/wiki/File:TrainAndPlatformDiagram1.svg}{esta} y \href{https://en.wikipedia.org/wiki/File:TrainAndPlatformDiagram2.svg}{esta} figura original.}
\label{sim03-04}
\end{figure}

Claramente, obtenemos $t_P<t_Q$ de modo que respecto a $K$ el evento $P$ ocurre antes que el evento $Q$. La diferencia de tiempo entre estos eventos es entonces dada por:
\begin{equation}
 \Delta t_{PQ}:=t_Q-t_P=\frac{1}{c}\gamma \beta d=\gamma\frac{V}{c^2}d >0.
\end{equation}


Por lo tanto, encontramos que respecto al SRI $K$ los eventos $P$ y $Q$ no son simultáneos: $P$ acontece antes que $Q$ y el intervalo de tiempo entre estos dos eventos es $\gamma{V}d/{c^2}$. Similarmente, respecto a un SRI $K''$ en el que el vagón se mueva hacia la izquierda, el evento $Q$ ocurrirá antes que $P$. Como veremos más adelante, la relatividad de la simultaneidad no contradice el principio de causalidad puesto que \textit{no existe relación causal entre los eventos} $P$ y $Q$.


\subsection{Dilatación del tiempo}
Considere dos eventos $P$ y $Q$ que, con respecto a un SRI comóvil $K'$, ocurren en la misma posición, pero en tiempos diferentes. Estos mismos dos eventos ocurren en posiciones diferentes y \textit{con un intervalo de tiempo también diferente} respecto a otro SRI $K$ en movimiento relativo respecto a $K'$. 

En efecto, en este caso $(\Delta x')_{PQ}=0$ y $(\Delta t')_{PQ}=(\Delta\tau)_{PQ}\neq 0$ y por lo tanto, usando las transformaciones (\ref{boost2}) tenemos que
\begin{equation}
(\Delta t_{PQ})=\gamma (\Delta\tau)_{PQ}, \qquad (\Delta x)_{PQ}=\gamma\beta c\,(\Delta\tau)_{PQ}.
\end{equation}
De este modo encontramos que $(\Delta t)_{PQ}>(\Delta\tau)_{PQ}$, es decir, que en el SRI $K$ se asignará un intervalo de tiempo mayor que el que $K'$ asociará \textit{al mismo par de eventos}. Un ejemplo clásico de \textit{esta dilatación del tiempo} lo constituye la vida media de las partículas. Si la vida media de una partícula es\footnote{Clásicamente la vida media de una partícula es ``el tiempo que ella tarda en desintegrarse cuándo está en reposo'' o, más precisamente, el tiempo que debe transcurrir para que una población de un gran número de partículas idénticas y en reposo se reduzca en una fracción $1/e$.} $\tau_0$, entonces la vida media de ella con respecto a un observador que la ve moverse con rapidez $v$ es $\tau=\gamma\tau_0 >\tau_0$.

Esta dilatación del tiempo es un efecto \textit{universal} que afecta a todo tipo de eventos y en particular a la ``velocidad de avance'' de relojes en movimiento. Esto se ilustra en el siguiente ejemplo.

\subsubsection{Reloj de Luz}
Considere el ``reloj de luz'' formado por un rayo de luz que rebota entre dos espejos paralelos separados una distancia $L$, como se esquematiza en la figura \ref{dt}.
\begin{figure}[!h]
\centerline{\includegraphics[height= 4cm]{fig/fig-diagrama-dilatacion-01.pdf}\hspace{1cm}
\includegraphics[height=4cm]{fig/fig-diagrama-dilatacion-02.pdf}}
 \caption{Dilatación del tiempo en el ``reloj de luz''. Izquierda: SRI comóvil $K'$. Derecha SRI $K$ respecto al cual el reloj se mueve con velocidad $v=\beta c$. Figuras adaptadas a partir de \href{http://en.wikipedia.org/wiki/File:Time-dilation-001.svg}{esta} y \href{http://en.wikipedia.org/wiki/File:Time-dilation-002.svg}{esta} figura original.}
\label{dt}
\end{figure}
Respecto al SRI comóvil $K'$ el tiempo entre la salida y la llegada del rayo de luz (un ``tic'' del reloj) es dado por
\begin{equation}
\Delta t'=\frac{2L}{c}.
\end{equation}
Por otro lado, respecto al SRI $K$ el rayo de luz tarda un tiempo $(\Delta t)$ en subir y bajar. Este tiempo puede ser determinado usando el teorema de Pitágoras en el triángulo indicado en la figura:	
\begin{equation}
\left(\frac{c\Delta t}{2}\right)^2=L^2+\left(\frac{v\Delta t}{2}\right)^2.
\end{equation}
A partir de esta relación, encontramos que
\begin{equation}
\Delta t=\gamma\,\Delta t'.
\end{equation}

\subsubsection{Derivación usando el intervalo}

Otra forma de derivar este resultado es usando directamente el intervalo,
\begin{equation}
 ds^2=c^2d\tau^2=c^2dt^2-dx^2=c^2dt^2(1-\beta^2)=\gamma^{-2}c^2dt^2,
\end{equation}
de modo que $dt=\gamma d\tau>d\tau$.


En definitiva, \textit{todo intervalo de tiempo que transcurre entre dos eventos dados parece mayor (dilatado) cuando se observa desde un SRI en movimiento, comparado con el intervalo de tiempo entre los mismos eventos respecto a un observador comóvil}.

Esta predicción de la teoría de la relatividad especial ha sido confirmada experimentalmente en múltiples ocasiones a través de mediciones del \textbf{efecto Doppler (transversal)} (experimento sugerido por el mismo Einstein en 1907 \cite{Einstein07}), ya que como consecuencia de la dilatación temporal, la radiación emitida por un átomo en movimiento transversal a la dirección de observación, presentará una \textbf{variación relativa de la longitud de onda} de la radiación emitida $\Delta\lambda/\lambda_0=\gamma-1\approx 10^{-5}$. La primera confirmación usando este efecto es debida a Ives y Stilwell en 1938 \cite{IS38}, usando átomos de Hidrógeno  ($v/c\approx 0.005$, $\gamma-1\approx 10^{-5}$). En 2003 Saathoff et. al. publicaron resultados de un test mejorado, usando ``espectroscopía laser de iones rápidos" \cite{Saathoff03} ($v/c\approx 0.064$, $\gamma-1\approx 2\times 10^{-3}$, $1\%$), y luego en 2007 Reinhardt et al. mejoraron la precisión usando ``relojes ópticos atómicos" \cite{Reinhardt07} ($v/c\approx 0.03$, $\gamma-1\approx 4\times 10^{-4}$). El ``record"\, lo tienen actualmente Chou et al., quienes lograron verificar la dilatación temporal con ``relojes atómicos" (de átomos aluminio) moviéndose a velocidades del orden de 10 m/s ! \cite{Chou10} ($v\approx 10 [m/s]$, $v/c\approx 3\times 10^{-8}$, $\gamma-1\approx 4\times 10^{-16}$). Otros métodos para verificar la dilatación temporal consisten en medir la vida media de partículas \cite{Bailey77} ($v/c\approx 0.9994$, $\gamma-1\approx 28$), y comparar las medidas de relojes atómicos en movimiento relativo (``paradoja de los gemelos'') \cite{HK72a,HK72b}. Para otras referencias sobre la confirmación experimental de la teoría de Relatividad Especial, ver \cite{SRexp}.

\subsection{Contracción de la longitud}
Considere un cuerpo, cuyos extremos están, respecto a un SRI $K'$ comóvil, fijos en las posiciones $x'_P$ y $x'_Q=x'_R$. Respecto a este SRI la longitud del cuerpo es $L_0=x'_Q-x'_P$. En otro SRI, respecto al cual el cuerpo se mueve con velocidad $v$, \textit{se define} la longitud $L$ como \textit{la distancia entre las posiciones de los puntos extremos del cuerpo en el mismo instante}.
\begin{figure}[!h]
\centerline{\includegraphics[height= 5cm]{fig/fig-diagrama-contraccion.pdf}}
 \caption{Líneas de mundo de los extremos de un cuerpo, respecto al SRI comóvil $K'$ y SRI $K$ donde éste se mueve con velocidad $v$.}
\label{fcl}
\end{figure}

 Usando la transformación de Lorentz para la posición tenemos que el evento $P$ (un extremo del cuerpo) con coordenadas $(ct_A,x_P)$ respecto al SRI $K$ tendrá, respecto al SRI $K'$ la siguiente posición:
\begin{equation}
x'_P=\gamma(x_P-\beta c t_A).
\end{equation}
Por otro lado, el evento $Q$ (en el otro extremo del cuerpo) que es \textit{simultáneo al evento} $P$ respecto al SRI $K$, tiene coordenadas
\begin{equation}
x'_Q=\gamma(x_Q-\beta c t_A).
\end{equation}
Por lo tanto,
\begin{equation}
x'_Q-x'_P=\gamma(x_Q-x_P),
\end{equation}
En otras palabras, los largos del cuerpo respecto a los dos SRI's están relacionados por
\begin{equation}
L=\frac{L_0}{\gamma}<L_0,
\end{equation}
donde $L_0:=x'_Q-x'_P$ es el largo del cuerpo respecto al SRI comóvil, también llamado \textbf{longitud en reposo} del cuerpo.

Resumiendo, \textit{todo cuerpo que tiene una longitud $L_0$ respecto a un SRI comóvil parece tener una longitud \textbf{menor} (contracción) en un SRI respecto al cual el cuerpo está en movimiento}.

\subsection{El cono de luz}
En la teoría de la Relatividad, un muy útil concepto es el \textbf{cono de luz}. Por definición, el cono de luz asociado a un evento $O$ es el \textit{conjunto de eventos en el espaciotiempo que pueden ser conectados con $O$ por señales luminosas}.
\begin{figure}[!h]
\centerline{\includegraphics[height= 5cm]{fig/fig-cono-de-luz-1D.pdf}}
 \caption{Cono de luz en 1+1 dimensiones: futuro, pasado absoluto y ``limbo''}
\label{lc}
\end{figure}
El cono de luz divide el espaciotiempo en tres regiones: ``dentro del cono de luz'', ``fuera del cono de luz'' (el ``limbo'') y ``sobre el cono de luz''. Esta clasificación es \textit{absoluta}, en el sentido que no depende del SRI respecto al cual se describan los eventos.

Los eventos \textit{dentro} del cono de luz asociado a $O$ son aquellos que \textit{pueden tener conexión causal} con $O$. Es posible además separar los eventos dentro del cono de luz en dos regiones: el \textbf{futuro absoluto} y el \textbf{pasado absoluto}. Los eventos en el futuro absoluto de $O$ son aquellos que pueden (en principio) ser alcanzados por partículas o señales emitidas desde  $O$  y que viajen con velocidades menores que la de la luz. Análogamente, los eventos en el pasado absoluto de $O$ son aquellos desde los cuales pueden ser emitidas partículas o señales viajando con velocidades menores que la de la luz y que pueden eventualmente llegar a $O$. 

Por otro lado, los eventos \textit{fuera} del cono de luz no tienen conexión causal con $O$, no pueden afectar a $O$, ni $O$ puede afectar eventos en esa región, ya que eso requeriría señales propagándose con velocidades superluminales.

La clasificación de eventos asociada el cono de luz está relacionada con los valores que asume el intervalo entre dos eventos. Si $(t_O,x_O)$ son las coordenadas del evento $O$ y $(t,x)$ las de un evento $P$ cualquiera, entonces podemos calcular $\Delta s^2:=c^2(\Delta t)^2-(\Delta x)^2$. Este valor puede ser positivo, negativo, o nulo.
\begin{itemize}
\item Si $\Delta s^2>0$ entonces el evento $P$ está dentro del cono de luz de $O$. Se dice que el vector $OP$ es \textbf{tipo tiempo}. Si además $\Delta t>0$ entonces $P$ está en el futuro absoluto de $O$, y si $\Delta t<0$, en su pasado absoluto. Los eventos $O$ y $P$ pueden tener conexión causal. Es posible encontrar un SRI respecto al cual $O$ y $P$ tienen la misma posición espacial, e.d. un SRI comóvil con estos eventos. No es posible encontrar un SRI respecto al cual $O$ y $P$ sean simultáneos.

\item Si $\Delta s^2<0$ entonces el evento $P$ está fuera del cono de luz de $O$. Se dice que el vector $OP$ es \textbf{tipo espacio}. Los eventos $O$ y $P$ no tienen conexión causal. Es posible encontrar un SRI respecto al cual $O$ y $P$ son simultáneos. También es posible encontrar SRI's respecto a los cuales $P$ anteceda a $O$, y viceversa. En otras palabras, en este caso $O$ y $P$ no tienen un orden temporal absoluto. No es posible encontrar un SRI respecto al cual $O$ y $P$ tienen la misma posición espacial.

\item Si $\Delta s^2=0$ entonces el evento $P$ está sobre del cono de luz de $O$. Se dice que el vector $OP$ es \textbf{tipo luz}. No es posible encontrar un SRI respecto al cual $O$ y $P$ son  simultáneos. Tampoco es posible encontrar un SRI respecto al cual $O$ y $P$ tienen la misma posición espacial. Sólo señales luminosas (o, en general, partículas o señales que se muevan a la velocidad de la luz) pueden conectar $O$ y $P$.
\end{itemize}

\begin{figure}[t]
\centering\includegraphics[height=7cm]{fig/fig-cono-de-luz-2D.pdf}
\caption{Cono de Luz, en 2+1 dimensiones.}
\end{figure}



\subsection{Boost en una dirección arbitraria}
En la sección \ref{secboostx} supusimos implícitamente que las coordenadas asociadas a las direcciones normales ($y$ y $z$) a la velocidad relativa de los dos SRI's (eje $x$) no son alteradas por la transformación (la contracción de Lorentz sólo se produce en la dirección de movimiento). En otras palabras, las transformaciones \eqref{boost1} son complementadas con $y'=y$ y $z'=z$. Podemos generalizar este resultado al caso en que la velocidad relativa entre los SRI's $K$ y $K'$ está dirigida en una dirección arbitraria, separando los vectores $\vec{x}$ y $\vec{x}'$ en componentes paralelas y perpendicular a la velocidad relativa $\vec{V}=c\vec{\beta}$. Usando el hecho que la componente perpendicular es inalterada, y además el boost unidimensional \eqref{boost1} para la componente paralela, es decir, $\vec{x}'_\perp=\vec{x}_\perp$ mientras que  $x'_\parallel=\gamma(x_\parallel-\beta ct)$, con $x_\parallel:=\hat{\beta}\cdot\vec{x}$, $\vec{x}_\perp=\vec{x}-x_\parallel\hat{\beta}$, $\hat{\beta}=\vec{V}/V=\vec{\beta}/\beta$, etc., es simple verificar que la TL correspondiente a un boost en una dirección arbitraria es de la forma
\begin{equation}\label{bg1}\marginnote{Boost dirección arbitraria}
\boxed{\vec{x}' =\vec{x}+\frac{\left(  \gamma-1\right)  }{\beta^2}\left(  \vec{\beta}\cdot\vec{x}\right)  \vec{\beta}-\gamma ct\vec{\beta},}
\end{equation}
\begin{equation}\label{bg2}
\boxed{ct' = \gamma\left( ct-\vec{\beta}\cdot\vec{x}\right).}
\end{equation}
Esta TL describe el cambio entre los SRI's $K$ y $K'$, tal que $K'$ se mueve con velocidad $\vec{V}=c\vec{\beta}$ respecto a $K$, pero donde los ejes espaciales de $K$ y $K'$ son paralelos (en este sentido, esta TL \textit{no incluye rotaciones}).

\subsubsection{Composición de velocidades.}

Considere una partícula moviéndose con una velocidad
$\vec{v}={d\vec{x}}/{dt}$ en el sistema $K$
\textquestiondown Con qué velocidad se mueve la partícula en el
sistema $K'$, que se mueve con una velocidad $c\vec{\beta}$ con respecto al sistema $K$?.

Considere la TL de un boost general dada por (\ref{bg1})-(\ref{bg2}). En este
caso, para dos eventos infinitesimalmente próximos con coordenadas
$(ct,\vec{x})$ y $(ct+cdt,\vec{x}+d\vec{x})$ respecto a $K$, las diferencias
$d\vec{x}'$ y $dt'$ asociadas a $K'$ están dadas por:
\begin{eqnarray}
d\vec{x}'  &  =&d\vec{x}+\frac{\left(  \gamma-1\right)  }{\beta^2}\left(  \vec{\beta}\cdot d\vec{x}\right)  \vec{\beta}-\gamma
c\,dt\vec{\beta},\\
cdt'  &  =&\gamma\left( cdt-\vec{\beta}\cdot d\vec{x}\right)  .
\end{eqnarray}
En el caso en que estos incrementos sean aquellos correspondientes al movimiento
de una partícula, velocidad es dada por
$\vec{v}:={d\vec{x}}/{dt}$ respecto a $K$ y
$\vec{v}':={d\vec{x}'}/{dt'}$ respecto a $K'$. Por lo tanto, encontramos que
\begin{equation}
\frac{1}{c}\vec{v}'=\frac{1}{c}\frac{d\vec{x}'}{dt'}=\frac{\frac{d\vec{x}}{dt}
+\frac{\left(  \gamma-1\right)  }{\beta^2}\left(  \vec{\beta}\cdot
\frac{d\vec{x}}{dt}\right)  \vec{\beta}-\gamma c\vec{\beta}}{\gamma\left(
c-\vec{\beta}\cdot \frac{d\vec{x}}{dt}\right) },
\end{equation}
es decir,
\begin{equation}
\boxed{
\vec{v}'=\frac{\vec{v}+\frac{\left(  \gamma-1\right)  }{\beta^2}\left(
\vec{\beta}\cdot \vec{v}\right)  \vec{\beta}-\gamma c\vec{\beta}}{\gamma\left(
1-\vec{\beta}\cdot \frac{\vec{v}}{c}\right) }.
} \label{transv}
\end{equation}
Equivalentemente,
\begin{equation}
\vec{v}'_\parallel=\frac{\vec{v}_\parallel- c\vec{\beta}}{1-\vec{\beta}\cdot\frac{\vec{v}_\parallel}{c} }, \qquad 
\vec{v}'_\perp=\frac{\vec{v}_\perp}{\gamma\left(
1-\vec{\beta}\cdot \frac{\vec{v}_\parallel}{c}\right) }
\end{equation}
donde hemos definido la velocidad paralela y perpendicular (respecto a $\vec\beta$), $\vec{v}_\parallel:=(\vec{v}\cdot \hat\beta)\hat\beta$,  $\vec{v}_\perp:=\vec{v}-\vec{v}_\parallel$.

\subsection{Incompatibilidad de las definiciones newtonianas de energía y momentum lineal con el Principio de Relatividad}

En mecánica no-relativista se define el momentum lineal de una partícula como
$\vec{p}:=m\vec{v}$, donde $\vec{v}$ es la velocidad de la partícula (que
depende del SRI), y $m$ es la masa de la partícula (que \textit{no depende} del SRI, es un escalar). La característica principal del momentum es su propiedad de conservación en un sistema aislado (por ejemplo, en un choque de partículas). Además, la \textbf{ley de conservación del momentum lineal} es válida (para un sistema aislado de partículas) \textit{en todo SRI}. Es simple verificar, usando la ley newtoniana (o``Galileana'') de  transformación de la velocidad de un cuerpo entre SRI's, que si el momentum de un sistema de partículas es conservado en un SRI entonces será conservado en todo SRI. 

Sin embargo, usando las leyes \textit{relativistas} de transformación de la velocidad, ec. \eqref{transv}, y suponiendo que el momentum sigue siendo $\vec{p}:=m\vec{v}$ y que $m$ es un escalar, entonces puede comprobarse que la ley de conservación del momentum de un sistema de partículas \textit{no será válida (``simultáneamente'') en todo SRI} (es decir, no respetará el Principio de Relatividad). Por ejemplo en un choque de dos partículas, si en un SRI particular $K$ se verifica que $m_1\vec{v}_{1\rm i}+m_2\vec{v}_{2\rm i} =m_1\vec{v}_{1\rm f}+m_2\vec{v}_{2\rm f}$, entonces, \eqref{transv} implican en general que, $m_1\vec{v}'_{1\rm i}+m_2\vec{v}'_{2\rm i} \neq m_1\vec{v}'_{1\rm f}+m_2\vec{v}'_{2\rm f}$ en otros SRI's $K'$.

Considere el siguiente caso simple de una colisión de partículas idénticas, todas de masa $m$.
Las velocidades iniciales y finales de las partículas $1$ y $2$ están dadas, respecto al SRI $K$ (del centro de masa), por
\begin{equation}
 \vec{v}_{1\rm i}=v \hat{x}, \qquad \vec{v}_{2\rm i}=-v \hat{x},
\end{equation}
\begin{equation}
 \vec{v}_{1\rm f}=v \hat{y}, \qquad \vec{v}_{2\rm f}=-v \hat{y}.
\end{equation}
Ahora realizamos un boost con velocidad relativa $c\vec\beta=v\hat{x}$, hasta el SRI $K'$ comóvil con la partícula 1 en su movimiento inicial (SRI ``de laboratorio''). Utilizando la ley de transformación relativista de velocidades (\ref{transv}), obtenemos las velocidades respecto al nuevo SRI:
\begin{equation}
 \vec{v}'_{1\rm i}=\vec{0}, \qquad \vec{v}'_{2\rm i}=-\frac{2v \hat{x}}{1+v^2/c^2},
\end{equation}
\begin{equation}
 \vec{v}'_{1\rm f}=-v\hat{x}+\frac{v}{\gamma} \hat{y}, \qquad \vec{v}'_{2\rm f}=-v\hat{x}-\frac{v}{\gamma} \hat{y}.
\end{equation}
Si el momentum de cada partícula es de la forma $\vec{p}=m\vec{v}$ entonces, el momentum total en el SRI de centro de masa es dado por
\begin{equation}
 \vec{p}_{\rm i}=\vec{p}_{1\rm i}+\vec{p}_{2\rm i}=\vec{0},
\end{equation}
\begin{equation}
 \vec{p}_{\rm f}=\vec{p}_{1\rm f}+\vec{p}_{2\rm f}=\vec{0},
\end{equation}
es decir, se conserva el momentum lineal total. Sin embargo, en el SRI de laboratorio tendremos que
\begin{equation}
 \vec{p}'_{\rm i}=\vec{p}'_{1\rm i}+\vec{p}'_{2\rm i}=-\frac{2mv\hat{x}}{1+v^2/c^2},
\end{equation}
mientras que
\begin{equation}
 \vec{p}'_{\rm f}=\vec{p}'_{1\rm f}+\vec{p}'_{2\rm f}=-2mv\hat{x}.
\end{equation}

Similarmente, puede verificarse que lo mismo ocurre con la energía. 
Si suponemos que la energía (cinética) de una partícula de masa $m$ y velocidad $\vec{v}$ es dada por la usual expresión norelativista, entonces en el SRI del centro de masa la energía se conserva en el proceso analizado:
\begin{equation}
 E_{\rm i}=\frac{1}{2}m\vec{v}_{1\rm i}^2+\frac{1}{2}m\vec{v}_{2\rm i}^2=mv^2,
\end{equation}
\begin{equation}
 E_{\rm f}=\frac{1}{2}m\vec{v}_{1\rm f}^2+\frac{1}{2}m\vec{v}_{2\rm f}^2=mv^2.
\end{equation}
Sin embargo, en el SRI de laboratorio
\begin{equation}
 E'_{\rm i}=\frac{1}{2}m\vec{v}'_{1\rm i}{}^2+\frac{1}{2}m\vec{v}'_{2\rm i}{}^2=\frac{2mv^2}{(1+v^2/c^2)^2},
\end{equation}
mientras que
\begin{equation}
 E'_{\rm f}=\frac{1}{2}m\vec{v}'_{1\rm f}{}^2+\frac{1}{2}m\vec{v}'_{2\rm f}{}^2=mv^2\left(2-v^2/c^2\right).
\end{equation}
Claramente, en general (es decir, para valores arbitrarios de $v$),  $ E'_{\rm i}\neq E'_{\rm f}$. ! Este resultado se contradice con el esperado a partir del Principio de Relatividad: que si la energía se conserva en un SRI, entonces debe conservarse en todo SRI. Si esto no fuese así, entonces existirían SRI's ``privilegiados'' o ``especiales'': aquel SRI donde sí se conserva la energía en un choque. El mismo argumeto es aplicable al momentum lineal.

En resumen, \textit{en RE la definición newtoniana de la energía y el momentum de un cuerpo es incompatible con el Prinicpio de Relatividad}, es decir, con la equivalencia entre SRI's. Por esto, es necesario reemplazar las definiciones newtonianas por unas que sí respeten este principio, es decir, que satisfagan el requerimiento que si la energía es conservada en un SRI, entonces sea conservada \textit{en todo SRI}, y similarmente para el momentum.

Como veremos más adelante, en la teoría de RE se reemplazan las definiciones newtonianas de energía y momentum lineal por nuevas definiciones (en términos de la masa y velocidad del cuerpo) que sí son compatibles en el Principio de Relatividad. Las \textbf{expresiones relativistas de energía y momentum} son
\begin{equation}
E=\frac{mc^2}{\sqrt{1-\frac{v^2}{c^2}}}, \qquad \vec{p}=\frac{m\vec{v}}{\sqrt{1-\frac{v^2}{c^2}}}.
\end{equation}

Para \textit{probar} que estas nuevas definiciones realmente satisfacen las propiedades requeridas, es de mucha utilidad expresar estas cantidades en términos de vectores y tensores \textit{respecto a Transformaciones de Lorentz} (que relacionan SRI's): los llamados 4-vectores y 4-tensores (``cuadri-vectores'' y ``cuadri-tensores''). Por esto, a continuación revisaremos rápidamente la definición y propiedades básicas de estos objetos.


\section{La visión Cuadridimensional}
En la teoría newtoniana el tiempo es absoluto, de modo que el espaciotiempo (es decir, el conjunto de todos los eventos) se separa naturalmente en espacio + tiempo. Matemáticamente, esto significa que
bajo las transformaciones más generales que relacionan un SRI con otro
(Transformaciones de Galileo, Rotaciones y translaciones) tanto los intervalos de
tiempo entre eventos (infinitesimalmente próximos), con coordenadas
$(t,\vec{x})$ y $(t+dt,\vec{x}+d\vec{x})$ respecto a un SRI, como la
distancia entre ellos permanecen invariantes, es decir,
\begin{equation}
dt'=dt, \qquad d\vec{x}^2=d\vec{x}'^2.
\end{equation}
La condición de invariancia de $d\vec{x}^2$, permite asociar un concepto
absoluto de distancia: \textit{una geometría tridimensional euclideana}, donde las coordenadas de un evento respecto a un SRI son relativas (cambian bajo transformación de SRI), pero la distancia $d\vec{x}^2$ es absoluta.

Por otro lado, en la teoría especial de la relatividad, la distancia espacial $d\vec{x}^2$ no es absoluta, como tampoco lo es $dt$ (``los boosts mezclan espacio y tiempo''). Sin embargo, el así llamado \textbf{intervalo},
\begin{equation}
ds^2:=c^2dt^2-d\vec{x}^2, \label{ds}
\end{equation}
es invariante bajo transformaciones entre SRI's. En este sentido, \textit{el
intervalo es una magnitud absoluta en RE}. En analogía al caso newtoniano, esto permite introducir una geometría, interpretando $ds^2$ como una especie de ``distancia'' en el espaciotiempo 4-dimensional: el \textbf{espacio de Minkowski}\footnote{Hermann Minkowski (1864-1909): matemático alemán. Ver \url{http://es.wikipedia.org/wiki/Hermann_Minkowski}.} (1907, ver \cite{Minkowski07}).

\begin{center}
\boxed{
\parbox{12cm}{$ds^2:=c^2dt^2-d\vec{x}^2$ define la geometría (pseudo-)euclideana del espaciotiempo de Minkowski.}}
\end{center}
Por ejemplo, el boost en la dirección $x$ puede ser expresado en términos de una matriz de transformación:
\begin{equation}
 \left(\begin{array}{c} ct' \\ x' \\ y' \\z' \end{array}\right)=\left(
\begin{array}{cccc}
\gamma & -\gamma\beta & 0 &0 \\
-\gamma\beta & \gamma  & 0  & 0 \\
0 & 0 & 1  & 0 \\
0 & 0 & 0  & 1
\end{array}
\right)\left(\begin{array}{c} ct \\ x \\ y \\z \end{array}\right).
\end{equation}

Ahora estamos en condiciones de definir más generalmente una Transformación de Lorentz. 

\begin{quotation}
\boxed{
\parbox{12cm}{Las \textbf{transformaciones de Lorentz} (TL's) son definidas como aquellas transformaciones lineales de las (4) coordenadas de un evento, entre 2 SRI's, y que dejan el intervalo $ds^2$ \textit{invariante}, es decir, tales que $c^2dt^2-d\vec{x}^2=c^2dt'^2-d\vec{x}'^2$.}}
\end{quotation}


Con esta definición, las TL's incluyen los boost's (TL's simples) en una dirección arbitraria, las rotaciones, reflexiones espaciales y temporales (estos tres últimos tipos de transformación dejan $dt^2$ y $d\vec{x}^2$ separadamente invariantes), \textit{y sus composiciones}. Matemáticamente, el conjunto de todas las TL's forma un \textbf{grupo}: el grupo de Lorentz.

Recuerde que la interpretación física de $ds$ es que, para eventos con separación tipo tiempo ($ds^2>0$) la cantidad $d\tau=ds/c$ es el \textit{tiempo propio entre dos eventos} (infinitesimalmente próximos), es decir el tiempo registrado por un observador comóvil con los eventos. Esto puede verificarse en el SRI en el que los dos eventos aparezcan en el mismo punto del espacio, es decir, $d\vec{x}^2=0$, ya que en este caso $ds=c\,dt$.
\subsection{4-vectores y 4-tensores}

Denotamos las coordenadas de un evento en el espaciotiempo,
respecto de un SRI $K$, colectivamente por $x^\mu$, con $\mu=0,1,2,3$, de modo que $x^0:=ct$, $x^1:=x$, $x^2:=y$, $x^3:=z$. En general, los índices griegos $\mu,\nu,\lambda,\cdots$ variarán de 0 a 3, mientras que los latinos
$i,j,k,\cdots$ lo harán de 1 a 3. Así, por ejemplo, denotaremos
$x^\mu=(x^0,x^i)$ o, equivalentemente, $x^\mu=(x^0,\vec{x})$.

Bajo una TL, las coordenadas de un evento cambiarán, por ejemplo de $x^\mu$ a
$x'^\mu$ cuando cambiemos de un SRI $K$ a otro $K'$. La TL más general, que
incluye boosts, ver (\ref{boost1}), y rotaciones, es lineal en las coordenadas
espaciotemporales y por tanto puede escribirse en la forma
\begin{equation}\label{xpLx}
x'^\mu=\Lambda^\mu_{\ \nu}x^\nu ,
\end{equation}
donde $\Lambda^\mu_{\ \nu}$ son, para cada transformación, 16 constantes. Además, las componentes $\Lambda^\mu_{\ \nu}$ deben ser tales que el intervalo permanezca invariante: $ds'^2=ds^2$.
En (\ref{xpLx}) hemos usado la convención de suma de Einstein, de suma de
índices repetidos, de modo que $\Lambda^\mu_{\
\nu}x^\nu:=\Lambda^\mu_{\ 0}x^0+ \Lambda^\mu_{\ 1}x^1+ \Lambda^\mu_{\ 2}x^2+\Lambda^\mu_{\ 3}x^3$.

Por ejemplo, la matriz $\Lambda^\mu_{\ \nu}$ correspondiente al boost (\ref{bg1})-(\ref{bg2}) es dada por:
\begin{equation}
\Lambda^\mu_{\ \nu}=\left(
\begin{array}
[c]{cccc}
\gamma & -\gamma\beta^x & -\gamma\beta^y & -\gamma\beta^z\\
-\gamma\beta^x & 1+\frac{(\beta^x)^2}{\beta^2}\left(\gamma-1\right)  &
\frac{\beta^x\beta^y}{\beta^2}\left(\gamma-1\right)  & \frac{\beta^x\beta^z}{\beta^2}\left(\gamma-1\right) \\
-\gamma\beta^y & \frac{\beta^y\beta^x}{\beta^2}\left(\gamma-1\right)
& 1+\frac{(\beta^y)^2}{\beta^2}\left(\gamma-1\right)  & \frac{\beta
^y\beta^z}{\beta^2}\left(  \gamma-1\right) \\
-\gamma\beta^z & \frac{\beta^z\beta^x}{\beta^2}\left(\gamma-1\right)
& \frac{\beta^z\beta^y}{\beta^2}\left(\gamma-1\right)  & 1+\frac{(\beta^z)^2}{\beta^2}\left(\gamma-1\right)
\end{array}
\right), \label{boostgen}
\end{equation}
con $\vec\beta=(\beta^x,\beta^y,\beta^z)$ y $\beta^2=(\beta^x)^2+(\beta^y)^2+(\beta^z)^2$.

Por otro lado, una \textbf{rotación general} es de la forma
\begin{equation}
\Lambda^\mu_{\ \nu}=\left(
\begin{array}[c]{c|c}
1 & 0_3 \\
\hline
0_3 & \mathbf{R}
\end{array}
\right), \label{rotgen}
\end{equation}
donde $\mathbf{R}$ es una matriz \textit{ortogonal} de $3\times 3$, es decir, que satisface $\mathbf{R}\mathbf{R}^\top=\mathbf{1}_{3\times 3}$. Esta transformación preserva $dt^2$ y $d\vec{x}^2$ \textit{por separado}.

Tal como en mecánica newtoniana es útil expresar las leyes físicas usando
vectores y tensores respecto a rotaciones, en RE es conveniente (pero no
obligatorio!) usar vectores, tensores, etc., definidos \textit{con respecto  a TL's}.

\subsubsection{Escalares (\textit{invariantes})}
Un escalar es una cantidad que \textit{no cambia su valor bajo TL's}. Si $\phi(P)$ es una cantidad asociada al evento $P$ respecto al SRI $K$, y $\phi'(P)$ es la misma cantidad, asociada al mismo evento, pero con respecto a otro SRI arbitrario $K'$, entonces $\phi(P)$ es un escalar si y sólo si
\begin{equation}
\phi'(P)=\phi(P).
\end{equation}
Ejemplos de cantidades escalares bajo TL's: las masas y las cargas de
las partículas, la rapidez de la luz, el intervalo.

\subsubsection{4-vector \textit{contravariante}}
Se dice que conjunto de 4 cantidades, $A^\mu$ ($\mu=0,1,2,3$), definidas en cada SRI (y denotadas, por convención, usando un \textbf{superíndice}), son las componentes de un 4-vector contravariante si sus valores respecto a dos SRI's relacionados por una TL arbitraria, están relacionadas tal como las coordenadas espaciotemporales, es decir, por
\begin{equation}
A'^\mu=\Lambda^\mu_{\ \nu} A^\nu .
\end{equation}
Ejemplos: 4-posición, 4-velocidad, 4-momentum.

\subsubsection{4-vector \textit{covariante}}
Se dice que un conjunto de 4 cantidades, $A_\mu$ ($\mu=0,1,2,3$), definidas en cada SRI (y denotadas, por convención, usando un \textbf{subíndice}), son las componentes de un 4-vector covariante si sus valores respecto a dos SRI's relacionados por una TL arbitraria, están relacionadas por
\begin{equation}
A'_\mu=\left( \Lambda^{-1}\right)^\nu _{\ \mu}A_{\nu},
\end{equation}
donde $\left( \Lambda^{-1}\right)^\nu _{\ \mu}$ es la TL \textit{inversa}, definida de modo que
\begin{equation}
\Lambda^\mu_{\ \lambda}\left( \Lambda^{-1}\right)^{\lambda}_{\
\nu}=\delta^\mu_\nu ,
\end{equation}
y donde $\delta^\mu_\nu$ es la delta de Kronecker, definida en todo SRI por 
$\delta^0_0=\delta^1_1=\delta^2_2=\delta^3_3=1$, y $\delta^\mu_\nu=0$ si
$\mu\neq\nu$. 

Ejemplo de vector covariante: el 4-\textit{gradiente} de un campo escalar, $A_\mu:={\partial\phi}/{\partial x^\mu}=\partial_\mu\phi$.

\subsubsection{Tensor de rango $(^p_q)$:}
Un conjunto de $4^{p+q}$ cantidades ($T^{\mu_1 \cdots\mu_p}_{\ \ \ \ \ \ \ \
\nu_1\cdots\nu_q}$) definidas en cada SRI se consideran componentes de un (4-)tensor de rango $(^p_q)$, si sus valores se relacionan, bajo TL's, por
\begin{equation}
T'^{\mu_1 \cdots\mu_p}_{\ \ \ \ \ \ \ \
\nu_1\cdots\nu_q}={\Lambda^{\mu_1}}_{\lambda_1} \cdots
{\Lambda^{\mu_p}}_{\lambda_p} \left( \Lambda^{-1}\right) ^{\rho_1}_{\
\nu_1}\cdots\left( \Lambda^{-1}\right) ^{\rho_q}_{\ \nu_q} T^{\lambda_1
\cdots\lambda_p}_{\ \ \ \ \ \ \ \rho_1\cdots\rho_q}. 
\end{equation}

\subsubsection{Multiplicación de 4-tensores}
 Si $A^{\mu_1 \cdots\mu_p}_{\ \ \ \ \ \ \ \
\nu_1\cdots\nu_q}$ y $B^{\mu_1 \cdots\mu_r}_{\ \ \ \ \ \ \ \
\nu_1\cdots\nu_s}$ son (las $4^{p+q}$ y $4^{r+s}$ componentes de) dos tensores de rango $(^p_q)$ y $(^r_s)$ respectivamente, entonces el
conjunto de $4^{p+q+r+s}$ cantidades definidas por
\begin{equation}
C^{\mu_1 \cdots\mu_{p+r}}_{\ \ \ \ \ \ \ \ \ \
\nu_1\cdots\nu_{q+s}}:=A^{\mu_1 \cdots\mu_p}_{\ \ \ \ \ \ \ \
\nu_1\cdots\nu_q}\times B^{\mu_{p+1} \cdots\mu_{p+r}}_{\ \ \ \ \ \ \ \ \ \ \ \ \, \nu_{q+1}\cdots\nu_{q+s}}
\end{equation}
definen un tensor de rango $(^{p+r}_{q+s})$ bajo TL's.

\subsubsection{Contracción de 4-tensores}
 Si $A^{\mu_1 \cdots\mu_p}_{\ \ \ \ \ \ \ \
\nu_1\cdots\nu_q}$ son las $4^{p+q}$ componentes de un tensor de rango $(^p_q)$, entonces las $4^{p+q-2}$ cantidades definidas por la ``contracción''
\begin{equation}
B^{\mu_1 \cdots\mu_{n-1}\mu_{n+1}\cdots\mu_p}_{\ \ \ \ \ \ \ \ \ \ \ \ \ \ \ \ \ \ \ \ \nu_1\cdots\nu_{m-1}\nu_{m+1}\cdots\nu_q}:=A^{\mu_1
\cdots\mu_{n-1}\rho\mu_{n+1}\cdots\mu_p}_{\ \ \ \ \ \ \ \ \ \ \ \ \ \ \ \ \ \ \
\ \ \nu_1\cdots\nu_{m-1}\rho\nu_{m+1}\cdots\nu_q},
\end{equation}
son componentes de un tersor de rango $(^{p-1}_{q-1})$.

\paragraph{Observaciones:}

\begin{itemize}
\item Un escalar es un tensor de rango $(^0_0)$.
\item Un vector contravariante es un tensor de rango $(^1_0)$.
\item Un vector covariante es un tensor de rango $(^0_1)$.
\item La delta de Kronecker define un \textbf{tensor invariante} de rango $(^1_1)$ bajo TL's, es decir, que sus componentes tienen el mismo valor en todo SRI.
\item Podemos definir además el \textbf{símbolo de Levi-Civita contravariante} $\hat\epsilon^{\mu\nu\lambda\rho}$ como el \textbf{(pseudo-)tensor invariante} bajo TL's que es totalmente antisimétrico y que satisface $\hat\epsilon^{0123}:=1$.
\item Análogamente, podemos definir el \textbf{símbolo de Levi-Civita
covariante} $\epsilon_{\mu\nu\lambda\rho}$ como el \textbf{(pseudo-)tensor invariante} bajo TL's (de determinante 1) que es totalmente antisimétrico y
que satisface $\epsilon_{0123}:=1$.

\item La utilidad de los tensores, reside en el hecho que las relaciones que
involucran tensores del mismo grado, son ecuaciones 
\textbf{invariantes de forma}\footnote{También llamadas, abusando un poco del lenguaje, ecuaciones \textbf{covariantes}.} bajo TL's, esto es, si
\begin{equation}
T^{\mu_1 \mu_2 ... \mu_r}=0,
\end{equation}
entonces en todo otro SRI 
\begin{equation}
T'^{\nu_1 \nu_2 ... \nu_r}=0.
\end{equation}
Como las TL's expresan matemáticamente el cambio desde un SRI a otro, entonces \textit{una forma de garantizar automáticamente que una cierta ley física respeta la equivalencia entre SRI's, es decir, el Principio de Relatividad, es expresando dicha ley en función de cantidades que sean tensores bajo TL's}. Esto es análogo al caso newtoniano, donde el hecho de escribir las leyes de la Física en términos de vectores y tensores tridimensionales (por ejemplo,
$\vec{F}=m\vec{a}$, o bien $D_i=\varepsilon_{ij}E_j$), que transforman homogéneamente (como vectores y tensores) bajo \textit{rotaciones}, asegura que esta ley tendrá \textit{la misma forma} en cualquier sistema de coordenadas cartesiano, independiente de la orientación espacial de sus ejes. En otras palabras, el uso de vectores y tensores tridimensionales para expresar las leyes físicas resulta conveniente puesto que incorpora naturalmente la equivalencia de sistemas de coordenadas cartesianos respecto a rotaciones. Del mismo modo, el expresar una ley física en términos de 4-tensores bajo TL's asegura que esta ley respeta el Principio de Relatividad de RE.

\end{itemize}

\subsubsection{Identidades}

\paragraph{Antisimetrización:}

\begin{equation}
T_{[\mu\nu]}:=\frac{1}{2}(T_{\mu\nu}-T_{\nu\mu}),
\end{equation}
\begin{equation}
T_{[\mu\nu\lambda]}:=\frac{1}{3}(T_{\mu[\nu\lambda]}+T_{\nu[\lambda\mu]}+T_{
\lambda[\mu\nu]}),
\end{equation}
\begin{equation}
T_{[\mu\nu\lambda\rho]}:=\frac{1}{4}(T_{\mu[\nu\lambda\rho]}-T_{\nu[
\lambda\rho\mu]}+T_{\lambda[\rho\mu\nu]}-T_{\rho[\mu\nu\lambda]})\equiv T_{[0123]}\,\epsilon_{\mu\nu\lambda\rho}.
\end{equation}

\paragraph{Propiedades de los símbolos de Levi-Civita:}
\begin{eqnarray}
\hat\epsilon^{\mu\nu\lambda\rho}\, \epsilon_{\alpha\beta\gamma\delta}
&\equiv& 4!\,\delta^\mu_{[\alpha}\delta ^\nu_\beta \delta^\lambda_\gamma
\delta^\rho_{\delta]}, \\
\hat\epsilon^{\mu\nu\lambda\rho}\, \epsilon_{\alpha\beta\gamma\rho}
&\equiv& 3!\,\delta^\mu_{[\alpha}\delta ^\nu_\beta \delta^\lambda_{\gamma]}, \\
\hat\epsilon^{\mu\nu\lambda\rho}\, \epsilon_{\alpha\beta\lambda\rho}
&\equiv& 2! \left(\delta^\mu_\alpha\delta ^\nu_\beta -  \delta^\nu_\alpha \delta^\mu_\beta \right),\\
\hat\epsilon^{\mu\nu\lambda\rho}\, \epsilon_{\alpha\nu\lambda\rho}
&\equiv& 3!\,\delta^\mu_\alpha,\\
\hat\epsilon^{\mu\nu\lambda\rho}\, \epsilon_{\mu\nu\lambda\rho}&\equiv& 4!.
\end{eqnarray}


\subsection{Métrica de Minkowski}
Podemos expresar el escalar $ds$ en términos de 4-tensores. Para esto, notamos
que el 4-desplazamiento $dx^\mu$ es un 4-vector (bajo TL's), de modo que
$dx^\mu dx^\nu$ es un tensor de rango $(^2_0)$. Podemos entonces escribir
el escalar $ds$ en términos de $dx^\mu dx^\nu$ y un nuevo \textbf{tensor
métrico} de rango $(^0_2)$, que denotaremos por $\eta_{\mu\nu}$, de modo que
\begin{equation}
ds^2:=\eta_{\mu\nu}dx^\mu dx^\nu. \label{dstens}
\end{equation}
Comparando (\ref{dstens}) con (\ref{ds}) encontramos que (en un SRI)
debemos tener que
\begin{equation}
\eta_{00}=-\eta_{11}=-\eta_{22}=-\eta_{33}=1, \qquad \eta_{\mu\nu}=0
\quad\rm{si}\quad \mu\neq\nu .
\end{equation}
Sin embargo, en RE el intervalo $ds$ no es cualquier escalar, sino que tiene el mismo valor en todo SRI \textit{cuando se lo calcula de la misma forma}, tal como en la definición (\ref{ds}). Esto significa que, en RE, el tensor métrico (o simplemente ``la métrica'') $\eta$ debe tener los mismos valores, en cada una de sus componentes, en todo SRI. En  otras palabras, $\eta$ debe ser un tensor \textit{invariante} de rango $(^0_2)$ bajo TL's. Esto significa que
\begin{equation}
\eta'_{\mu\nu}=\left( \Lambda^{-1}\right)^{\lambda}_{\ \mu}\left(
\Lambda^{-1}\right)^{\rho}_{\ \nu}\eta_{\lambda\rho}=\eta_{\mu\nu},
\end{equation}
o, equivalentemente
\begin{equation}
\boxed{\Lambda^\lambda_{\ \mu}\Lambda^\rho_{\ \nu}\eta_{\lambda\rho}=\eta_{\mu\nu}.}
\label{condLor}
\end{equation}
De hecho, es posible considerar (\ref{condLor}) como la condición general que define una TL $\Lambda$.

Teniendo el tensor métrico a nuestra disposición es posible definir el \textbf{tensor métrico inverso} $\eta^{\mu\nu}$, de rango $(^2_0)$, de modo que
\begin{equation}
\boxed{\eta^{\mu\nu}\eta_{\nu\lambda}=\delta^\mu_\lambda .}\label{definv}
\end{equation}

En notación matricial $\eta^{\mu\nu}$ corresponde a la matriz inversa a la
métrica $\eta_{\mu\nu}$. Más aún, las componentes de la métrica (de
Minkowski) inversa coinciden con aquellas de $\eta_{\mu\nu}$
\begin{equation}
\eta^{00}=-\eta^{11}=-\eta^{22}=-\eta^{33}=1, \qquad \eta^{\mu\nu}=0
\quad\rm{si}\quad \mu\neq\nu .
\end{equation}

El tensor métrico $\eta_{\mu\nu}$ permite \textit{asociar} a cada vector
\textbf{contravariante} $A^\mu$ un nuevo vector \textbf{covariante}  (abusando
un poco de la notación) $A_\mu$, definido por
\begin{equation}
A_\mu:=\eta_{\mu\nu}A^\nu.
\end{equation}
Similarmente, la métrica inversa $\eta^{\mu\nu}$ permite asociar un vector
\textit{contravariante} $A^\mu$ a cada vector \textit{covariante} $A_\mu$, por
medio de
\begin{equation}
A^\mu:=\eta^{\mu\nu}A_\nu.
\end{equation}
A partir de (\ref{definv}) es simple verificar que la secuencia de operaciones consistente en tomar un vector contravariante cualquiera, definir su vector covariante asociado, y a partir de este último calcular el vector contravariante correspondiente, retorna al vector contravariante original. Lo mismo ocurre si partimos de un vector covariante. Esta propiedad justifica (parcialmente) la convención de uso de la misma letra para un vector covariante y su vector contravariante asociado.

% \subsubsection{Notación Matricial}
%
% Usaremos la convención de que el índice superior (contravariante)
% representa las filas, y el inferior (covariante) las columnas de la matriz.
% Por ejemplo, las siguientes expresiones se muestran en notación vectorial
% y tensorial:
% \begin{align}
% x^{\alpha} &  =\Lambda_{\ \beta}^{\alpha}y^{\beta}\Leftrightarrow\left(
% \begin{array}
% [c]{c}%
% x^1\\
% x^2\\
% x^3\\
% x^4%
% \end{array}
% \right)  =\left(
% \begin{array}
% [c]{cccc}%
% \Lambda_1^1 & \Lambda_2^1 & \Lambda_3^1 & \Lambda_{4}^1\\
% \Lambda_1^2 & \Lambda_2^2 & \Lambda_3^2 & \Lambda_{4}^2\\
% \Lambda_1^3 & \Lambda_2^3 & \Lambda_3^3 & \Lambda_{4}^3\\
% \Lambda_1^4 & \Lambda_2^4 & \Lambda_3^4 & \Lambda_{4}^4%
% \end{array}
% \right)  \left(
% \begin{array}
% [c]{c}%
% y^1\\
% y^2\\
% y^3\\
% y^4%
% \end{array}
% \right)  \\
% x_{\beta} &  =y_{\alpha}\Omega_{\ \beta}^{\alpha}\Leftrightarrow\left(
% \begin{array}
% [c]{cccc}%
% x^0 & x^1 & x^2 & x^3%
% \end{array}
% \right)  =\left(
% \begin{array}
% [c]{cccc}%
% y^0 & y^1 & y^2 & y^3%
% \end{array}
% \right)  \left(
% \begin{array}
% [c]{cccc}%
% \Omega_1^1 & \Omega_2^1 & \Omega_3^1 & \Omega_{4}^1\\
% \Omega_1^2 & \Omega_2^2 & \Omega_3^2 & \Omega_{4}^2\\
% \Omega_1^3 & \Omega_2^3 & \Omega_3^3 & \Omega_{4}^3\\
% \Omega_1^4 & \Omega_2^4 & \Omega_3^4 & \Omega_{4}^4%
% \end{array}
% \right)  \\
% \Lambda_{\ \beta}^{\alpha}\Omega_{\ \gamma}^{\beta} &  =\left[  \hat{\Lambda
% }\times\hat{\Omega}\right]  _{\ \gamma}^{\alpha}%
% \end{align}


\subsection{Transformaciones de Lorentz infinitesimales*}
Considere una TL \textit{infinitesimal}, es decir, muy cercana a la identidad:
\begin{equation}
\Lambda^\mu_{\ \nu}=\delta^\mu_\nu+M^\mu_{\ \nu}, \qquad M^\mu_{\ \nu}\ll 1.
\end{equation}
La condición de Lorentz (\ref{condLor}) implica entonces que la matriz
(infinitesimal) $M^\mu_{\ \nu}$ debe satisfacer
\begin{equation}
\eta_{\mu\lambda}M^\lambda_{\ \nu}+\eta_{\nu\lambda}M^\lambda_{\ \mu}=0.
\label{cond2}
\end{equation}
Si definimos $M_{\mu\nu}:=\eta_{\mu\lambda}M^\lambda_{\  \nu}$, entonces la
condición (\ref{cond2}) nos dice que $M_{\mu\nu}$ debe ser antisimétrico, es
decir $M_{\mu\nu}=-M_{\nu\mu}$. Por tanto, cualquier matriz antisimétrica
$M_{\mu\nu}$ definirá una TL infinitesimal. En términos de esta matriz
antisimétrica, la TL queda expresada como
\begin{equation}
\Lambda^\mu_{\ \nu}=\delta^\mu_\nu+\eta^{\mu\lambda}M_{\lambda\nu}.
\end{equation}
Ahora, es posible escribir la matriz antisimétrica más general posible en la
forma
\begin{equation}
M_{\mu\nu}=\left( \begin{array}{cccc}
0 & \varepsilon^4 & \varepsilon^{5} & \varepsilon^{6} \\
-\varepsilon^4 & 0 & -\varepsilon^3 & \varepsilon^2 \\
-\varepsilon^{5} & \varepsilon^3 & 0 & -\varepsilon^1 \\
-\varepsilon^{6} & -\varepsilon^2 & \varepsilon^1 & 0
\end{array}
\right) ,
\end{equation}
donde $\varepsilon^\alpha$, $\alpha=1,\dots,6$ son parámetros arbitrarios. Con
esto, la matriz $M^\mu_{\ \nu}$ más general puede escribirse como
\begin{equation}
M^\mu_{\ \nu}=\left( \begin{array}{cccc}
0 & \varepsilon ^4 & \varepsilon ^{5} & \varepsilon ^{6} \\
\varepsilon ^4 & 0 & -\varepsilon ^3 & \varepsilon^2 \\
\varepsilon ^{5} & \varepsilon ^3 & 0 & -\varepsilon ^1 \\
\varepsilon ^{6} & -\varepsilon^2 & \varepsilon ^1 & 0
\end{array}
\right) .\label{Marab}
\end{equation}
De esto modo, hemos encontrado que una TL infinitesimal arbitraria queda
determinada por 6 parámetros independientes. Estos parámetros
(infinitesimales) corresponden físicamente a las componentes de la velocidad
relativa del nuevo SRI respecto al primero (3 parámetros), a la dirección del
eje de rotación entre SRI's (2 parámetros) y al ángulos de rotación entre
SRI's en torno a este eje (1 parámetro).

Podemos en general descomponer la matriz (\ref{Marab}) como una combinación
lineal de los siguientes \textbf{generadores}
\begin{equation}
\left(J_1\right)^\mu_{\ \nu}     =\left(
\begin{array}
[c]{cccc}
0 & 0 & 0 & 0\\
0 & 0 & 0 & 0\\
0 & 0 & 0 & -i\\
0 & 0 & +i & 0
\end{array}
\right), \qquad
\left(J_2\right)^\mu_{\ \nu}=\left(\begin{array}
[c]{cccc}
0 & 0 & 0 & 0\\
0 & 0 & 0 & +i\\
0 & 0 & 0 & 0\\
0 & -i & 0 & 0
\end{array}
\right),
\end{equation}
\begin{equation}
\left(J_3\right)^\mu_{\ \nu}=\left(
\begin{array}
[c]{cccc}
0 & 0 & 0 & 0\\
0 & 0 & -i & 0\\
0 & +i & 0 & 0\\
0 & 0 & 0 & 0
\end{array}
\right), \qquad
\left(K_1\right)^\mu_{\ \nu} =\left(
\begin{array}
[c]{rrrr}
0 & +i & 0 & 0\\
+i & 0 & 0 & 0\\
0 & 0 & 0 & 0\\
0 & 0 & 0 & 0
\end{array}
\right) ,
\end{equation}
\begin{equation}
\left(K_2\right)^\mu_{\ \nu}=\left(
\begin{array}
[c]{rrrr}
0 & 0 & i & 0\\
0 & 0 & 0 & 0\\
i & 0 & 0 & 0\\
0 & 0 & 0 & 0
\end{array}
\right), \qquad
\left(K_3\right)^\mu_{\ \nu}=\left(
\begin{array}
[c]{rrrr}
0 & 0 & 0 & i\\
0 & 0 & 0 & 0\\
0 & 0 & 0 & 0\\
i & 0 & 0 & 0
\end{array}
\right),
\end{equation}
de modo que
\begin{equation}
M^\mu_{\ \nu}= -i\sum_{j=1}^3\varepsilon^j\left(J_j\right)^\mu_{\
\nu}-i\sum_{j=1}^3\zeta_j\left(K_j\right)^\mu_{\ \nu}, \label{M}
\end{equation}
donde hemos abreviado $\zeta_j:=\varepsilon^{j+3}$. Note que los factores
imaginarios han sido introducidos sólo por convención (las TL's son reales).


\subsection{Transformaciones de Lorentz finitas*}
Es posible expresar una TL finita (no infinitesimal) como una composición de
TL's infinitesimalres. Esto permite escribir una TL (de determinante 1) como
\begin{equation}
\Lambda^\mu_{\ \nu}=\left( e^M\right)^\mu_{\ \nu}=\delta^\mu_\nu+M^\mu_{\
\nu}+\frac{1}{2!}M^\mu_{\ \lambda}M^\lambda_{\ \nu}+\frac{1}{3!}M^\mu_{\
\lambda}M^\lambda_{\ \rho}M^\rho_{\ \mu}+\cdots ,
\end{equation}
donde la matriz $M^\mu_{\ \nu}$ es de la forma (\ref{M}), pero con parámetros
$\varepsilon^j$ no (necesariamente) infinitesimales.

\subsection{Caso de Boost General*}

Para el caso en que $\varepsilon_j=0$, $j=1,2,3$, tenemos (en notación
matricial):
\begin{equation}
\Lambda=\sum_{n=0}^{\infty}\frac{M^{n}}{n!}=\exp(M)=\exp\left(
-i\sum_{j=1}^3\zeta_j K_j\right),
\end{equation}
con $M$, dado por
\begin{equation}
M=\left(
\begin{array}
[c]{cccc}
0 & -\zeta_1 & -\zeta_2 & -\zeta_3\\
-\zeta_1 & 0 & 0 & 0\\
-\zeta_2 & 0 & 0 & 0\\
-\zeta_3 & 0 & 0 & 0
\end{array}
\right).
\end{equation}
Entonces tenemos que
\begin{equation}
M^2=\left(
\begin{array}
[c]{cccc}
\zeta_1^2+\zeta_2^2+\zeta_3^2 & 0 & 0 & 0\\
0 & \zeta_1^2 & \zeta_1\zeta_2 & \zeta_1\zeta_3\\
0 & \zeta_2\zeta_1 & \zeta_2^2 & \zeta_2\zeta_3\\
0 & \zeta_3\zeta_1 & \zeta_3\zeta_2 & \zeta_3^2
\end{array}
\right).
\end{equation}
Si llamamos
\begin{equation}
\zeta^2:=\zeta_1^2+\zeta_2^2+\zeta_3^2
\end{equation}
entonces,
\begin{equation}
M^2=\zeta^2\left(
\begin{array}
[c]{cccc}
1 & 0 & 0 & 0\\
0 & \frac{\zeta_1^2}{\zeta^2} & \frac{\zeta_1\zeta_2}{\zeta^2} &
\frac{\zeta_1\zeta_3}{\zeta^2}\\
0 & \frac{\zeta_2\zeta_1}{\zeta^2} & \frac{\zeta_2^2}{\zeta^2} &
\frac{\zeta_2\zeta_3}{\zeta^2}\\
0 & \frac{\zeta_3\zeta_1}{\zeta^2} & \frac{\zeta_3\zeta_2}{\zeta
^2} & \frac{\zeta_3^2}{\zeta^2}
\end{array}
\right).
\end{equation}
Además, se verifica directamente que
\begin{equation}
M^3=\zeta^2\left(
\begin{array}
[c]{cccc}
0 & -\zeta_1 & -\zeta_2 & -\zeta_3\\
-\zeta_1 & 0 & 0 & 0\\
-\zeta_2 & 0 & 0 & 0\\
-\zeta_3 & 0 & 0 & 0
\end{array}
\right)=\zeta^2 M .
\end{equation}
Así sucesivamente, encontramos que
\begin{align}
M^4  &  =\zeta^2 M\times M=\zeta^2 M^2 ,\\
M^{5}  &  =\zeta^2 M^3=\zeta^4 M ,\\
M^{6}  &  =\zeta^4 M^2 ,\\
M^{7}  &  =\zeta^4 M^3=\zeta^{6}M
\end{align}
o, resumiendo,
\begin{equation}
M^{2n+1}   =\zeta^{2n} M, \qquad M^{2n}    =\zeta^{2n-2} M^2.
\end{equation}
Con estas relaciones, podemos calcular $\Lambda$ de la forma siguiente:
\begin{eqnarray}
\Lambda &=& I+\sum_{m=0}^{\infty}\frac{M^{2m+1}}{\left(  2m+1\right)  !}
+\sum_{m=1}^{\infty}\frac{M^{2m}}{\left(  2m\right)  !}\\
&=& I+\sum_{m=0}^{\infty}\frac{\zeta^{2m}M}{\left(  2m+1\right)  !}+\sum
_{m=1}^{\infty}\frac{\zeta^{2m-2}M^2}{\left(  2m\right)  !}\\
&=& I+M\sum_{m=0}^{\infty}\frac{\zeta^{2m}}{\left(  2m+1\right)  !}+M^2
\sum_{m=1}^{\infty}\frac{\zeta^{2m-2}}{\left(  2m\right)  !}\\
&=& I+\frac{1}{\zeta}M\sum_{m=0}^{\infty}\frac{\zeta^{2m+1}}{\left(
2m+1\right)  !}+\frac{1}{\zeta^2}M^2\sum_{m=1}^{\infty}\frac{\zeta^{2m}}{\left(
2m\right)  !}\\
&=& I+\frac{1}{\zeta}M\sum_{m=0}^{\infty}\frac{\zeta^{2m+1}}{\left(
2m+1\right)  !}+\frac{1}{\zeta^2}M^2\left(
\sum_{m=0}^{\infty}\frac{\zeta^{2m}}{\left(  2m\right)  !}-1\right).
\end{eqnarray}
Las series en la expresión anterior corresponden a las funciones $\cosh$ y
$\sinh$, ya que
\begin{equation}
\cosh x =\sum_{n=0}^{\infty}\frac{x^{2n}}{\left(  2n\right)  !}, \qquad
\sinh x =\sum_{n=0}^{\infty}\frac{x^{2n+1}}{\left(  2n+1\right)  !}.
\end{equation}
De esta forma encontramos que
\begin{equation}
\Lambda
=I+M\frac{1}{\zeta}\sinh\zeta+M^2\frac{1}{\zeta^2}\left(\cosh\zeta-1\right).
\end{equation}
Reemplazando, obtenemos:
\begin{equation}
\Lambda=\left(
\begin{array}
[c]{cccc}
\cosh\zeta & \frac{-\zeta_1}{\zeta}\sinh\zeta & \frac
{-\zeta_2}{\zeta}\sinh\zeta & \frac{-\zeta_3}{\zeta
}\sinh\zeta\\
\frac{-\zeta_1}{\zeta}\sinh\zeta & 1+\frac{\zeta_1^2}{\zeta^2}\left(  \cosh\zeta-1\right)  & \frac{\zeta_1\zeta_2}{\zeta
^2}\left(  \cosh\zeta-1\right)  & \frac{\zeta_1\zeta_3}{\zeta^2}\left(  \cosh\zeta-1\right) \\
\frac{-\zeta_2}{\zeta}\sinh\zeta & \frac{\zeta_2\zeta_1}{\zeta^2}\left(  \cosh\zeta-1\right)  & 1+\frac{\zeta_2^2}{\zeta^2}\left(  \cosh\zeta-1\right)  & \frac{\zeta_2\zeta_3}{\zeta^2}\left(
\cosh\zeta-1\right) \\
\frac{-\zeta_3}{\zeta}\sinh\zeta & \frac{\zeta_3\zeta_1}{\zeta^2}\left(  \cosh\zeta-1\right)  & \frac{\zeta_3\zeta_2}{\zeta
^2}\left(  \cosh\zeta-1\right)  & 1+\frac{\zeta_3^2}{\zeta^2}\left(
\cosh\zeta-1\right) \label{genboost}
\end{array}
\right).
\end{equation}
Finalmente, definiendo
\begin{equation}
\beta^i :=\tanh\zeta^i,
\end{equation}
de modo que (identidades)
\begin{equation}
\sinh\zeta^i =\gamma\beta^i, \qquad \beta   =\tanh\zeta, \qquad
\cosh\zeta  =\gamma, \qquad \sinh\zeta  =\gamma\beta, \label{idb1}
\end{equation}
y además
\begin{equation}
\frac{\zeta_i}{\zeta}=\frac{\beta^i}{\beta}. \label{idb2}
\end{equation}
Reemplazando (\ref{idb1}) y (\ref{idb2}) en (\ref{genboost}) obtenemos el boost general (\ref{boostgen}).


\section{Mecánica Relativista.}
\subsection{4-velocidad}
Definiremos la \textbf{4-velocidad} como una cantidad que sea un vector bajo
TL's y que está \textit{relacionada} con la usual velocidad. Recuerde que la velocidad de una partícula \textit{no forma} directamente un 4-vector bajo TL's, ver (\ref{transv}).
Para esto, usaremos el intervalo de tiempo propio $d\tau={ds}/{c}=\gamma^{-1}dt$ sobre la trayectoria de una partícula, que es un escalar bajo TL's, \textit{para parametrizar la línea de mundo de la misma partícula}. En lugar de usar $\vec{x}=\vec{x}(t)$, usaremos $x^\mu=x^\mu(\tau)$, y definimos la 4-velocidad (instantánea) por
\begin{equation}\marginnote{4-velocidad}
\boxed{u^\mu :=\frac{dx^\mu }{d\tau}. \label{def4vel}}
\end{equation}
Debido a las propiedades de transformación de las coordenadas $x^\mu$ y del
tiempo propio $d\tau$, es directo verificar que la 4-velocidad $u^\mu$ es un
4-vector (contravariante) bajo TL's. Además, la definición
(\ref{def4vel}) implica que
\begin{equation}
u^0=\frac{dx^0}{d\tau}=c\frac{dt}{d\tau}=c\gamma,
\end{equation}
\begin{equation}
\vec{u}=\frac{d\vec{x}}{d\tau}=\frac{d\vec{x}}{dt}\frac{dt}{d\tau}=
\vec{v}\gamma.
\end{equation}
Por lo tanto,
\begin{equation}
\boxed{u^\mu =(\gamma c, \gamma\vec{v}). \label{comp4vel}}
\end{equation}
Similarmente, si conocemos las componentes de la 4-velocidad, entonces las
componentes de la velocidad (tridimensional, a veces llamada ``velocidad
ordinaria'') quedan dados por
\begin{equation}
\vec{v}=c\frac{\vec{u}}{u^0}.
\end{equation}
Lo anterior muestra que \textit{la 4-velocidad $u^\mu$ contiene la misma información que la velocidad} $\vec{v}$, con la diferencia que $u^\mu$ transforma como un 4-vector bajo TL's.

Además, es directo verificar que
\begin{eqnarray}
u^\mu u_\mu &=&\gamma^ 2c^2-\gamma^2 v^2\\
&=&(c^2-v^2)\gamma^2\\
&=&\frac{c^2-v^2}{1-v^2/c^2}\\
&=&c^2.
\end{eqnarray}
En resumen,
\begin{equation}
\boxed{u^\mu u_\mu \equiv c^2. \label{uuc2}}
\end{equation}
Observe que $u_\mu=\eta_{\mu\nu}u^\nu=(u^0,-u^i)=(u^0,-\vec{u})=(\gamma
c,-\gamma \vec{v})$.

\subsection{4-aceleración}
Análogamente a la 4-velocidad, definimos la \textbf{4-aceleración}
$\mathsf{a}^\mu$ de una partícula, por
\begin{equation}\marginnote{4-aceleración}
\boxed{\mathsf{a}^\mu:=\frac{du^\mu}{d\tau}.}
\end{equation}
Usando la identidad (\ref{uuc2}) es directo probar que
\begin{equation}
\mathsf{a}^\mu u_\mu\equiv 0.
\end{equation}

Podemos expresar las componentes de la 4-aceleración
$\mathsf{a}^\mu$ en términos de la aceleración (``ordinaria'') $\vec{a}:=d\vec{v}/dt$:
\begin{eqnarray}
\mathsf{a}^\mu  & =&\frac{du^\mu}{d\tau}\\
& =&\frac{d}{d\tau}\left(  \gamma c,\gamma\vec{v}\right)  \\
& =&\frac{dt}{d\tau}\frac{d}{dt}\left(  \gamma c,\gamma\vec{v}\right)  \\
& =&\gamma\frac{d}{dt}\left(  \gamma c,\gamma\vec{v}\right)  ,
\end{eqnarray}
pero
\begin{align}
\frac{d}{dt}\gamma &
=\frac{d}{dt}\frac{1}{\sqrt{1-\frac{\vec{v}\cdot\vec{v}}{c^2}}}\\
& =-\frac{1}{2}\frac{(-\frac{2}{c^2})\vec{v}\cdot\vec{a}}{\left(
1-\frac{\vec{v}\cdot\vec{v}}{c^2}\right)  ^{\frac{3}{2}}}\\
& =\frac{1}{c^2}\gamma^3\vec{v}\cdot\vec{a},
\end{align}
\newline luego
\begin{align}
\mathsf{a}^\mu  & =\left(
c\gamma\frac{d}{dt}\gamma,\gamma\frac{d}{dt}\gamma\vec
{v}+\gamma^2\vec{a}\right)  \\
& =\left(  c\gamma\frac{1}{c^2}\gamma^3\vec{v}\cdot\vec{a},\gamma\frac{1}{c^2}\gamma^3(\vec{v}\cdot\vec{a})\vec{v}+\gamma^2\vec{a}\right)  \\
&
=\left(\frac{1}{c}\gamma^4\vec{v}\cdot\vec{a},\frac{1}{c^2}\gamma^4(\vec{v}
\cdot\vec{a})\vec{v}+\gamma^2\vec{a}\right)  ,
\end{align}
de donde
\begin{align}
\mathsf{a}^0  & =\frac{1}{c}\gamma^4\vec{v}\cdot\vec{a}, \label{a0} \\
\vec{\mathsf{a}}  &
=\frac{1}{c^2}\gamma^4(\vec{v}\cdot\vec{a})\vec{v}+\gamma^2\vec{a}. \label{a123}
\end{align}
De aquí podemos encontrar que
\begin{equation}
 \mathsf{a}^\mu\mathsf{a}_\mu=-\frac{\gamma^6}{c^2}(\vec{v}\cdot\vec{a})^2-\gamma^4\vec{a}^2=-\gamma^6\left[\vec{a}^2-\frac{1}{c^2}(\vec{v}\times\vec{a})^2\right]. \label{4a2}
\end{equation}
Finalmente, es posible escribir la aceleración de la partícula en función de
las componentes de la 4-aceleración:
\begin{equation}
\vec{a}=\frac{1}{\gamma^2}\left(\vec{\mathsf{a}}-\frac{\vec{v}}{c}\mathsf{a}
^0\right).
\end{equation}


\subsection{4-Momentum y Energía}

Una forma simple de motivar la (nueva) definición relativista de la energía y momentum lineal es usando una \textbf{formulación covariante}, es decir, usando vectores bajo TL's. Si el
momentum $\vec{p}$ corresponde a las componentes espaciales de un 4-vector
$p^\mu$, es decir, si existe un 4-vector $p^\mu$ tal que $p^\mu=(p^0,\vec{p})$, entonces la ley de conservación del momentum
($p^\mu_{\rm 1,i}+p^\mu_{2,\rm i}=p^\mu_{1,\rm f}+p^\mu_{2,\rm f}$) será con seguridad válida en
todo SRI.

\subsubsection{4-momentum}
Motivados por lo anterior, definimos el \textbf{4-momentum} como:
\begin{equation}
\boxed{p^\mu :=mu^\mu, \label{def4mom}}
\end{equation}
donde $m$ es un \textbf{escalar} que describe las propiedades inerciales de la
partícula. Algunos autores se refieren a $m$ como la `masa en reposo'\ de la
partícula, para distinguirla de la `masa relativista'\ $m(v)=\gamma
m={m}/{\sqrt{1-{v^2}/{c^2}}}$. Aquí, por el contrario, diremos que $m$
es simplemente \textit{la} masa (inercial) de la partícula (la única que
definiremos en RE).

Usando (\ref{comp4vel}) encontramos que las componentes del 4-momentum
$p^\mu=(p^0,\vec{p})$, están dadas por:
\begin{equation}
\boxed{p^0=m u^0=m\gamma c=\frac{mc}{\sqrt{1-\frac{v^2}{c^2}}}},
\end{equation}
\begin{equation}
\boxed{\vec{p}=m \vec{u}=m\gamma \vec{v}=\frac{m\vec{v}}{\sqrt{1-\frac{v^2}{c^2}}}.}
\end{equation}
De esta forma, encontramos que si definimos el momentum como
$\vec{p}=m\gamma\vec{v}$ (es decir, básicamente la expresión no-relativista
`corregida'\ con un factor relativista $\gamma$), entonces la ley de
conservación del momentum será válida en todo SRI.

El 4-momentum, por otro lado, incluye una componente extra $p^0$, que es
necesaria para formar un 4-vector y así asegurar la validez de la ley de
conservación del momentum en todo SRI, y que también describirá una ley de
conservación. La componente temporal $p^0=m\gamma c$ puede ser indentificada
con la \textit{energía de la partícula}, por medio de $E=p^0 c=\gamma mc^2$.
Podemos motivar esta interpretación notando que para velocidades
no-relativistas $p^0 c$ incluye la usual energá cinética (no-relativista) de
la partícula:
\begin{equation}
p^0c=m\gamma c^2=mc^2+\frac{1}{2}mv^2+O(\frac{v^4}{c^4}).
\end{equation}

En RE, por tanto, se interpreta la componente temporal $p^0$ del 4-momentum como proporcional a la energía de una partícula, $p^0=E/c$, de modo que
\begin{equation}
 \boxed{E=m\gamma c^2=\frac{mc^2}{\sqrt{1-v^2/c^2}}.} \label{emgc2}
\end{equation}
 Esta identificación tiene como consecuencia que incluso una  partícula en reposo, con
$\vec{v}=\vec{0}$, poseerá una energía no nula, no presente en mecánica
no-relativista, llamada \textbf{energía en reposo}:
\begin{equation}
\boxed{E_0=mc^2.} \label{e0mc2}
\end{equation}

Ha sido comprobado experimentalmente que esta energía en reposo, que cada
cuerpo posee por el sólo hecho de poseer masa, es ``real'', en el
sentido que \textit{puede ser transformada a otras formas de energía}, por ejemplo, en procesos nucleares. Más aún, es posible interpretar (\ref{e0mc2}) como una expresión que \textit{define} la masa de un cuerpo: la masa de un cuerpo es una medida de la energía que posee cuando está en reposo: $m=E_0/c^2$. Si esta interpretación es correcta, si aumentamos la energía de un cuerpo en reposo, aumentamos también su masa, es decir, su inercia. Como consecuencia, la masa de un litro de agua a $80^{\rm o} C$ sería un poco \textit{mayor} que la masa de un litro de agua a $20^{\rm o} C$ (¿cuánto?), la masa de un átomo de hidrógeno un poco \textit{menor} que la suma de las masas de un protón y un electrón (¿cuánto?), etc.

La relación entre la energía de un cuerpo y su masa fue considerada por primera vez por Einstein en su paper de Septiembre de 1905 \cite{Einstein05}, cuyo título se traduciría como \textit{¿Es la inercia de un cuerpo dependiente de su contenido de energía?}. En 1935 Einstein escribió una ``versión elemental de la equivalencia entre masa y energía" \cite{Einstein35}. Un test moderno de esta relación puede encontrarse en \cite{Rainville05}, donde se observan procesos donde un núcleo atómico captura un neutrón y emite un rayo $\gamma$, ya que la diferencia de masa de los estados iniciales y finales, multiplicados por $c^2$, debería ser igual a la energía del rayo $\gamma$ emitido. Las mediciones confirman la relación masa-energía con un error máximo de 0.00004\%!. Por otro lado, en \cite{Duerr08} se presenta el primer cálculo de las masas de los protones, neutrones, y otros hadrones livianos a partir de sus constituyentes, calculando a energía de cada sistema y usando $m=E/c^2$.

La interpretación de la componente temporal del 4-momentum en términos de la energía expresa el caracter conjugado de estas dos variables. Recuerde que en mecánica clásica energía y momentum lineal son variables conjugadas al tiempo y a las coordenadas espaciales, respectivamente. En particular, es conocido que si un sistema mecánico (su lagrangiano o hamiltoniano) es invariante bajo desplazamientos (``translaciones'') temporales, entonces la energía del sistema será conservada. Por otro lado, si el sistema presenta invariancia bajo translaciones (espaciales), entonces su momentum lineal total será conservado. Es entonces satisfactorio que en RE la energía y el momentum lineal de un cuerpo sean respectivamente (proporcionales a) las componentes temporales y espaciales del 4-momentum.

En RE (re-)definimos la \textbf{energía cinética} como la diferencia entre la
energía total y la energía en reposo:
\begin{eqnarray}
E_{c}&:=&E-E_0,\\
&=&mc^2(\gamma-1)\\
&=&\frac{1}{2}mv^2+\frac{3}{8}\frac{m}{c^2}v^4+\emph{O}(\frac{v^{6}}{c^{6}}).
\end{eqnarray}

En resumen, en RE el 4-momentum condensa el momentum lineal y la energía de una
partícula:
\begin{equation}
\boxed{p^\mu=(\frac{E}{c},\vec{p}) ,\label{pep}}
\end{equation}
con
\begin{equation}
\boxed{E=m\gamma c^2, \qquad \vec{p}=m\gamma \vec{v}.}
\end{equation}


Una relación interesante es aquella que suministra la velocidad de la
partícula en función de su momentum y energía:
\begin{equation}
\vec{v}=c^2\frac{\vec{p}}{E}. \label{vpe}
\end{equation}
Note que esta relación \textit{no involucra la masa} de la partícula.

\subsubsection{Energía, momentum y teorema trabajo-energía}
Podemos verificar que la identificación de la componente temporal del 4-momentum con la energía es consistente con el \textbf{teorema de trabajo-energía} en su formulación usual: \textit{El trabajo total realizado sobre un cuerpo es igual al incremento en su energía}.

El trabajo que una fuerza $\vec{F}$ realiza sobre un cuerpo, cuando éste se desplaza desde un punto a otro es dado por
\begin{equation}
W=\int\vec{F} \cdot d\vec{x},
\end{equation}
donde la fuerza $\vec{F}$ es (por definición) la derivada temporal del momentum $\vec{p}$, de modo
que
\begin{eqnarray}
W&=&\int\frac{d\vec{p}}{dt} \cdot d\vec{x}\\
&=&\int\frac{d\vec{p}}{dt} \cdot \frac{d\vec{x}}{dt}\,dt\\
&=&\int\frac{d\vec{p}}{dt} \cdot \vec{v}\,dt.
\end{eqnarray}
Evaluando el producto en el integrando encontramos que
\begin{eqnarray}
\frac{d\vec{p}}{dt} \cdot
\vec{v}&=&\frac{d}{dt}(\frac{m\vec{v}}{\sqrt{1-\frac{v^2}{c^2}}})\cdot
\vec{v}\\
&=&\frac{m\vec{v}}{(1-\frac{v^2}{c^2})^{\frac{3}{2}}}\cdot
\frac{d\vec{v}}{dt}\\
&=&\frac{d}{dt}(\frac{mc^2}{\sqrt{1-\frac{v^2}{c^2}}}) \\
&=&\frac{dE}{dt},
\end{eqnarray}
de modo que:
\begin{equation}
\boxed{W=\Delta E.}
\end{equation}

Note que, como consecuencia de lo anterior, y de la expresión relativista de la energía (\ref{emgc2}), el trabajo requerido para acelerar un cuerpo aumenta a medida que su velocidad es mayor. En particular, se requeriría \textit{infinita energía} para acelerar un cuerpo hasta alcanzar la velocidad de la luz. En otras palabras, \textbf{no es posible acelerar un cuerpo de modo que este viaje a la velocidad de la luz}.

\subsubsection{Energía en función del momentum}

Es directo encontrar a partir de (\ref{uuc2}) y (\ref{def4mom}) que $p^\mu
p_\mu=m^2c^2$. Además, usando (\ref{pep}) podemos escribir
\begin{equation}
p^\mu p_\mu =\frac{E^2}{c^2}-\vec{p}^2=m^2c^2.
\end{equation}
Despejando la energía en función del momentum, encontramos
\begin{equation}
\boxed{E=\sqrt{m^2c^4+\vec{p}^2c^2}.} \label{ep}
\end{equation}
Combinando (\ref{ep}) con (\ref{vpe}) podemos escribir la velocidad de un cuerpo en términos de su masa en reposo y su momentum
\begin{equation}
\vec{v}=\frac{\vec{p}c}{\sqrt{\vec{p}^2+m^2c^2}}. \label{vpp}
\end{equation}
De aquí confirmamos que necesariamente $v<c$, para $m\neq 0$.

El caso límite en que la rapidez sea igual a la de la luz, $v=c$, al mismo tiempo que $E\neq 0$, sólo podría ocurrir si $E=pc$ y $m=0$. En efecto, de acuerdo a (\ref{vpe}), $v=c$ implica $E=pc$ y por otro lado (\ref{ep}) requiere $m=0$ en este caso. Resumiendo, una partícula que se mueva a la velocidad de la luz puede ser descrita como caso particular de las relaciones cinemáticas de RE siempre que se suponga que $m=0$, y que satisfaga
\begin{equation}
E=pc.
\end{equation}
Sabemos que los \textbf{fotones} (los cuantos de energía electromagnética) satisfacen estas propiedades: tienen energía $E=\hbar\omega$ y momentum $\vec{p}=\hbar\vec{k}$, donde $\omega$ es la frecuencia angular y $\vec{k}$ el vector de onda de la radiación, que satisfacen la relación de dispersión (en el vació) $\omega=|\vec{k}|c$. De esta forma vemos que los fotones también pueden ser incorporados a la descripción que la teoría de RE suministra para la energía y momentum de un cuerpo, siempre que se considere su masa nula. Las expresiones relativistas de energía y momentum son en este sentido \textit{universales}, puesto que se aplican a todo tipo de partícula o cuerpo conocido.



\subsection{Ejemplos}
\begin{itemize}
\item Efecto Compton. Ver figura (\ref{fig:Compton}).
\begin{figure}[ht]
\centerline{\includegraphics[height=4cm]{fig/fig-Compton.pdf}}
\caption{Scattering por un electrón, visto desde el SRI comóvil con el electrón inicial}
\label{fig:Compton}
\end{figure}
La conservación del 4-momentum (es decir, energía y momentum), en este caso
es:
\begin{equation}
p^\mu_{\rm e,i}+p^\mu_{\rm f,i}=p^\mu_{\rm e,f}+p^\mu_{\rm f,f}.
\end{equation}
Es conveniente reordenar los términos y calcular el escalar correspondiente al
``cuadrado'' del 4-vector correspondiente, de modo que
\begin{eqnarray}
\left( p_{\rm e,i}-p_{\rm e,f}\right)^2 &=&\left( p_{\rm f,f}-p_{\rm
f,i}\right)^2 ,\\
p^2_{\rm e,i}+p^2_{\rm e,f}-2\,p^\mu_{\rm e,i}\,p_\mu^{\rm e,f} &=& p^2_{\rm
f,f}+p^2_{\rm f,i}-2\,p^\mu_{\rm f,i}\,p_\mu^ {\rm f,f},\\
m_{\rm e}^2 c^2+m_{\rm e}^2 c^2-2\left[\frac{E_{\rm e,i}}{c}\frac{E_{\rm e,f}}{c}-\vec{0}\cdot\vec{p}_{\rm e,f}\right] &=&
0+0-2\left[\frac{E_{\rm f,i}}{c}\frac{E_{\rm f,f}}{c}-\vec{p}_{\rm
f,i}\cdot\vec{p}_{\rm f,f}\right] ,\\
m_{\rm e}^2 c^2-m_{\rm e}E_{\rm e,f} &=& -\left[
\frac{\hbar\omega_{\rm i}}{c}\frac{\hbar\omega_{\rm f}}{c}-|\vec{p}_{\rm f,i}||\vec{p}_{\rm f,f}|\cos\theta\right] , \\
m_{\rm e}^2 c^2-m_{\rm e}E_{\rm e,f} &=&
-\frac{\hbar\omega_{\rm i}}{c}\frac{\hbar\omega_{\rm f}}{c}\left(1 -\cos\theta\right) .
\end{eqnarray}
Aquí hemos usado que en el SRI comóvil con el electrón inicial $p^\mu_{\rm
e,i}=({E_{\rm e,i}}/{c},\vec{0})$, y que para fotones $E_{\rm
f}=|\vec{p}_{\rm f}|c=\hbar\omega$. Ahora reescribiremos la energía $E_{\rm
e,f}$ usando la conservación de la energía:
\begin{eqnarray}
E_{\rm e,i}+E_{\rm f,i}&=&E_{\rm e,f}+E_{\rm f,f}, \\
m_{\rm e}c^2+\hbar\omega_{\rm i} &=&E_{\rm e,f}+\hbar\omega_{\rm f} .
\end{eqnarray}
Entonces, podemos escribir
\begin{eqnarray}
m_{\rm e}^2 c^2-m_{\rm e}\left(m_{\rm e}c^2+\hbar\omega_{\rm i}
-\hbar\omega_{\rm f} \right)  &=&
-\frac{\hbar\omega_{\rm i}}{c}\frac{\hbar\omega_{\rm f}}{c}\left(1 -\cos\theta\right), \\
-m_{\rm e}\hbar\omega_{\rm i} +m_{\rm e}\hbar\omega_{\rm f}   &=&
-\frac{\hbar\omega_{\rm i}}{c}\frac{\hbar\omega_{\rm f}}{c}\left(1 -\cos\theta\right) ,\\
\omega_{\rm f}\left[ m_{\rm e}+\frac{\hbar\omega_{\rm i}}{c^2}\left(1
-\cos\theta\right) \right]   &=& m_{\rm e}\omega_{\rm i} .
\end{eqnarray}
Así, encontramos que la frecuencia del fotón emitido en un ángulo $\theta$ es de la forma
\begin{equation}
\omega_{\rm f}=\frac{\omega_{\rm i}}{\left[ 1+\frac{\hbar\omega_{\rm i}}{m_{\rm
e}c^2}\left(1 -\cos\theta\right) \right] }
\end{equation}
o, equivalentemente, su longitud de onda es
\begin{eqnarray}
\lambda_{\rm f}&=&\lambda_{\rm i}\left[ 1+\frac{\hbar\omega_{\rm i}}{m_{\rm
e}c^2}\left(1 -\cos\theta\right) \right]\\
&=&\lambda_{\rm i}+\frac{2\pi\hbar}{m_{\rm e}c}\left(1 -\cos\theta\right).
\end{eqnarray}
De aquí obtenemos directamente la expresión para el \textit{cambio de longitud de onda del fotón}:
\begin{equation}
\Delta \lambda=\frac{h}{m_{\rm e}c}\left(1-\cos\theta\right)=\frac{2h}{m_{\rm e}c}\sin^2\left(\frac{\theta}{2}\right). \label{compton}
\end{equation}
 Desde el punto de vista de una teoría corpuscular de la luz, la reducción de frecuencia de la radiación emitida es natural debido a que parte de la energía del fotón inicial es transferida al electrón (recoil). Por otra parte, este resultado no puede ser entendido en el marco de la teoría ondulatoria clásica. Compton dedujo y confirmó experimentalmente (\ref{compton}) en 1922 \cite{Compton23}. El recoil del electrón fue medido un año más tarde por Wilson, usando una cámara de niebla.
La longitud de Compton del electrón, ${h}/{m_{\rm e}c}\approx 2,426\times 10^{-12}\,m$.

\item Imposibilidad de emisión de un fotón por un electrón libre. Ver figura (\ref{fig:no-emision}).
\begin{figure}[h!]
\centerline{\includegraphics[height=4cm]{fig/fig-no-emision.pdf}}
\caption{Diagrama para la (prohibida) emisión de un fotón por un electrón libre.}
\label{fig:no-emision}
\end{figure}

\begin{equation}
p^\mu_{\rm e,i}=p^\mu_{\rm e,f}+p^\mu_{\rm f}.
\end{equation}
\begin{eqnarray}
\left( p_{\rm e,i}-p_{\rm f}\right)^2 &=&\left( p_{\rm e,f}\right)^2 ,\\
p^2_{\rm e,i}+p^2_{\rm \rm f}-2\,p^\mu_{\rm e,i}\,p_{\mu, \rm f} &=& p^2_{\rm e,f} ,\\
m_{\rm e}^2 c^2+0-2\,p^\mu_{\rm e,i}\,p_{\mu, \rm f} &=& m_{\rm e}^2 c^2.
\end{eqnarray}
Por lo tanto,
\begin{equation}
p^\mu_{\rm e,i}\,p_{\mu, \rm f}=0.\label{sorry}
\end{equation}
En el sistema en reposo del electrón inicial, tenemos
\begin{equation}
p^\mu_{\rm e,i}\,p_{\mu, \rm f}=\frac{E_{\rm e,i}}{c}\frac{E_{\rm f}}{c} =m_{\rm
e}E_{\rm f},
\end{equation}
de modo que (\ref{sorry}) implica que $E_{\rm f}=0$ y por tanto $p^\mu_{\rm
f}=0$, es decir, que no hay fotón!.
\end{itemize}
