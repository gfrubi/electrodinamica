\begin{thebibliography}{99}
\bibitem{VFP} \textit{VectorFieldPlot}, programa Python escrito por \href{http://commons.wikimedia.org/wiki/User:Geek3}{Geek3}, \url{http://commons.wikimedia.org/wiki/User:Geek3/VectorFieldPlot}. Ejemplos disponibles en \url{http://commons.wikimedia.org/wiki/Category:Created_with_VectorFieldPlot}.

%------- fin refs. parte 1

\bibitem{MIT} MIT OpenCourseWare  ``Physics II: Electricity and Magnetism'', \url{https://ocw.mit.edu/courses/physics/8-02-physics-ii-electricity-and-magnetism-spring-2007/index.htm}.

\bibitem{hyper}
\url{http://hypertextbook.com/physics/waves/refraction/}

\bibitem{Larmor} Larmor, J., \href{doi.org/10.1080/14786449708621095}{\textit{On the theory of the magnetic influence on spectra; and on the radiation from moving ions}}, \textsl{Philosophical Magazine}. 5. \textbf{44} (271): 503-512.

\bibitem{MM87} A. Michelson and E. Morley, {\it On the relative motion of the earth and the luminiferous ether}, {\sl Am. Jour. Sci.} {\bf 34} (1887) 333-345.

\bibitem{CODATA00} P.J. Mohr and B.N. Taylor, {\it CODATA recommended values of the fundamental physical constants: 1998*}, {\sl Rev. Mod. Phys.} {\bf 72}
(2000) 351.

\bibitem{MHBSP03} H. M\"uller et al., {\it Modern Michelson-Morley
Experiment using Cryogenic Optical Resonators}, {\sl Phys. Rev. Lett.} {\bf 91} (2003) 020401.

\bibitem{GR00} I.S. Gradshteyn and I.M. Ryzhik, {\it Table of integral, series and products}, 6ta edición (2000), Academic Press.

\bibitem{AW01} G.B. Arfken and H.J. Weber, {\it Mathematical methods for physicists}, 5a  edición (2001), Academic Press.

%\bibitem{MagLab2012} {\it Magnetic Field Researchers Meet Hundred-Tesla Goal}. {\sl \href{http://www.magnet.fsu.edu/mediacenter/news/pressreleases/2012/100tshot.html}{Comunidado de prensa}, National High Magnetic Field Laboratory, Los Alamos National Laboratory} (22 de Marzo, 2012). Video relacionado \href{http://youtu.be/N0R8dyyXtTo}{aquí}.

%\bibitem{HZDR2011} {\it World Record: The Highest Magnetic Fields Are Created in Dresden}, {\sl \href{http://www.hzdr.de/db/Cms?pNid=99&pOid=33768}{Comunidado de prensa}, Helmholtz-Zentrum Dresden-Rossendorf} (28 de Junio, 2011). 

\bibitem{Brecord2018}\href{https://www.europapress.es/ciencia/laboratorio/noticia-nuevo-record-campo-magnetico-alcanza-idoneo-fusion-nuclear-20180918110647.html}{\textit{Nuevo récord de campo magnético alcanza el idóneo para la fusión nuclear}}, Noticia en Europapress (18 de Septiembre, 2018). Artículo: D. Nakamura, A. Ikeda, H. Sawabe, et al., \href{https://doi.org/10.1063/1.5044557}{\textit{Record indoor magnetic field of $1200 T$ generated by electromagnetic flux-compression}}, \textsl{Rev. Sci. Instrum.} \textbf{89}, 095106 (2018).



\bibitem{Nolting} W. Nolting, {\it Grundkurs Theoretische Physik 3: Electrodynamik}, 7. Auflage, Springer, (2004).

\bibitem{Einstein07} A. Einstein, {\it \"Uber die Möglichkeit einer neuen Pr\"ufung des Relativitätsprinzips}, {\sl Annalen der Physik} {\bf 23} (1907) 197-198.

\bibitem{IS38} H.E. Ives and G.R. Stilwell, {\it An Experimental Study of the Rate of a Moving Atomic Clock}, {\sl J. Opt. Soc. Am.} {\bf 28} (1938) 215-219. \url{http://dx.doi.org/10.1364/JOSA.28.000215}.

\bibitem{Saathoff03} G. Saathoff, S. Karpuk et al., {\it Improved Test of Time Dilation in Special Relativity}, {\sl Phys.Rev.Lett.} {\bf 91} (2003) 190403. \url{http://link.aps.org/doi/10.1103/PhysRevLett.91.190403}.

\bibitem{Reinhardt07} S. Reinhardt, G. Saathoff et al., {\it Test of relativistic time dilation with fast optical atomic clocks at different velocities}, {\sl Nature} {\bf 3} (2007) 861-864. \url{http://www.nature.com/doifinder/10.1038/nphys778}.

\bibitem{Chou10} C. W. Chou, D. B. Hume, T. Rosenband and D. J. Wineland, {\em Optical Clocks and Relativity}, {\sl Science} {\bf 329} (2010) 1630-1633. \url{http://www.sciencemag.org/content/329/5999/1630.full.html}.

\bibitem{Bailey77} J. Bailey, K. Borer et al., {\it Measurements of relativistic time dilatation for positive and negative muons in a circular orbit}, {\sl Nature} {\bf 268} (1977) 301-305. \url{http://dx.doi.org/10.1038/268301a0}.

\bibitem{HK72a} J.C. Hafele and R. E. Keating. {\it Around-the-World Atomic Clocks: Predicted Relativistic Time Gains}, {\sl Science} {\bf 177} (1972) 166-168. \url{http://www.jstor.org/stable/1734833}.

\bibitem{HK72b} J.C. Hafele and R. E. Keating. {\it Around-the-World Atomic Clocks: Observed Relativistic Time Gains}, {\sl Science} {\bf 177} (1972) 168-170. \url{http://www.jstor.org/stable/1734834}.

\bibitem{SRexp} T. Roberts and S. Schleif, {\it What is the experimental basis of Special Relativity?},  \url{http://math.ucr.edu/home/baez/physics/Relativity/SR/experiments.html} (2007).

\bibitem{Minkowski07} H. Minkowski, \href{http://de.wikisource.org/wiki/Die_Grundgleichungen_f\%C3\%BCr_die_elektromagnetischen_Vorg\%C3\%A4nge_in_bewegten_K\%C3\%B6rpern}{\it Die Grundgleichungen f\"ur die elektromagnetischen Vorgänge in bewegten Körpern}, {\sl Nachrichten von der Gesellschaft der Wissenschaften zu Göttingen, Mathematisch-Physikalische Klasse} (1907) 53–111. Wikisource translation: \href{http://en.wikisource.org/wiki/The_Fundamental_Equations_for_Electromagnetic_Processes_in_Moving_Bodies}{\it The Fundamental Equations for Electromagnetic Processes in Moving Bodies}.

\bibitem{Einstein05} A. Einstein, \href{https://doi.org/10.1002\%2Fandp.19053231314}{\textit{Ist die trägheit eines Körpers von seinem Energieinhalt abhängig?}}, {\sl Annalen der Physik} {\bf 18} (1905) 639-641. Una traducción al inglés está disponible aquí: \href{{https://www.fourmilab.ch/etexts/einstein/E_mc2/e_mc2.pdf}}{\textit{Does the Inertia of a Body Depend on its energy Content?}}.

\bibitem{Einstein35} A. Einstein, \href{http://www.ams.org/journals/bull/2000-37-01/S0273-0979-99-00805-8/S0273-0979-99-00805-8.pdf}
{\textit{Elementary derivation of the equivalence of mass and energy}}, {\sl Bulletin of the American Mathematical Society} {\bf 37} (1935) 39-44.

\bibitem{Rainville05} S. Rainville \textit{et al.}, \href{https://doi.org/10.1038/4381096a}{\it World Year of Physics: A direct test of $E=mc^2$}, {\sl Nature} {\bf 438} (2005) 1096-1097.

\bibitem{Duerr08} S. Dürr \textit{et al.}, \href{https://doi.org/10.1126/science.1163233}{\it Ab Initio Determination of Light Hadron Masses}, {\sl Science} {\bf 322} (2008) 1224-1227

\bibitem{Compton23} A. Compton, \href{http://prola.aps.org/abstract/PR/v21/i5/p483_1}{\it A Quantum Theory of the Scattering of X-rays by Light Elements}, {\sl Phys. Rev.} {\bf 21} (1923) 483-502.

\bibitem{Planck06} M. Planck {\it Das Prinzip der Relativität un die Grundgleichungen der Mechanik}, {\sl Verhandlungen Deutsche Physikalische Gesellschaft} {\bf 8} (1906) 136-141. Transcripción de la versión original (en alemán) \href{http://wikilivres.ca/wiki/Das_Prinzip_der_Relativit\%C3\%A4t_und_die_Grundgleichungen_der_Mechanik}{aquí (wikilivres)}. Traducción al inglés  \href{http://en.wikisource.org/wiki/Translation:The_Principle_of_Relativity_and_the_Fundamental_Equations_of_Mechanics}{aquí (wikilivres)}.

\end{thebibliography}.
